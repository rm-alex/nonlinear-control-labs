\documentclass[a4paper,14pt]{extarticle}

\usepackage[T2A]{fontenc}
\usepackage[utf8]{inputenc}
\usepackage[english, russian]{babel}

\usepackage[left=30mm, right=10mm, top=20mm, bottom=20mm]{geometry}

\usepackage{tempora}
\usepackage{setspace}
\onehalfspacing

\usepackage{titlesec}
\titleformat{\section}[block]{\bfseries\centering\MakeUppercase}{\thesection.}{1em}{}
\titleformat{\subsection}[block]{\bfseries}{\thesubsection.}{1em}{}
\titleformat{\subsubsection}[block]{\bfseries}{\thesubsubsection.}{1em}{}

\renewcommand{\contentsname}{\hfill \textbf{СОДЕРЖАНИЕ} \hfill\null}

\usepackage{indentfirst}
\setlength{\parindent}{1.25cm}

\usepackage{amsmath, amsfonts, amssymb}
\usepackage{graphicx}
\usepackage{caption}
\usepackage{subcaption}
\usepackage{float}
\usepackage{tikz}
\usetikzlibrary{patterns}
\usepackage{cmap}
\usepackage{hyperref}
\usepackage{xcolor}
\usepackage{listings}

\definecolor{LightGray}{gray}{0.7}

\lstdefinestyle{code}{
    language=Python, % change if needed
    basicstyle=\small\ttfamily,
    numbers=left,
    numberstyle=\small\color{LightGray},
    stepnumber=1,
    numbersep=5pt,
    backgroundcolor=\color{white},
    showspaces=false,
    showstringspaces=false,
    showtabs=false,
    tabsize=4,
    captionpos=b,
    breaklines=true,
    breakatwhitespace=false,
    frame=single,
    rulecolor=\color{LightGray},
    linewidth=\linewidth,
    keywordstyle=\color{blue}\bfseries,
    commentstyle=\color{green!40!black},
    stringstyle=\color{violet},
    escapeinside={\%*}{*)},
    xleftmargin=10pt,
    xrightmargin=10pt,
    framexleftmargin=0pt,
    framexrightmargin=0pt
}
\lstset{style=code}

\hypersetup{
    colorlinks=true,
    linkcolor=blue,
    filecolor=magenta,
    urlcolor=cyan,
    pdftitle={ncs1},
    pdfauthor={Rumyantsev Alexey},
    pdfsubject={control},
    pdfkeywords={LaTeX, PDF},
    pdfpagemode=FullScreen,
}

\graphicspath{{src/images/}}

\begin{document}

\begin{titlepage}
    \begin{center}
        МИНИСТЕРСТВО НАУКИ И ВЫСШЕГО ОБРАЗОВАНИЯ РОССИЙСКОЙ ФЕДЕРАЦИИ\\
        \vspace*{2.5mm}
        Федеральное государственное автономное образовательное учреждение высшего образования
        «НАЦИОНАЛЬНЫЙ ИССЛЕДОВАТЕЛЬСКИЙ УНИВЕРСИТЕТ ИТМО»\\
        \vspace*{2.5mm}
        \textbf{ФАКУЛЬТЕТ СИСТЕМ УПРАВЛЕНИЯ И РОБОТОТЕХНИКИ}
        \vfill

        {\large ОТЧЕТ ПО ЛАБОРАТОРНОЙ РАБОТЕ №1}\\
        {\large по дисциплине}\\
        {\large\bfseries «НЕЛИНЕЙНЫЕ СИСТЕМЫ УПРАВЛЕНИЯ»}\\
        {\large на тему}\\
        {\large\bfseries «АНАЛИЗ И СТАБИЛИЗАЦИЯ НЕЛИНЕЙНЫХ ДИНАМИЧЕСКИХ СИСТЕМ»}\\
        \vfill

        \begin{flushright}
            Выполнил: студент гр. R3441\\
            Румянцев А. А.\medskip\\

            Проверил: преподаватель\\
            Зименко К. А.
        \end{flushright}

        \vfill

        Санкт-Петербург\\
        2025
    \end{center}
\end{titlepage}

\setcounter{page}{2}
\tableofcontents
\newpage

\section{Задание 1}
\subsection{Условие}
Для каждой из данных систем найти все точки равновесия.
На основе метода линеаризации в точке определить тип каждого
изолированного состояния равновесия. С использованием перехода
к полярным координатам определить устойчивость предельного
цикла для системы 4.


Численно построить фазовый портрет каждой системы и сравнить с полученными
результатами (кроме системы 7).


\subsection{Выполнение}
\subsubsection{Первая система}
Найдем точки равновесия системы (\ref{syseq:1})
\begin{align}
\begin{cases}
    \dot{x}_1=-x_1+2x_1^3+x_2,\\
    \dot{x}_2=-x_1-x_2
\end{cases}\label{syseq:1}
\end{align}


Приравняем выражения к нулю
и найдем корни системы
$$
\begin{cases}
    -x_1+2x_1^3+x_2=0,\\
    -x_1-x_2=0
\end{cases}\Rightarrow
\begin{cases}
    x_1=-x_2,\\
    x_2\left( 1-x_2^2 \right)=0
\end{cases}\Rightarrow
\left[
\begin{matrix}
    \left( 0;0 \right),\\
    \left( 1;-1 \right),\\
    \left( -1;1 \right)
\end{matrix}
\right.
$$


Определим тип каждого изолированного состояния равновесия.


Составим матрицу Якоби
$$
J=\dfrac{\partial f}{\partial x}=\begin{bmatrix}
    \dfrac{\partial \dot{x}_1}{\partial x_1} &\dfrac{\partial \dot{x}_1}{\partial x_2}\\
    \dfrac{\partial \dot{x}_2}{\partial x_1} &\dfrac{\partial \dot{x}_2}{\partial x_2}\\
\end{bmatrix}=
\begin{bmatrix}
    6x_1^2-1 &1\\
    -1 &-1
\end{bmatrix}
$$


Найдем матрицы $A_i$ и их собственные числа
$$
A_1=J|_{\left( 0;0 \right)}=\begin{bmatrix}
    -1 &1\\
    -1 &-1
\end{bmatrix},\ \lambda_{A_1\, 1,2}=\left\{ -1\pm i \right\},
$$
$$
A_2=J|_{\left( 1;-1 \right)}=\begin{bmatrix}
    5 &1\\
    -1 &-1
\end{bmatrix},\ \lambda_{A_2\, 1,2}=\left\{4.83, -0.83\right\},
$$
$$
A_3=J|_{\left( -1;1 \right)}=\begin{bmatrix}
    5 &1\\
    -1 &-1
\end{bmatrix},\ \lambda_{A_3\, 1,2}=\lambda_{A_2\, 1,2}
$$


Таким образом, 
\begin{align*}
    &(0;0):\lambda_{A_1\, 1,2}=\left\{ -1\pm i \right\} \text{ -- устойчивый фокус},\\
    &(1;-1),(-1;1):\lambda_{A_{2,3}\, 1,2}=\left\{4.83, -0.83\right\} \text{ -- седла (неуст.)}
\end{align*}


Седловые точки находятся рядом с устойчивым фокусом,
из-за чего некоторые фазовые траектории будут уходить на бесконечность
вдоль седловых направлений.


Таким образом, любой контур вокруг фокуса не может
полностью замкнуться, а значит предельного цикла не существует.


Численно построим фазовый портрет системы.


\begin{figure}[H]
    \centering
    \includegraphics[scale=0.75]{pps1.png}
    \caption{Фазовый портрет системы (\ref{syseq:1})}
    \label{fig:pps1}
\end{figure}


Точки равновесия совпадают с найденными ранее.
На графике получились устойчивый фокус и два седла.


\subsubsection{Вторая система}
Рассмотрим систему (\ref{syseq:2})
\begin{align}
    \begin{cases}
        \dot{x}_1=x_1+x_1x_2,\\
        \dot{x}_2=-x_2+x_2^2+x_1x_2-x_1^3
    \end{cases}\label{syseq:2}
\end{align}


Проведем аналогичные действия
$$\begin{cases}
    x_1+x_1x_2=0,\\
    -x_2+x_2^2+x_1x_2-x_1^3=0
\end{cases}\Rightarrow\begin{cases}
    x_1\left( x_2+1 \right)=0,\\
    -x_2-x_2^2+x_1x_2-x_1^3=0
\end{cases}$$


Получим две системы
$$
\begin{cases}
    x_1=0,\\
    x_2\left( x_2-1 \right)=0
\end{cases}\vee \ \ \ \begin{cases}
    x_2=-1,\\
    x_1^3+x_1-2=0
\end{cases}
$$


Решая первую систему, получим
$$
\begin{cases}
    x_1=0,\\
    x_2=0
\end{cases}\vee \ \ \ \begin{cases}
    x_1=0,\\
    x_2=1
\end{cases}
$$


Подставив $x_1=1$ во вторую систему, получим равенство нулю.
После деления полинома $x_1^3+x_1-2$ на $x_1-1$ получим
$$
\begin{cases}
    x_2=-1,\\
    \left( x_1-1 \right)\left( x_1^2+x_1+2 \right)=0
\end{cases}
$$


Второе выражение даст комплексные корни. Для анализа
системы и построения фазовых портретов берем только вещественные, тогда
$$
\begin{cases}
    x_1=1,\\
    x_2=-1
\end{cases}
$$


Таким образом, имеем точки равновесия системы
$$
\left[
\begin{matrix}
    \left( 0;0 \right),\\
    \left( 0;1 \right),\\
    \left( 1;-1 \right)
\end{matrix}
\right.
$$


Составим матрицу Якоби
$$
J=\dfrac{\partial f}{\partial x}=\begin{bmatrix}
    x_2+1 &x_1\\
    -3x_1^2+x_2 &x_1+2x_2-1
\end{bmatrix}
$$


Найдем матрицы $A_i$ и их собственные числа
$$
A_1=J|_{\left( 0;0\right)}=\begin{bmatrix}
    1 &0\\
    0 &-1
\end{bmatrix},\ \lambda_{A_1\,1,2}=\left\{ -1,1 \right\},
$$
$$
A_2=J|_{\left( 0;1\right)}=\begin{bmatrix}
    2 &0\\
    1 &1
\end{bmatrix},\ \lambda_{A_2\,1,2}=\left\{ 1,2 \right\},
$$
$$
A_3=J|_{\left( 1;-1\right)}=\begin{bmatrix}
    0     &1\\
    -4    &-2
\end{bmatrix},\ \lambda_{A_3\,1,2}=\left\{ -1 \pm 1.73i \right\}
$$


Таким образом,
\begin{align*}
    &\left( 0;0 \right):\lambda_{A_1\,1,2}=\left\{ -1,1 \right\}\text{ -- седло},\\
    &\left( 0;1 \right):\lambda_{A_2\,1,2}=\left\{ 1,2 \right\}\text{ -- неустойчивый узел},\\
    &\left( 1;-1 \right):\lambda_{A_3\,1,2}=\left\{ -1 \pm 1.73i \right\}\text{-- устойчивый фокус}
\end{align*}


Численно построим фазовый портрет системы.


\begin{figure}[H]
    \centering
    \includegraphics[scale=0.75]{pps2.png}
    \caption{Фазовый портрет системы (\ref{syseq:2})}
    \label{fig:pps2}
\end{figure}


Точки равновесия совпадают с найденными ранее.
На графике получились седло, неустойчивый узел и устойчивый фокус.


\subsubsection{Третья система}
Рассмотрим систему:
\begin{align}
\begin{cases}
    \dot{x}_1=x_2,\\
    \dot{x}_2=-x_1+x_2\left( 1-x_1^2 +0.1x_1^4\right)
\end{cases}   \label{syseq:3} 
\end{align}


Точки равновесия:
$$
\begin{cases}
    x_2=0,\\
    -x_1=0
\end{cases}\Rightarrow
\begin{matrix}
    \left( 0;0 \right)
\end{matrix}
$$


Матрица Якоби:
$$
J=\begin{bmatrix}
    0 &1\\
    -1+x_2\left( -2x_1+0.4x_1^3 \right) &1-x_1^2+0.1x_1^4
\end{bmatrix}
$$


Матрица $A$ и ее собственные числа:
$$
A=J|_{\left( 0;0 \right)}=\begin{bmatrix}
    0&1\\
    -1&1
\end{bmatrix},\ \lambda_{1,2}=\left\{ 0.5 \pm 0.87i \right\}
$$


Вывод: неустойчивый фокус. Действительная часть мнимной пары
больше нуля.


Фазовый портрет:
\begin{figure}[H]
    \centering
    \includegraphics[scale=0.75]{pps3.png}
    \caption{Фазовый портрет системы (\ref{syseq:3})}
    \label{fig:pps3}
\end{figure}


Точка равновесия совпадает с найденной ранее. На портрете неустойчивый фокус.


\subsubsection{Четвертая система}
Рассмотрим систему:
\begin{align}
    \begin{cases}
        \dot{x}_1=\left( x_1-x_2 \right)\left( 1-x_1^2-x_2^2 \right),\\
        \dot{x}_2=\left( x_1+x_2 \right)\left( 1-x_1^2-x_2^2 \right)
    \end{cases}\label{syseq:4}
\end{align}


Точки равновесия:
$$
\begin{cases}
    \left( x_1-x_2 \right)\left( 1-\left( x_1^2+x_2^2 \right) \right)=0,\\
    \left( x_1+x_2 \right)\left( 1-\left( x_1^2+x_2^2 \right) \right)=0
\end{cases}
$$


Рассмотрим случай $1-\left( x_1^2+x_2^2 \right)\neq0$:
$$
\begin{cases}
    1-\left( x_1^2+x_2^2 \right)\neq0,\\
    x_1-x_2=0,\\
    x_1+x_2=0
\end{cases}\Rightarrow\begin{cases}
    x_1=x_2,\\
    x_2=-x_2
\end{cases}\Rightarrow
\begin{matrix}
    \left( 0;0 \right)
\end{matrix}
$$


Рассмотрим случай $1-\left( x_1^2+x_2^2 \right)=0$:
$$
\begin{cases}
    1-\left( x_1^2+x_2^2 \right)=0,\\
    x_1-x_2\neq0,\\
    x_1+x_2\neq0
\end{cases}\Rightarrow\left\{ \left( x_1,x_2 \right)|\, x_1^2+x_2^2=1 \right\}
$$


В случае $x_1^2+x_2^2=1$ окружность единичного радиуса является множеством равновесий.


Матрица Якоби:
$$
J=\begin{bmatrix}
    -2x_1^2-\left( x_1-x_2 \right)^2+1&2x_2^2+\left( x_1-x_2 \right)^2-1\\
    -2x_1^2-\left( x_1+x_2 \right)^2+1&-2x_2^2-\left( x_1+x_2 \right)^2+1
\end{bmatrix}
$$


Матрица $A$ и ее собственные числа:
$$
A=J|_{\left(0;0\right)}=\begin{bmatrix}
    1&-1\\
    1&1
\end{bmatrix},\ \lambda_{1,2}=\left\{ 1\pm i \right\}
$$


Таким образом, в точке равновесия $\left( 0;0 \right)$
неустойчивый фокус. Действительная часть комплексной пары
больше нуля.


Перейдем к полярным координатам:
$$
r=\sqrt{x_1^2+x_2^2},\ \ \theta=\arctan{\dfrac{x_2}{x_1}},\ \
\begin{cases}
    x_1=r\cos{\theta},\\
    x_2=r\sin{\theta}
\end{cases}
$$


Приведем систему (\ref{syseq:4}) к виду:
$$
\dot{r}=\dfrac{x_1\dot{x}_1+x_2\dot{x}_2}{r}=r\left( 1-r^2 \right)
$$
% Эта формула получается при взятии производной от
% выражения $r$ с корнем.


% Домножим оба выражения на недостающие $x_i$:
% $$
% \begin{cases}
%     \dot{x}_1=\left( x_1-x_2 \right)\left( 1-r^2 \right) |\cdot x_1,\\
%     \dot{x}_2=\left( x_1+x_2 \right)\left( 1-r^2 \right) |\cdot x_2
% \end{cases}\Rightarrow\begin{cases}
%     x_1\dot{x}_1=x_1\left( x_1-x_2 \right)\left( 1-r^2 \right),\\
%     x_2\dot{x}_2=x_2\left( x_1+x_2 \right)\left( 1-r^2 \right)
% \end{cases}
% $$


% Сложим два выражения, вынесем общий множитель и упростим:
% $$
% x_1\dot{x}_1+x_2\dot{x}_2=\left( x_1\left( x_1-x_2 \right)+x_2\left( x_1+x_2 \right) \right)\left( 1-r^2 \right),
% $$
% $$
% \left( x_1\left( x_1-x_2 \right)+x_2\left( x_1+x_2 \right) \right)\left( 1-r^2 \right)
% =\left( x_1^2+x_2^2 \right)\left( 1-r^2 \right)=r^2\left( 1-r^2 \right)
% $$


% Разделим на $r$ и получим:
% $$
% \dot{r}=\dfrac{x_1\dot{x}_1+x_2\dot{x}_2}{r}=r\left( 1-r^2 \right)
% $$


Аналогично выведем производную полярного угла:
$$
\dot{\theta}=\dfrac{x_1\dot{x}_2-x_2\dot{x}_1}{r^2}=1-r^2
$$
% Эта формула получается при взятии производной от выражения $\theta$
% с арктангенсом.


% Приведем систему:
% $$
% \begin{cases}
%     \dot{x}_1=\left( x_1-x_2 \right)\left( 1-r^2 \right) |\cdot (-x_2),\\
%     \dot{x}_2=\left( x_1+x_2 \right)\left( 1-r^2 \right) |\cdot x_1
% \end{cases}\Rightarrow\begin{cases}
%     -x_2\dot{x}_1=-x_2\left( x_1-x_2 \right)\left( 1-r^2 \right),\\
%     x_1\dot{x}_2=x_1\left( x_1+x_2 \right)\left( 1-r^2 \right)
% \end{cases}
% $$
% $$
% x_1\dot{x}_2-x_2\dot{x}_1=\left( x_1\left( x_1+x_2 \right)-x_2\left( x_1-x_2 \right) \right)\left( 1-r^2 \right)=r^2\left( 1-r^2 \right)
% $$


% Разделим на $r^2$ и получим:
% $$
% \dot{\theta}=\dfrac{x_1\dot{x}_2-x_2\dot{x}_1}{r^2}=1-r^2
% $$


Найдем стационарные радиусы через радиальное уравнение:
$$
\dot{r}=r\left( 1-r^2 \right),\ r\left( 1-r^2 \right)=0\Rightarrow
\left[
\begin{matrix}
    r=0,\\
    r=1
\end{matrix}
\right.
$$


Получили равновесие в начале координат $r=0$ и окружность радиуса
$r=1$.


Чтобы окружность радиуса $r=1$ являлась предельным циклом,
скорости точек вблизи нее и на ней не должны быть равны нулю и,
соседние траектории должны стремиться к ней (устойчивый предельный цикл)
или отдаляться от нее (неустойчивый).


Подставим $r^2=x_1^2+x_2^2=1$ в систему (\ref{syseq:4}):
$$
\begin{cases}
        \dot{x}_1|_{x_1^2+x_2^2=1}=\left( x_1-x_2 \right)\left( 1-1 \right),\\
        \dot{x}_2|_{x_1^2+x_2^2=1}=\left( x_1+x_2 \right)\left( 1-1 \right)
    \end{cases}\Rightarrow\begin{cases}
        \dot{x}_1|_{x_1^2+x_2^2=1}=0,\\
        \dot{x}_2|_{x_1^2+x_2^2=1}=0
    \end{cases}
$$


Так как скорости точек на окружности равны нулю,
то все они являются равновесиями
-- т.е. в данной системе нет предельного цикла, но есть
множество равновесий на окружности радиуса $r=1$.


Проверим устойчивость множества равновесий по радиальному направлению.


Зафиксируем радиус равновесия $R=1$ и введем малое отклонение:
$$
s=r-R\Rightarrow r=s+R
$$


Введем обозначение:
$$
g(r):=\dot{r}=r\left( 1-r^2 \right)
$$


Рассмотрим производную малого отклонения по времени:
$$
\dot{s}=\dfrac{d}{dt}\left( r-R \right)=\dot{r}-\dot{R}=\dot{r}\Rightarrow \dot{s}=g\left( s+R \right)
$$


Разложим $g\left( R+s \right)$ в ряд Тейлора:
$$
g\left( s+R \right)=g(R)+\dot{g}(R)s+\dfrac{1}{2}\ddot{g}(R)s^2+\hdots\approx g(R)+\dot{g}(R)s+O(s^2)
$$


Первое приближение -- полагаем отклонение $s$ мало, т.е. $s^n\to0$.


Найдем $\dot{s}=g\left( s+R \right)$:
$$
g\left( R \right)=1\left( 1-1^2 \right)=0,\ \dot{g}(R)=1-3r^2|_{R=1}=-2\Rightarrow \dot{s}=g(s+R)\approx-2s
$$


Решая дифференциальное уравнение $\dot{s}(t)\approx-2s(t)$, получим:
$$
s\left( t \right)=Ce^{-2t},\ C=s\left( 0 \right)
$$


Рассмотрим предел малого отклонения при $t\to\infty$:
$$
\lim\limits_{t\to\infty}s(t)=\lim\limits_{t\to\infty}Ce^{-2t}=C\cdot0=0,
$$
$$
s=\left(r-R\right)\xrightarrow[t \to \infty]{}0
$$


Тогда радиус:
$$
r\xrightarrow[t \to \infty]{}R=1
$$


Все траектории стремятся к окружности радиуса $r=1$, следовательно
множество равновесий радиально устойчиво.


Фазовый портрет:
\begin{figure}[H]
    \centering
    \includegraphics[scale=0.75]{pps4.png}
    \caption{Фазовый портрет системы (\ref{syseq:4})}
    \label{fig:pps4}
\end{figure}


Точка равновесия совпадает с найденной ранее.
На портрете есть множество равновесий -- окружность радиуса 1.
Траектории притягиваются к ней и остаются неподвижны.


\subsubsection{Пятая система}
Рассмотрим систему:
\begin{align}
    \begin{cases}
        \dot{x}_1=-x_1^3+x_2,\\
        \dot{x}_2=x_1-x_2^3
    \end{cases}\label{syseq:5}
\end{align}


Точки равновесия:
$$
\begin{cases}
        -x_1^3+x_2=0,\\
        x_1-x_2^3=0
    \end{cases}\Rightarrow\begin{cases}
        x_1=x_2^3,\\
        x_2\left( 1-x_2^8 \right)=0
    \end{cases}\Rightarrow\left[
\begin{matrix}
    \left( 0;0 \right),\\
    \left( 1;1 \right),\\
    \left( -1;-1 \right)
\end{matrix}
\right.
$$


Выражение $x_2^8=1$ дает 8 корней -- 3 комплексные пары и 2 действительных числа.
Для анализа системы рассматриваем действительные точки равновесия, т.к. они обладают
физическим смыслом.


Матрица Якоби:
$$
J=\begin{bmatrix}
    -3x_1^2&1\\
    1&-3x_2^2
\end{bmatrix}
$$


Матрицы $A_i$ и их собственные числа:
$$
A_1=J|_{\left( 0;0\right)}=\begin{bmatrix}
    0 &1\\
    1 &0
\end{bmatrix},\ \lambda_{A_1\,1,2}=\left\{ -1,1 \right\},
$$
$$
A_2=J|_{\left( 1;1\right)}=\begin{bmatrix}
    -3 &1\\
    1 &-3
\end{bmatrix},\ \lambda_{A_2\,1,2}=\left\{ -4,-2 \right\},
$$
$$
A_3=J|_{\left( -1;-1\right)}=A_2,\ \lambda_{A_3\,1,2}=\lambda_{A_2\,1,2}
$$


Таким образом:
\begin{align*}
    &\left( 0;0 \right):\lambda_{A_1\,1,2}=\left\{ -1,1 \right\}\text{ -- седло},\\
    &\left( 1;1 \right),\left( -1,-1 \right):\lambda_{A_{2,3}\,1,2}=\left\{ -4,-2 \right\}\text{ -- устойчивые узлы}
\end{align*}


Фазовый портрет:
\begin{figure}[H]
    \centering
    \includegraphics[scale=0.75]{pps5.png}
    \caption{Фазовый портрет системы (\ref{syseq:5})}
    \label{fig:pps5}
\end{figure}


Точки равновесия совпадают с найденными ранее. На графике два устойчивых
узла и седло.


\subsubsection{Шестая система}
Рассмотрим систему:
\begin{align}
    \begin{cases}
        \dot{x}_1=-x_1^3+x_2^3,\\
        \dot{x}_2=x_2^3x_1-x_2^3
    \end{cases}\label{syseq:6}
\end{align}


Точки равновесия:
$$
\begin{cases}
    -x_1^3+x_2^3=0,\\
    x_2^3x_1-x_2^3=0
\end{cases}\Rightarrow
\begin{cases}
    x_2=x_1,\\
    x_1^3\left( x_1-1 \right)=0
\end{cases}\Rightarrow\left[
\begin{matrix}
    \left( 0;0 \right),\\
    \left( 1;1 \right)
\end{matrix}
\right.
$$


Матрица Якоби:
$$
J=\begin{bmatrix}
    -3x_1^2&3x_2^2\\
    x_2^3&3x_2^2\left(x_1-1\right)
\end{bmatrix}
$$


Матрицы $A_i$ и их собственные числа:
$$
A_1=J|_{\left( 0;0\right)}=\begin{bmatrix}
    0 &0\\
    0 &0
\end{bmatrix},\ \lambda_{A_1\,1,2}=\left\{ 0,0 \right\},
$$
$$
A_2=J|_{\left( 1;1\right)}=\begin{bmatrix}
    -3 &3\\
    1 &0
\end{bmatrix},\ \lambda_{A_2\,1,2}=\left\{ -3.79,0.79 \right\}
$$


Точка $\left( 0;0 \right)$ является негиперболической (вещественные части собственных чисел нулевые).


Исследуем эту точку на устойчивость с помощью функции Ляпунова:
$$
V=\dfrac{1}{2}\left( x_1^2+x_2^2 \right),\ \dot{V}=x_1\dot{x}_1+x_2\dot{x}_2    
$$
$$
\dot{V}=x_1\left( -x_1^3+x_2^3 \right)+x_2\left( x_2^3\left( x_1-1 \right) \right)=-x_1^4-x_2^4+x_1x_2^3+x_1x_2^4
$$


Воспользуемся неравенством Юнга для произведения:
$$
ab\leq \dfrac{a^p}{p}+\dfrac{b^q}{q},\ a,b\geq0,\ \dfrac{1}{p}+\dfrac{1}{q}=1
$$


Берем $p=4,q=4/3$:
$$
|x_1||x_2|^3\leq\dfrac{1}{4}|x_1|^4+|x_2|^{3\cdot4/3}\div4/3=\dfrac{1}{4}x_1^4+\dfrac{3}{4}x_2^{4}
$$


По определению:
$$
f(x)=O\left( r^n \right)\text{ при }r\to0\text{, если } |f(x)|\leq Cr^n, C>0, r \text{ малы}
$$


При $r^2=x_1^2+x_2^2$:
$$
|x_1|\leq r,|x_2|\leq r\Rightarrow |x_1x_2^4|\leq |x_1|\cdot|x_2|^4\leq r\cdot r^4=r^5
$$


Следовательно:
$$
x_1x_2^4=O\left( r^5 \right) \text{ при }r\to0
$$


Производная функции Ляпунова примет вид:
$$
\dot{V}\leq -x_1^4-x_2^4+\left( \frac{1}{4}x_1^4+\frac{3}{4} x_2^4\right)+O\left( r^5 \right)
$$
$$
\dot{V}\leq-\frac{3}{4}x_1^4-\frac{1}{4}x_2^4+O\left( r^5 \right)
$$


При $r\to0$ степенные члены отрицательны, $O\left( r^5 \right)$ мало:
$
\dot{V}<0
$.


Таким образом:
\begin{align*}
    &\left( 0;0 \right):\lambda_{A_1\,1,2}=\left\{ 0,0 \right\}\text{ -- локально асимптотически устойчивая}\\
    &\ \ \ \ \ \ \ \ \ \ \ \ \ \ \ \ \ \ \ \ \ \ \ \ \ \ \ \ \ \ \ \ \ \ \ \ \ \ \ \ \ \ \text{негиперболическая точка},\\
    &\left( 1;1 \right):\lambda_{A_2\,1,2}=\left\{ -3.79,0.79 \right\}\text{ -- седло}
\end{align*}


Фазовый портрет:
\begin{figure}[H]
    \centering
    \includegraphics[scale=0.75]{pps6.png}
    \caption{Фазовый портрет системы (\ref{syseq:6})}
    \label{fig:pps6}
\end{figure}


Точки равновесия совпадают с найденными ранее.
На графике седло и локально асимптотически устойчивая
негиперболическая точка.


\subsubsection{Седьмая система}
Рассмотрим систему:
\begin{align}
    \begin{cases}
        \dot{x}_1=-x_1^3+x_2^3,\\
        \dot{x}_2=x_1+3x_3-x_2^3,\\
        \dot{x}_3=x_1x_3-x_2^3-\sin{x_1}
    \end{cases}\label{syseq:7}
\end{align}


Точки равновесия:
$$
\begin{cases}
    -x_1^3+x_2^3=0,\\
    x_1+3x_3-x_2^3=0,\\
    x_1x_3-x_2^3-\sin{x_1}=0
\end{cases}\Rightarrow\begin{cases}
    x_2=x_1,\\
    x_3=3^{-1}x_1\left( x_1^2-1 \right),\\
    x_1^4-3x_1^3-x_1^2-3\sin{\left( x_1 \right)}=0
\end{cases}\Rightarrow
$$
$$
\Rightarrow\left[
\begin{matrix}
    \left( 0;0;0 \right),\\
    \left( 3.29,3.29,10.79 \right)
\end{matrix}
\right.
$$


Матрица Якоби:
$$
J=\begin{bmatrix}
    -3x_1^2&3x_2^2&0\\
    1&-3x_2^2&3\\
    x_3-\cos{\left( x_1 \right)}&-3x_2^2&x_1
\end{bmatrix}
$$


Матрицы $A_i$ и их собственные числа:
$$
A_1=J|_{\left( 0;0;0 \right)}=\begin{bmatrix}
    0     &0     &0\\
     1     &0     &3\\
    -1     &0     &0
\end{bmatrix},\ \lambda_{A_1\, 1,2}=\left\{ 0,0,0 \right\}
$$
$$
A_2=J|_{\left( 3.29,3.29,10.79 \right)}=\begin{bmatrix}
    -32.5&32.5&0\\
    1&-32.5&3\\
    11.78&-32.5&3.29
\end{bmatrix},
$$
$$
\lambda_{A_2\, 1,2}=\left\{ 1.36,-30.36,-32.71 \right\}
$$


Точка $\left( 0;0;0 \right)$ негиперболическая.


Возьмем функцию Четаева:
$$
V\left( x_1 \right)=\frac{1}{2}x_1^2,\ V\left( 0 \right)=0,\ V\left( x_1\neq0 \right)>0
$$


Ее производная:
$$
\dot{V}=x_1\dot{x}_1=x_1\left( -x_1^3+x_2^3 \right)=-x_1^4+x_1x_2^3
$$


Рассмотрим область $x_1>0,x_2>0$ вблизи нуля, тогда
$x_1x_2^3>0$, $-x_1^4$ меньшего порядка $\Rightarrow$
$\dot{V}>0$ в некоторой окрестности нуля,
т.е. существует область,
в которой производная функции Четаева положительна,
значит точка равновесия неустойчива (вблизи равновесия вдоль $|x_1|$ траектории уходят от точки равновесия).


Таким образом,
\begin{align*}
    &\left( 0;0;0 \right):\lambda_{A_1\,1,2}=\left\{ 0,0,0 \right\}\text{ -- негиперболическая неустойчивая точка},\\
    &\left( 3.29,3.29,10.79 \right):\lambda_{A_2\,1,2}=\left\{ 1.36,-30.36,-32.71 \right\}\text{ -- седло}
\end{align*}


\section{Задание 2}
\subsection{Условие}
Для каждой из данных систем найти все изолированные
точки равновесия и построить локально стабилизирующий
регулятор.


Представить результаты численного моделирования.


\subsection{Выполнение}
\subsubsection{Первая система}
Рассмотрим систему:
\begin{align}
    \begin{cases}
        \dot{x}_1=-x_1+2x_1^3+x_2+\sin{\left(u_1\right)},\\
        \dot{x}_2=-x_1-x_2+3\sin{\left(u_2\right)}
    \end{cases}
\end{align}


Точки равновесия:
$$\begin{cases}
    -x_1+2x_1^3+x_2+\sin{\left(u_1\right)}=0,\\
    -x_1-x_2+3\sin{\left(u_2\right)}=0
\end{cases}\Rightarrow\left[
\begin{matrix}
    \left( 0;0 \right),\\
    \left( 1;-1 \right),\\
    \left( -1;1 \right)
\end{matrix}
\right.$$


Линейное локальное управление:
$$
u-u_{ss}=-K\left( x-x_{ss} \right),
$$
где $x_{ss}$ -- точка равновесия, $u_{ss}$
-- управляющее воздействие, которое удерживает систему в равновесии.


Найдем $u_{ss}$ при $\dot{x}_1=0,\dot{x}_2=0$ в точке равновесия $\left( -1;1 \right)$:
$$
\begin{cases}
    1-2+1+\sin{\left(u_1\right)}=0,\\
    1-1+3\sin{\left(u_2\right)}=0
\end{cases}\Rightarrow\begin{cases}
    \sin{u_1}=0,\\
    \sin{u_2}=0
\end{cases}\Rightarrow\begin{cases}
    u_1^{ss}=\pi n, n\in\mathbb{Z},\\
u_2^{ss}=\pi n, n\in \mathbb{Z}
\end{cases}
$$


При условии $u_{ss}\neq0$ выберем:
$$
u_{ss}=\begin{bmatrix}
    u_{1}^{ss}\\u_{2}^{ss}
\end{bmatrix}=\begin{bmatrix}
    \pi\\\pi
\end{bmatrix}
$$


Обозначим:
$$
\tilde{x}=x-x_{ss},\ \tilde{u}=u-u_{ss}
$$


Составим систему:
$$
\dot{\tilde{x}}=A\tilde{x}+B\tilde{u}
$$


Найдем матрицы $A,B$:
$$
A=\dfrac{\partial f}{\partial x}\bigg|_{x_{ss},u_{ss}}=\begin{bmatrix}
    -1+6x_1^2&1\\-1&-1
\end{bmatrix}\Bigg|_{x_{ss},u_{ss}}=\begin{bmatrix}
    5&1\\-1&-1
\end{bmatrix},
$$
$$
B=\dfrac{\partial f}{\partial u}\bigg|_{x_{ss},u_{ss}}=\begin{bmatrix}
    \cos{\left( u_1 \right)}&0\\0&3\cos{\left( u_2 \right)}
\end{bmatrix}\Bigg|_{x_{ss},u_{ss}}=\begin{bmatrix}
    -1&0\\0&-3
\end{bmatrix}
$$


Собственные числа системы:
$$
\sigma\left( A \right)=\left\{ 4.8284,-0.8284 \right\}
$$


Система неустойчива. Нужно управление. Проверим управляемость:
$$
A=PJP^{-1},\ J=\begin{bmatrix}
    -0.8284        & 0\\
         0   & 4.8284
\end{bmatrix},\ B_J=P^{-1}B=\begin{bmatrix}
    -0.18   &-3.09\\
    0.18    &0.09
\end{bmatrix}
$$


Система полностью управляема (неустойчивому собственному значению соответствуют
ненулевые элементы матрицы $B_J$).


Синтезируем локально стабилизирующий
регулятор по состоянию:
$$
\tilde{u}=-K\tilde{x},\ K=\begin{bmatrix}
    k_1&k_2\\k_3&k_4
\end{bmatrix},\ \dot{\tilde{x}}=\left( A-BK \right)\tilde{x},
$$
$$
A-BK=\begin{bmatrix}
    5&1\\-1&-1
\end{bmatrix}-\begin{bmatrix}
    -1&0\\0&-3
\end{bmatrix}\begin{bmatrix}
     k_1&k_2\\k_3&k_4
\end{bmatrix}=\begin{bmatrix}
  k_1+5	  &k_2+1\\
3k_3-1	&3k_4-1
\end{bmatrix}
$$


Условия устойчивости:
$$
\begin{cases}
    \operatorname{trace}\left( A-BK \right)<0,\\
    \det{\left( A-BK \right)}>0,\\
    \Re{\left( \lambda_i \right)}<0
\end{cases}$$
$$
\begin{cases}
    \operatorname{trace}\left( A-BK \right)=k_1+3k_4+4<0,\\
    \det{\left( A-BK \right)}=-k_1+k_2-3k_2k_3-3k_3+3k_1k_4+15k_4-4>0
\end{cases}
$$


Пусть:
$$
K=\begin{bmatrix}
    -6&0\\0&0.5
\end{bmatrix}\Rightarrow\begin{cases}
    -6+3\cdot0.5+4=-0.5<0,\\
    6-3\cdot6\cdot0.5+15\cdot0.5-4=0.5>0
\end{cases}
$$


Собственные числа замкнутой системы:
$$
\sigma\left( A-BK \right)=\left\{  -0.25 \pm 0.66i \right\}
$$


Схема моделирования:
\begin{figure}[H]
    \centering
    \includegraphics[scale=0.4]{sch1.png}
    \caption{Схема моделирования системы}
    \label{fig:sch1}
\end{figure}


В схеме учтено смещение $u=u_{ss}-K\left( x-x_{ss} \right)$.


Начальные условия:
$$x_0=\begin{bmatrix}
    -0.9\\1.1
\end{bmatrix}$$


Графики вектора состояния системы и управления: % 1001,205,953,498
\begin{figure}[H]
    \centering
    \includegraphics[scale=0.6]{sch1m.png}
    \caption{Вектор состояния объекта управления}
    \label{fig:sch1m}
\end{figure}
\begin{figure}[H]
    \centering
    \includegraphics[scale=0.6]{sch1m2.png}
    \caption{Управление $u=u_{ss}-K\left( x-x_{ss} \right)$}
    \label{fig:sch1m2}
\end{figure}


Система стабилизировалась в точке равновесия $\left( -1;1 \right)$.


\subsubsection{Вторая система}
Рассмотрим систему:
\begin{align}
    \begin{cases}
        \dot{x}_1=x_2+x_1x_2+u^3,\\
        \dot{x}_2=-x_2+x_2^2-x_1^3+\sin{\left( u \right)}
    \end{cases}
\end{align}


Точки равновесия:
$$
\begin{cases}
    x_2+x_1x_2+u^3=0,\\
    -x_2+x_2^2-x_1^3+\sin{\left( u \right)}=0
\end{cases}\Rightarrow\left( 0;0 \right)
$$


Найдем $u_{ss}$ в точке равновесия:
$$
\begin{cases}
    u^3=0,\\\sin{\left( u \right)}=0
\end{cases}\Rightarrow\begin{cases}
    u=0,\\u=\pi n, n\in\mathbb{Z}
\end{cases}\Rightarrow u_{ss}=0 
$$


Составим систему. Так как $x_{ss}=\left( 0;0 \right)$ и $u_{ss}=0$:
$$
\dot{x}=Ax+Bu,\ u=-Kx
$$


Найдем матрицы $A,B$:
$$
A=\dfrac{\partial f}{\partial x}\bigg|_{x_{ss},u_{ss}}=\begin{bmatrix}
    x_2&1+x_1\\-3x_1^2&-1+2x_2
\end{bmatrix}\Bigg|_{x_{ss},u_{ss}}=\begin{bmatrix}
    0&1\\0&-1
\end{bmatrix},
$$
$$
B=\dfrac{\partial f}{\partial u}\bigg|_{x_{ss},u_{ss}}=\begin{bmatrix}
    3u^2\\\cos{\left( u \right)}
\end{bmatrix}\Bigg|_{x_{ss},u_{ss}}=\begin{bmatrix}
    0\\1
\end{bmatrix}
$$


Собственные числа системы:
$$
\sigma\left( A \right)=\left\{ 0,-1 \right\}
$$


Система устойчива, не асимптотически.


Управляемость:
$$
J=\begin{bmatrix}
    0    & 0\\
     0    &-1
\end{bmatrix},\ B_J=\begin{bmatrix}
    1\\1
\end{bmatrix}
$$


Система полностью управляема.


Синтезируем регулятор:
$$
K=\begin{bmatrix}
    k_1&k_2
\end{bmatrix},\ \dot{x}=\left( A-BK \right)x,
$$
$$
A-BK=\begin{bmatrix}
     0&1\\0&-1
\end{bmatrix}-\begin{bmatrix}
    0\\1
\end{bmatrix}\begin{bmatrix}
    k_1&k_2
\end{bmatrix}=\begin{bmatrix}
   0	    & 1\\
-k_1	&-k_2-1
\end{bmatrix}
$$


Условия устойчивости:
$$
\begin{cases}
    \operatorname{trace}\left( A-BK \right)<0,\\
    \det{\left( A-BK \right)}>0,\\
    \Re{\left( \lambda_i \right)}<0
\end{cases}\Rightarrow\begin{cases}
    -k_2-1<0,\\
    k_1>0
\end{cases}
$$


Пусть:
$$
K=\begin{bmatrix}
    2&2
\end{bmatrix}\Rightarrow\begin{cases}
    -2-1=-3<0,\\
    2>0
\end{cases}
$$


Собственные числа:
$$
\sigma\left( A-BK \right)=\left\{ -1,-2 \right\}
$$


Схема моделирования аналогична рис. (\ref{fig:sch1}).
В блоке $u_{ss}$ константа 0, в $x_{ss}$ нулевой вектор.


Начальные условия:
$$
x_0=\begin{bmatrix}
    0.2\\-0.2
\end{bmatrix}
$$


Графики вектора состояния системы и управления:
\begin{figure}[H]
    \centering
    \includegraphics[scale=0.6]{sch2m.png}
    \caption{Вектор состояния объекта управления}
    \label{fig:sch2m}
\end{figure}
\begin{figure}[H]
    \centering
    \includegraphics[scale=0.6]{sch2m2.png}
    \caption{Управление $u=-K\left( x \right)$}
    \label{fig:sch2m2}
\end{figure}


Система стабилизировалась в точке равновесия $\left( 0;0 \right)$.


\section{Вывод}
В ходе выполнения лабораторной работы
для различных нелинейных систем
были найдены точки равновесия
и определены их типы
методом линеаризации.
Для одной из систем был исследован
предельный цикл -- в результате
был сделан вывод, что это множество равновесий.
Были построены фазовые портреты систем.
Результаты моделирования подтвердили вычисления.
Были синтезированы локально стабилизирующие
регуляторы для нелинейных систем и проверены
моделированием -- регуляторы синтезированы корректно.


% \appendix
% \renewcommand{\thesection}{\Asbuk{section}}

% \section{Приложение}

\end{document}
