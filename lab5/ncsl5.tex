\documentclass[a4paper,14pt]{extarticle}

\usepackage[T2A]{fontenc}
\usepackage[utf8]{inputenc}
\usepackage[english, russian]{babel}

\usepackage[left=30mm, right=10mm, top=20mm, bottom=20mm]{geometry}

\usepackage{tempora}
\usepackage{setspace}
\onehalfspacing

\usepackage{titlesec}
\titleformat{\section}[block]{\bfseries\centering\MakeUppercase}{\thesection.}{1em}{}
\titleformat{\subsection}[block]{\bfseries}{\thesubsection.}{1em}{}
\titleformat{\subsubsection}[block]{\bfseries}{\thesubsubsection.}{1em}{}

\renewcommand{\contentsname}{\hfill \textbf{СОДЕРЖАНИЕ} \hfill\null}

\usepackage{indentfirst}
\setlength{\parindent}{1.25cm}

\usepackage{amsmath, amsfonts, amssymb}
\usepackage{graphicx}
\usepackage{caption}
\usepackage{subcaption}
\usepackage{float}
\usepackage{tikz}
\usetikzlibrary{patterns}
\usepackage{cmap}
\usepackage{hyperref}
\usepackage{xcolor}
\usepackage{listings}

\definecolor{LightGray}{gray}{0.7}

\lstdefinestyle{code}{
    language=Python, % change if needed
    basicstyle=\small\ttfamily,
    numbers=left,
    numberstyle=\small\color{LightGray},
    stepnumber=1,
    numbersep=5pt,
    backgroundcolor=\color{white},
    showspaces=false,
    showstringspaces=false,
    showtabs=false,
    tabsize=4,
    captionpos=b,
    breaklines=true,
    breakatwhitespace=false,
    frame=single,
    rulecolor=\color{LightGray},
    linewidth=\linewidth,
    keywordstyle=\color{blue}\bfseries,
    commentstyle=\color{green!40!black},
    stringstyle=\color{violet},
    escapeinside={\%*}{*)},
    xleftmargin=10pt,
    xrightmargin=10pt,
    framexleftmargin=0pt,
    framexrightmargin=0pt
}
\lstset{style=code}

\hypersetup{
    colorlinks=true,
    linkcolor=blue,
    filecolor=magenta,
    urlcolor=cyan,
    pdftitle={ncs3},
    pdfauthor={Rumyantsev Alexey},
    pdfsubject={control},
    pdfkeywords={LaTeX, PDF},
    pdfpagemode=FullScreen,
}

\graphicspath{{src/images/}}

\begin{document}

\begin{titlepage}
    \begin{center}
        МИНИСТЕРСТВО НАУКИ И ВЫСШЕГО ОБРАЗОВАНИЯ РОССИЙСКОЙ ФЕДЕРАЦИИ\\
        \vspace*{2.5mm}
        Федеральное государственное автономное образовательное учреждение высшего образования
        «НАЦИОНАЛЬНЫЙ ИССЛЕДОВАТЕЛЬСКИЙ УНИВЕРСИТЕТ ИТМО»\\
        \vspace*{2.5mm}
        \textbf{ФАКУЛЬТЕТ СИСТЕМ УПРАВЛЕНИЯ И РОБОТОТЕХНИКИ}
        \vfill

        {\large ОТЧЕТ ПО ЛАБОРАТОРНОЙ РАБОТЕ №5}\\
        {\large по дисциплине}\\
        {\large\bfseries <<НЕЛИНЕЙНЫЕ СИСТЕМЫ УПРАВЛЕНИЯ>>}\\
        {\large на тему}\\
        {\large\bfseries <<СИНТЕЗ РАЗРЫВНОГО И НЕПРЕРЫВНОГО СТАБИЛИЗИРУЮЩИХ РЕГУЛЯТОРОВ НА ОСНОВЕ СКОЛЬЗЯЩИХ РЕЖИМОВ>>}\\
        \vfill

        \begin{flushright}
            Выполнили: студенты\\
            Румянцев А. А., R3441\\
            Дьячихин Д. Н., R3480\medskip\\

            Проверил: преподаватель\\
            Зименко К. А.
        \end{flushright}

        \vfill

        Санкт-Петербург\\
        2025
    \end{center}
\end{titlepage}

\setcounter{page}{2}
\tableofcontents
\newpage
\section{Задание 1}
\subsection{Условие}
Рассмотрим систему:
$$
\begin{cases}
    \dot{x}_1=x_2+\sin{x_1},\\
    \dot{x}_2=\theta_1x_1^2+\left( 2+\theta_2 \right)u,
\end{cases}
$$
где $|\theta_1|\leq1,|\theta_2|\leq1$. Весь вектор
состояния измерим. Необходимо:
\begin{enumerate}
    \item синтезировать стабилизирующий разрывный регулятор на основе скользящих режимов;
     \item синтезировать стабилизирующий непрерывный регулятор на основе скользящих режимов;
	\item провести соответствующий анализ устойчивости;
	\item провести математическое моделирование.
\end{enumerate}


\subsection{Выполнение}
Выберем скользящую поверхность,
на которой упростим динамику системы
и зададим ей желаемые свойства:
$$
s=ax_1+x_2=0,\ a>0
$$


На поверхности $s=0$:
$$
x_2=-ax_1,\ x_2=\dot{x}_1-\sin{x_1}\Rightarrow \dot{x}_1=-ax_1+\sin{x_1}
$$


При $V=0.5x_1^2$:
$$
\dot{V}=x\dot{x}=-ax_1^2+x_1\sin{x_1}
$$


Так как $|\sin{x_1}|\leq|x_1|\,\forall x_1$:
$$
\dot{V}\leq-ax_1^2+|x_1||\sin{x_1}|\leq-ax_1^2+x_1^2=-(a-1)x_1^2=-2\left( a-1 \right)V
$$


Для асимптотической устойчивости $a>1\Rightarrow \dot{V}<0,\lim\limits_{t\to\infty}x_1(t)=0$.


Динамика скользящей переменной:
$$
s=ax_1+x_2\Rightarrow \dot{s}=a\dot{x}_1+\dot{x}_2
$$


Подставим $\dot{x}_1,\dot{x}_2$:
$$
\dot{s}=a\left( x_2+\sin{x_1} \right)+\theta_1x_1^2+\left( 2+\theta_2 \right)u,
$$
$$
\dot{s}=\left( 2+\theta_2 \right)u+\Delta(x),\ \Delta(x)\equiv a\left( x_2+\sin{x_1} \right)+\theta_1x_1^2
$$


Необходимо выбрать $u$ так, чтобы
выполнялось условие достижения скользящего режима:
$$
s\dot{s}\leq-\eta|s|,\ \eta>0
$$


Это эквивалентно:
$$
\dot{V}\leq-\eta|s|,\ V=0.5s^2
$$


Синтезируем дискретный регулятор на основе скользящих режимов.


Выберем управление как сумму эквивалентной и переключающей составляющих:
$$
u=u_\text{eq}+u_\text{sw}
$$


Эквивалентная часть при $s=0$:
$$
u_\text{eq}=-\frac{\Delta(x)}{2+\theta_2}
$$


Параметры $\theta_{1},\theta_{2}$ неизвестны,
поэтому положим их нулевыми.


Такое допущение возможно вследствие
наличия переключающей части $\beta(x)\operatorname{sign}(s)$ с
коэффициентом переключения $\beta_0$,
который при верном подборе компенсирует влияние отклонений $\theta_i$.


Кроме того, скользящий режим является робастным
к параметрическим неопределенностям и возмущениям.


Тогда, номинальная эквивалентная часть:
$$
u_\text{eq,nom}=-\frac{a}{2}\left( x_2+\sin{x_1} \right)
$$


Добавим переключающую часть:
$$
u_\text{sw}=-\beta(x)\operatorname{sign}(s)
$$


Разрывный регулятор:
$$
u=-\frac{a}{2}\left( x_2+\sin{x_1} \right)-\beta(x)\operatorname{sign}(s)
$$


Подставим $u$ в $\dot{s}$:
$$
\dot{s}=a\left( x_2+\sin{x_1} \right)+\theta_1x_1^2+\left( 2+\theta_2 \right)\left( -\frac{a}{2}\left( x_2+\sin{x_1} \right)-\beta(x)\operatorname{sign}(s) \right),
$$
% $$
% \dot{s}=\theta_1x_1^2-\frac{2+\theta_2}{2}\left( ax_2+a\sin{x_1} \right)+ax_2+a\sin{x_1}-\left( 2+\theta_2 \right)\beta(x)\operatorname{sign}(s),
% $$
$$
\dot{s}=-\left( 2+\theta_2 \right)\beta(x)\operatorname{sign}(s)+\theta_1x_1^2-\frac{\theta_2}{2}a\left( x_2+\sin{x_1} \right),
$$
$$
\dot{s}=-\left( 2+\theta_2 \right)\beta(x)\operatorname{sign}(s)+\delta(x)
$$


Оценим модуль <<невязки>> $\delta(x)$.
Получим верхнюю оценку при $|\theta_1|\leq1,|\theta_2|\leq1,|\sin{x_1}|\leq|x_1|$:
$$
|\delta(x)|=\left|\theta_1x_1^2-\frac{\theta_2}{2}a\left( x_2+\sin{x_1} \right)\right|\leq |x_1|^2+\frac{a}{2}\left( |x_2|+|x_1| \right)\equiv \rho(x)
$$


Тогда:
$$
\dot{s}\leq -\left( 2+\theta_2 \right)\beta(x)\operatorname{sign}(s)+\rho(x)
$$


Умножим выражение на $s$, учтем $\operatorname{sign}(s)\cdot s=|s|$
и оценим верхнюю границу:
$$
s\dot{s}\leq -\left( 2+\theta_2 \right)\beta(x)|s|+\rho(x)s\leq -\left( 2+\theta_2 \right)\beta(x)|s|+|\rho(x)||s|
$$


Вынесем общий множитель $-|s|$, положим $g_0:=\min{\left( 2+\theta_2 \right)}=1$
и оценим верхнюю границу:
$$
s\dot{s}\leq -\left( \left( 2+\theta_2 \right)\beta(x)-|\rho(x)| \right)|s|\leq -\left( g_0\beta(x)-|\rho(x)| \right)|s|,
$$
$$
s\dot{s}\leq -\left( \beta(x)-|\rho(x)| \right)|s|
$$


Если выбрать $\beta(x)\geq |\rho(x)|+\beta_0,\ \beta_0>0$:
$$
s\dot{s}\leq -\beta_0|s|<0\,\forall s\neq0,\ \beta_0>0
$$


Это аналогично записи через функцию Ляпунова и ее производную:
$$
V=0.5s^2,\ \dot{V}=s\dot{s}\leq-\beta_0|s|=-\beta_0\sqrt{2V}<0\,\forall s\neq0,\ \beta_0>0,
$$
что гарантирует достижение $s=0$ за конечное время.


Выберем:
$$
\beta(x)\geq|x_1|^2+\frac{a}{2}\left( |x_2|+|x_1| \right)+\beta_0,\ \beta_0>0
$$


Разрывный закон управления:
$$
u=-\frac{a}{2}\left( x_2+\sin{x_1} \right)-\left(|x_1|^2+\frac{a}{2}\left( |x_2|+|x_1| \right)+\beta_0\right)\operatorname{sign}(s)
$$


Синтезируем непрерывный регулятор на основе скользящих режимов.


Для уменьшения колебаний
из-за задержки переключения управления,
заменим в управлении функцию $\operatorname{sign}$
функцией насыщения:
$$
\operatorname{sat}\left( \frac{s}{\varepsilon} \right)=
\begin{cases}
    s/\varepsilon, &|s|\leq\varepsilon,\\
    \operatorname{sign}(s), &|s|>\varepsilon
\end{cases}
$$


Непрерывный регулятор:
$$
u=-\frac{a}{2}\left( x_2+\sin{x_1} \right)-\beta(x)\operatorname{sat}\left( \frac{s}{\varepsilon} \right)
$$


Аналогично подставим $u$ в $\dot{s}$, оценим модуль невязки $\delta(x)$ абсолютной величиной
$\rho(x)$ и коэффициент при $\beta(x)$ константой $g_0$:
$$
\dot{s}=-\left( 2+\theta_2 \right)\beta(x)\operatorname{sat}\left( \frac{s}{\varepsilon} \right)+\theta_1x_1^2-\frac{\theta_2}{2}\left( ax_2+a\sin{x_1} \right),
$$
$$
\dot{s}\leq-g_0\beta(x)\operatorname{sat}\left( \frac{s}{\varepsilon} \right)+\rho(x),\ g_0:=\min{\left( 2+\theta_2 \right)}=1
$$


Зададим функцию Ляпунова $V=0.5s^2$, тогда:
$$
\dot{V}=s\dot{s}\leq-\beta(x)s\operatorname{sat}\left( \frac{s}{\varepsilon} \right)+\rho(x)|s|
$$


При $|s|>\varepsilon:\operatorname{sat}(s/\varepsilon)=\operatorname{sign}(s)$
-- аналогичный разрывному случай: $\dot{V}\leq-\beta_0|s|,\beta_0>0$
-- уменьшение $s$ до $|s|\leq\varepsilon$.


При $|s|\leq\varepsilon:\operatorname{sat}(s/\varepsilon)=s/\varepsilon$,
тогда:
$$
\dot{V}\leq-\frac{\beta(x)}{\varepsilon}s^2+\rho(x)|s|
$$


Подставим $\beta(x)$:
$$
\dot{V}\leq-\frac{\rho(x)+\beta_0}{\varepsilon}s^2+\rho(x)|s|=-\frac{\beta_0}{\varepsilon}s^2-\frac{\rho(x)}{\varepsilon}s^2+\rho(x)|s|
$$


Два последних члена:
$$
-\frac{\rho(x)}{\varepsilon}s^2+\rho(x)|s|=\rho(x)|s|\left( 1-\frac{|s|}{\varepsilon} \right)\geq0,
$$
так как $|s|\leq\varepsilon$ внутри полосы.


Следовательно, худший случай $\dot{V}_{\max}$
достигается, когда $\rho(x)$ максимально.


Пусть $\rho_{\max}=\sup\limits_{x\in\mathcal{D}}{ \rho(x) }$,
где $\mathcal{D}$ -- ограниченная и замкнутая область в пространстве состояний,
в которой оценивается верхняя граница $|\delta(x)|\leq\rho(x)\leq \rho_{\max}$.


Тогда, оценка сверху:
$$
\dot{V}\leq-\frac{\beta_0}{\varepsilon}s^2+\rho_{\max}|s|
$$


Тогда, $\dot{V}<0$ при:
$$
|s|>\frac{\varepsilon\rho_{\max}}{\beta_0}
$$


Следовательно, все траектории входят в полосу:
$$
|s|\leq\varepsilon^*_s:=\frac{\varepsilon\rho_{\max}}{\beta_0}
$$
и остаются в ней.


Начало координат практически устойчиво --
траектории не расходятся и в конечном счете
входят в малую окрестность
нуля радиуса $\mathcal{O}(\varepsilon)$
и остаются в ней,
так как динамика на поверхности устойчива.


Если $\varepsilon\to0$, то $\varepsilon^*_s\to0$ --
приближение к идеальному скользящему режиму
с отклонением $\mathcal{O}(\varepsilon)$.


Так как $\varepsilon=const.,\operatorname{sat}(s/\varepsilon)=s/\varepsilon,\beta(x)\geq\rho(x)+\beta_0,\beta_0>0$
и система достигает полосы $|s|\leq\varepsilon$ за конечное время и остается в ней, то в окрестности поверхности
$s=0$ возникает квазискользящий режим.


Моделирование системы при $a=2,\beta_0=0.1,\varepsilon=0.01,x_0=\left[ 1\ 0 \right]^T,\theta_1=0.5,\theta_2=-0.5$:
\begin{figure}[H]
    \centering
    \includegraphics[scale=0.55]{1xsu.png}
    \caption{Графики $x_i(t),s(t),u(t)$}
    \label{fig:1xsu}
\end{figure}


Все графики сошлись к нулю, кроме разрывного управления.


В разрывном управлении две составляющие --
эквивалентная и переключающая. Первая стремится к нулю
вместе с состоянием. Вторая не стремится к нулю, а
переключается между $\pm\beta_0$, из-за чего
получаются высокочастотные колебания -- график
$u(t)$ выглядит как импульсный сигнал
при устоявшейся системе.


\section{Задание 2}
\subsection{Условие}
Рассмотрим систему:
$$
\begin{cases}
    \dot{x}_1=x_2+a_1x_1\sin{x_1},\\
    \dot{x}_2=a_2x_1x_2+3u,
\end{cases}
$$
где $a_1,a_2$ -- неизвестные параметры,
$|a_1-1|\leq1,|a_2-1|\leq1$.
Весь вектор состояния измерим.
Необходимо синтезировать стабилизирующий
регулятор на основе скользящих режимов,
провести соответствующий анализ
устойчивости и провести математическое моделирование.


\subsection{Выполнение}
Выберем поверхность:
$$
s=ax_1+x_2,\ a>0
$$


На поверхности $s=0$:
$$
x_2=-ax_1,\ x_2=\dot{x}_1-a_1x_1\sin{x}_1\Rightarrow \dot{x}_1=-ax_1+a_1x_1\sin{x_1},
$$
$$
\dot{x}_1=x_1\left( -a+a_1\sin{x_1} \right)
$$


Функция Ляпунова и ее производная для подсистемы на поверхности:
$$
V_1(x_1)=0.5x^2\Rightarrow \dot{V}_1=x_1\dot{x}_1=x_1^2\left( -a+a_1\sin{x_1} \right)
$$


Так как $|\sin{x_1}|\leq|x_1|,|a_1|\leq2$:
$$
a_1\sin{x_1}\leq|a_1|\cdot|\sin{x_1}|\leq2|x_1|
$$


Тогда:
$$
\dot{V}_1\leq x_1^2\left( -a+2|x_1| \right)<0\,\forall x_1: 0<|x_1|<\frac{a}{2}
$$


Начало координат локально асимптотически устойчиво.


Рассмотрим $\dot{s}$:
$$
\dot{s}=a\left( x_2+a_1x_1\sin{x_1} \right)+a_2x_1x_2+3u=\Delta(x)+3u
$$


Предположим номинальные значения $|a_i-1|=0\Rightarrow a_1=1,a_2=1$.


Тогда, номинальная эквивалентная часть регулятора,
обеспечивающая $\\\dot{s}\to0$ в номинальном случае:
$$
\Delta_{a_i=1}(x)+3u_\text{eq,nom}=0\Rightarrow u_\text{eq,nom}=-\frac{1}{3}\left( ax_2+ax_1\sin{x_1}+x_1x_2 \right)
$$


Добавим разрывную часть:
$$
u=u_\text{eq,nom}+u_\text{sw},\ u_\text{sw}=-\beta(x)\operatorname{sign}(s)
$$


Разрывный регулятор:
$$
u=-\frac{1}{3}\left( ax_2+ax_1\sin{x_1}+x_1x_2 \right)-\beta(x)\operatorname{sign}(s)
$$


Подставим в $\dot{s}$:
$$
\dot{s}=\Delta(x)+3\left( u_\text{eq,nom}+u_\text{sw} \right)=3u_\text{eq,nom}+\Delta_{a_i=1}(x)+3u_\text{sw}+\Delta(x)-\Delta_{a_i=1}(x),
$$
$$
\dot{s}=3u_\text{sw}+a\left( a_1-1 \right)x_1\sin{x_1}+\left( a_2-1 \right)x_1x_2=3u_\text{sw}+\delta(x)
$$


Оценим модуль невязки $\delta(x)$ при $|a_1-1|\leq1,|a_2-1|\leq1$:
$$
|\delta(x)|=\left|a\left( a_1-1 \right)x_1\sin{x_1}+\left( a_2-1 \right)x_1x_2\right|,
$$
$$
|\delta(x)|\leq a|x_1\sin{x_1}|+|x_1x_2|\leq ax_1^2+|x_1x_2|\equiv\rho(x)
$$


Тогда:
$$
\dot{s}\leq-3\beta(x)\operatorname{sign}(s)+ax_1^2+|x_1x_2|=-3\beta(x)\operatorname{sign}(s)+\rho(x)
$$


Оценим производную функции Ляпунова $V=0.5s^2$:
$$
\dot{V}=s\dot{s}\leq s\left( -3\beta(x)\operatorname{sign}(s)+\rho(x) \right)=-3\beta(x)|s|+\rho(x)s,
$$
$$
\dot{V}\leq-3\beta(x)|s|+\rho(x)s\leq-3\beta(x)|s|+|\rho(x)||s|=-\left( 3\beta(x)-|\rho(x)| \right)|s|
$$


Если выбрать $3\beta(x)\geq|\rho(x)|+\beta_0,\ \beta_0>0$:
$$
\dot{V}\leq-\beta_0|s|=-\beta_0\sqrt{2V}<0\,\forall s\neq0,\ \beta_0>0
$$


Это гарантирует достижение поверхности $s=0$
за конечное время.


Выберем:
$$
\beta(x)\geq\frac{1}{3}\left( ax_1^2+|x_1x_2|+\beta_0 \right),\ \beta_0>0
$$


Разрывный закон управления:
$$
u=-\frac{1}{3}\left( ax_2+ax_1\sin{x_1}+x_1x_2 \right)-\frac{1}{3}\left( ax_1^2+|x_1x_2|+\beta_0 \right)\operatorname{sign}(s)
$$


Заменим $\operatorname{sign}(s)$ на $\operatorname{sat}(s/\varepsilon)$
и получим непрерывный закон управления:
$$
u=-\frac{1}{3}\left( ax_2+ax_1\sin{x_1}+x_1x_2 \right)-\frac{1}{3}\left( ax_1^2+|x_1x_2|+\beta_0 \right)\operatorname{sat}\left( \frac{s}{\varepsilon} \right)
$$


Проверим устойчивость:
$$
V=0.5s^2,\ \dot{V}\leq -3\beta(x)s\operatorname{sat}\left(\frac{s}{\varepsilon}\right)+\rho(x)|s|
$$


% Функция насыщения:
% $$
% \operatorname{sat}\left( \frac{s}{\varepsilon} \right)=
% \begin{cases}
%     s/\varepsilon, &|s|\leq\varepsilon,\\
%     \operatorname{sign}(s), &|s|>\varepsilon
% \end{cases}
% $$


При $|s|>\varepsilon:\operatorname{sat}(s/\varepsilon)=\operatorname{sign}(s)$,
т.е. случай аналогичен дискретному:
$$
\dot{V}\leq-\beta_0|s|=-\beta_0\sqrt{2V}<0\,\forall s\neq0,\ \beta_0>0
$$


Уменьшение $s$ до $|s|\leq\varepsilon$.


При $|s|\leq\varepsilon:\operatorname{sat}(s/\varepsilon)=s/\varepsilon$, тогда:
$$
\dot{V}\leq -3\beta(x)\frac{s^2}{\varepsilon}+\rho(x)|s|
$$


Подставим $\beta(x)$:
$$
\dot{V}\leq-3\frac{\beta_0}{\varepsilon}s^2-3\frac{\rho(x)}{\varepsilon}s^2+\rho(x)|s|
$$


Два последних члена:
$$
-3\frac{\rho(x)}{\varepsilon}s^2+\rho(x)|s|=\rho(x)|s|\left( 1-3\frac{|s|}{\varepsilon} \right)=Q(x,s)
$$


Оценим $Q(x,s)$:
$$
Q(x,s)\geq0: |s|\leq\frac{\varepsilon}{3};\
Q(x,s)<0: \frac{\varepsilon}{3}<|s|\leq\varepsilon
$$


Т.е. вблизи $s=0$ возможно $\dot{V}>0$,
тогда, аналогично предыдущему пункту:
$$
\dot{V}\leq-3\frac{\beta_0}{\varepsilon}+\rho_{\max}|s|,\ |s|>\varepsilon^*_s:=\frac{\varepsilon\rho_{\max}}{3\beta_0}: \dot{V}<0,
$$
что гарантирует практическую устойчивость с предельной полосой $|s(t)|\leq \varepsilon^*_s$ при $t\to\infty$.


Моделирование системы при $a=1,\beta_0=0.1,\varepsilon=0.01,x_0=\left[ 1\ 0 \right]^T,a_1=0.5,a_2=1.5$:
\begin{figure}[H]
    \centering
    \includegraphics[scale=0.55]{2xsu.png}
    \caption{Графики $x_i(t),s(t),u(t)$}
    \label{fig:2xsu}
\end{figure}


\section{Задание 3}
\subsection{Условие}
Рассмотрим уравнение движения для маятника в виде:
$$
ml\ddot{\theta}+mg\sin{\theta}+kl\dot{\theta}=
\frac{T}{l}+mh(t)\cos{\theta},
$$
где $h$ -- горизонтальное ускорение,
$T$ -- управляющий момент.


Предположим, что:
$$
0.8\leq l\leq 1,\ 0.5\leq m \leq 1,\ 0.1\leq k\leq 0.2,\ |h(t)|\leq0.5
$$
и $g = 9.81$.
Требуется стабилизировать маятник при
$\theta=0$ для произвольных начальных условий.
Необходимо разработать непрерывный регулятор
на основе скользящего режима с обратной связью по состоянию.


\subsection{Выполнение}
Обозначим:
$$
x_1=\theta,\ x_2=\dot{\theta}
$$


Выразим из уравнения $\ddot{\theta}$:
$$
\ddot{\theta}=\frac{1}{ml}\left( \frac{T}{l}+mh(t)\cos{\theta}-kl\dot{\theta}-mg\sin{\theta} \right),
$$
$$
\ddot{\theta}=\frac{1}{ml^2}T+\frac{1}{l}h(t)\cos{\theta}-\frac{k}{m}\dot{\theta}-\frac{g}{l}\sin{\theta}
$$


Обозначим внутреннюю известную динамику:
$$
F(x)=-\frac{k}{m}x_2-\frac{g}{l}\sin{x_1}
$$
и возмущение:
$$
D(t,x)=\frac{1}{l}h(t)\cos{x_1}
$$


Тогда:
$$
\dot{x}_1=x_2,\ \dot{x}_2=\frac{1}{ml^2}T+F(x)+D(t,x)
$$


Коэффициент при управляющем моменте:
$$b(x)=\frac{1}{ml^2}>0,\ b(x)\in\left[ \frac{1}{m_{\max}l_{\max}^2}=1,\frac{1}{m_{\min}l_{\min}^2}=3.125 \right]$$


Оценка возмущения:
$$
|D(t,x)|\leq\frac{|h(t)|}{l_{\min}}\leq\frac{0.5}{0.8}=0.625
$$


Выберем поверхность скольжения:
$$
s=ax_1+x_2=a\theta+\dot{\theta},\ a>0
$$


На поверхности $s=0$:
$$
\dot{\theta}=-a\theta
$$


Для устойчивости достаточно $a>0$.


Вычислим $\dot{s}$:
$$
\dot{s}=a\dot{\theta}+\ddot{\theta}=ax_2+\frac{1}{ml^2}T+F(x)+D(t,x)
$$


Обозначим:
$$
\Phi(x)=F(x)+ax_2=-\frac{k}{m}x_2-\frac{g}{l}\sin{x_1}+ax_2
$$


Тогда:
$$
\dot{s}=\frac{1}{ml^2}T+\Phi(x)+D(t,x)
$$


Хотим $\dot{s}=-\beta_0$. Положим управляющий момент:
$$
T=ml^2\left( -\Phi(x)+u_\text{sw}(s) \right),
$$
где переключающее управление:
$$
u_\text{sw}=-\beta_0\operatorname{sat}\left( \frac{s}{\varepsilon} \right),\ \beta_0>0
$$


Так как параметры неизвестны, то положим номинальные значения:
$$
\hat{m}=m_{\min}=0.5,\ \hat{l}=l_{\min}=0.8
$$


Тогда, регулятор:
$$
T(x)=\hat{m}\hat{l}^2\left( -\hat{\Phi}(x)-\beta_0\operatorname{sat}\left( \frac{s(x)}{\varepsilon} \right) \right),
$$
где:
$$
\hat{\Phi}(x)=-\frac{\hat{k}}{\hat{m}}x_2-\frac{g}{\hat{l}}\sin{x_1}+ax_2,\ \hat{k}=k_{\max}=0.2
$$


Для функции Ляпунова $V=0.5s^2$ производная:
$$
\dot{V}=s\dot{s}=s\left( \frac{1}{ml^2}T+\Phi(x)+D(t,x) \right)
$$


Подставим $T$ в $\dot{V}$:
$$
\dot{V}=-\gamma\beta_0s\operatorname{sat}\left( \frac{s}{\varepsilon} \right)+s\left( \Phi-\gamma\hat{\Phi} \right)+sD,\
\gamma=\frac{\hat{m}\hat{l}^2}{ml^2}
$$


При $|s|>\varepsilon:\operatorname{sat}(s/\varepsilon)=\operatorname{sign}(s)$:
$$
\dot{V}=-\gamma\beta_0|s|+s\left( \Phi+\gamma\hat{\Phi} \right)+sD
$$


Оценим положительные слагаемые по абсолютной величине:
$$
s\left( \Phi-\gamma\hat{\Phi} \right)\leq|s||\Phi-\gamma\hat{\Phi}|,\ sD\leq|s||D|
$$


Тогда:
$$
\dot{V}\leq-\gamma\beta_0|s|+|s|\left( |\Phi-\gamma\hat{\Phi}| +|D|\right)
$$


Обозначим:
$$
\rho=|\Phi-\gamma\hat{\Phi}|+|D|
$$


Тогда:
$$
\dot{V}\leq-\left( \gamma\beta_0-\rho \right)|s|
$$


При $\gamma\beta_0>\sup\limits_{x\in\mathcal{D}}{\rho(x)}$
получим $\dot{V}<0$, т.е. $s$ уменьшается до $|s|\leq\varepsilon$.


При $|s|\leq\varepsilon:\operatorname{sat}(s/\varepsilon)=s/\varepsilon$:
$$
\dot{V}=-\gamma\beta_0\frac{s^2}{\varepsilon}+s\left( \Phi-\gamma\hat{\Phi} \right)+sD\leq-\gamma\beta_0\frac{s^2}{\varepsilon}+\rho(x)|s|
$$


Применим неравенство Юнга:
$$
\rho(x)|s|\leq\frac{\gamma\beta_0}{2\varepsilon}s^2+\frac{\varepsilon}{2\gamma\beta_0}\rho^2(x)
$$


Подставим в $\dot{V}$:
$$
\dot{V}\leq-\frac{\gamma\beta_0}{2\varepsilon}s^2+\frac{\varepsilon}{2\gamma\beta_0}\rho^2(x)
$$


Худший случай $\rho(x)=\sup\limits_{x\in\mathcal{D}}{\rho(x)}=\rho_{\max}$:
$$
\dot{V}\leq-\frac{\gamma\beta_0}{2\varepsilon}s^2+\frac{\varepsilon}{2\gamma\beta_0}\rho_{\max}^2
$$


Следовательно, $s$ остается ограниченным и асимптотически
попадает в окрестность:
$$
|s|\leq\varepsilon^*_s:=\frac{\varepsilon\rho_{\max}}{\gamma\beta_0},
$$
что говорит о практической устойчивости.


При выполнении $\gamma\beta_0>\rho_{\max}:\varepsilon^*_s<\varepsilon$,
т.е. после достижения слоя $|s|\leq\varepsilon$ ошибка дополнительно
сжимается до $|s|\leq\varepsilon^*_s<\varepsilon$.


Моделирование системы (см. рис. \ref{fig:3xsu}) при
\begin{align*}
    &g=9.81,\\
    &\hat{m}=0.5, m=0.8,\\
    &\hat{l}=0.8, l=1,\\
    &\hat{k}=0.2, k=0.15,\\
    &a=2,\\
    &\beta_0=15,\\
    &\varepsilon=0.05,\\
    &x_0=\begin{bmatrix}
        1\\0
    \end{bmatrix}
\end{align*}
\begin{figure}[H]
    \centering
    \includegraphics[scale=0.55]{3xsu.png}
    \caption{Графики $\theta,\dot{\theta},s(t),u(t)$}
    \label{fig:3xsu}
\end{figure}


Управление ненулевое, так как в системе
присутствует внешнее возмущение $D(t,x)=l^{-1}h(t)\cos{x_1}$
(горизонтальное ускорение с коэффициентом и гармоникой).


\section{Вывод}
В ходе выполнения лабораторной работы
для различных нелинейных систем
были синтезированы разрывный и/или
непрерывный стабилизирующие регуляторы
на основе скользящих режимов.
В каждом случае был проведен анализ устойчивости,
в ходе которого были получены асимптотическая
и практическая устойчивости систем в зависимости от условия
на скольжение. Также было проведено математическое
моделирование каждой из систем. Результаты показали
корректность выполненных расчетов.
\end{document}