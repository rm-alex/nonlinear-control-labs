\documentclass[a4paper,14pt]{extarticle}

\usepackage[T2A]{fontenc}
\usepackage[utf8]{inputenc}
\usepackage[english, russian]{babel}

\usepackage[left=30mm, right=10mm, top=20mm, bottom=20mm]{geometry}

\usepackage{tempora}
\usepackage{setspace}
\onehalfspacing

\usepackage{titlesec}
\titleformat{\section}[block]{\bfseries\centering\MakeUppercase}{\thesection.}{1em}{}
\titleformat{\subsection}[block]{\bfseries}{\thesubsection.}{1em}{}
\titleformat{\subsubsection}[block]{\bfseries}{\thesubsubsection.}{1em}{}

\renewcommand{\contentsname}{\hfill \textbf{СОДЕРЖАНИЕ} \hfill\null}

\usepackage{indentfirst}
\setlength{\parindent}{1.25cm}

\usepackage{amsmath, amsfonts, amssymb}
\usepackage{graphicx}
\usepackage{caption}
\usepackage{subcaption}
\usepackage{float}
\usepackage{tikz}
\usetikzlibrary{patterns}
\usepackage{cmap}
\usepackage{hyperref}
\usepackage{xcolor}
\usepackage{listings}

\definecolor{LightGray}{gray}{0.7}

\lstdefinestyle{code}{
    language=Python, % change if needed
    basicstyle=\small\ttfamily,
    numbers=left,
    numberstyle=\small\color{LightGray},
    stepnumber=1,
    numbersep=5pt,
    backgroundcolor=\color{white},
    showspaces=false,
    showstringspaces=false,
    showtabs=false,
    tabsize=4,
    captionpos=b,
    breaklines=true,
    breakatwhitespace=false,
    frame=single,
    rulecolor=\color{LightGray},
    linewidth=\linewidth,
    keywordstyle=\color{blue}\bfseries,
    commentstyle=\color{green!40!black},
    stringstyle=\color{violet},
    escapeinside={\%*}{*)},
    xleftmargin=10pt,
    xrightmargin=10pt,
    framexleftmargin=0pt,
    framexrightmargin=0pt
}
\lstset{style=code}

\hypersetup{
    colorlinks=true,
    linkcolor=blue,
    filecolor=magenta,
    urlcolor=cyan,
    pdftitle={ncs3},
    pdfauthor={Rumyantsev Alexey},
    pdfsubject={control},
    pdfkeywords={LaTeX, PDF},
    pdfpagemode=FullScreen,
}

\graphicspath{{src/images/}}

\begin{document}

\begin{titlepage}
    \begin{center}
        МИНИСТЕРСТВО НАУКИ И ВЫСШЕГО ОБРАЗОВАНИЯ РОССИЙСКОЙ ФЕДЕРАЦИИ\\
        \vspace*{2.5mm}
        Федеральное государственное автономное образовательное учреждение высшего образования
        «НАЦИОНАЛЬНЫЙ ИССЛЕДОВАТЕЛЬСКИЙ УНИВЕРСИТЕТ ИТМО»\\
        \vspace*{2.5mm}
        \textbf{ФАКУЛЬТЕТ СИСТЕМ УПРАВЛЕНИЯ И РОБОТОТЕХНИКИ}
        \vfill

        {\large ОТЧЕТ ПО ЛАБОРАТОРНОЙ РАБОТЕ №5}\\
        {\large по дисциплине}\\
        {\large\bfseries <<НЕЛИНЕЙНЫЕ СИСТЕМЫ УПРАВЛЕНИЯ>>}\\
        {\large на тему}\\
        {\large\bfseries <<СИНТЕЗ РАЗРЫВНОГО И НЕПРЕРЫВНОГО СТАБИЛИЗИРУЮЩИХ РЕГУЛЯТОРОВ НА ОСНОВЕ СКОЛЬЗЯЩИХ РЕЖИМОВ>>}\\
        \vfill

        \begin{flushright}
            Выполнили: студенты\\
            Румянцев А. А., R3441\\
            Дьячихин Д. Н., R3480\medskip\\

            Проверил: преподаватель\\
            Зименко К. А.
        \end{flushright}

        \vfill

        Санкт-Петербург\\
        2025
    \end{center}
\end{titlepage}

\setcounter{page}{2}
\tableofcontents
\newpage
\section{Задание 1}
\subsection{Условие}
Рассмотрим систему:
$$
\begin{cases}
    \dot{x}_1=x_2+\sin{x_1},\\
    \dot{x}_2=\theta_1x_1^2+\left( 2+\theta_2 \right)u,
\end{cases}
$$
где $|\theta_1|\leq1,|\theta_2|\leq1$. Весь вектор
состояния измерим. Необходимо:
\begin{enumerate}
    \item синтезировать стабилизирующий разрывный регулятор на основе скользящих режимов;
     \item синтезировать стабилизирующий непрерывный регулятор на основе скользящих режимов;
	\item провести соответствующий анализ устойчивости;
	\item провести математическое моделирование.
\end{enumerate}


\subsection{Выполнение}
...


\section{Задание 2}
\subsection{Условие}
Рассмотрим систему:
$$
\begin{cases}
    \dot{x}_1=x_2+a_1x_1\sin{x_1},\\
    \dot{x}_2=a_2x_1x_2+3u,
\end{cases}
$$
где $a_1,a_2$ -- неизвестные параметры,
$|a_1-1|\leq1,|a_2-1|\leq1$.
Весь вектор состояния измерим.
Необходимо синтезировать стабилизирующий
регулятор на основе скользящих режимов,
провести соответствующий анализ
устойчивости и провести математическое моделирование.


\subsection{Выполнение}
...


\section{Задание 3}
\subsection{Условие}
Рассмотрим уравнение движения для маятника в виде:
$$
ml\ddot{\theta}+mg\sin{\theta}+kl\dot{\theta}=
\frac{T}{l}+mh(t)\cos{\theta},
$$
где $h$ -- горизонтальное ускорение,
$T$ -- управляющий момент.


Предположим, что:
$$
0.8\leq l\leq 1,\ 0.5\leq m \leq 1,\ 0.1\leq k\leq 0.2,\ |h(t)|\leq0.5
$$
и $g = 9.81$.
Требуется стабилизировать маятник при
$\theta=0$ для произвольных начальных условий.
Необходимо разработать непрерывный регулятор
на основе скользящего режима с обратной связью по состоянию.


\subsection{Выполнение}
...


\section{Вывод}
...

\end{document}