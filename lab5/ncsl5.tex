\documentclass[a4paper,14pt]{extarticle}

\usepackage[T2A]{fontenc}
\usepackage[utf8]{inputenc}
\usepackage[english, russian]{babel}

\usepackage[left=30mm, right=10mm, top=20mm, bottom=20mm]{geometry}

\usepackage{tempora}
\usepackage{setspace}
\onehalfspacing

\usepackage{titlesec}
\titleformat{\section}[block]{\bfseries\centering\MakeUppercase}{\thesection.}{1em}{}
\titleformat{\subsection}[block]{\bfseries}{\thesubsection.}{1em}{}
\titleformat{\subsubsection}[block]{\bfseries}{\thesubsubsection.}{1em}{}

\renewcommand{\contentsname}{\hfill \textbf{СОДЕРЖАНИЕ} \hfill\null}

\usepackage{indentfirst}
\setlength{\parindent}{1.25cm}

\usepackage{amsmath, amsfonts, amssymb}
\usepackage{graphicx}
\usepackage{caption}
\usepackage{subcaption}
\usepackage{float}
\usepackage{tikz}
\usetikzlibrary{patterns}
\usepackage{cmap}
\usepackage{hyperref}
\usepackage{xcolor}
\usepackage{listings}

\definecolor{LightGray}{gray}{0.7}

\lstdefinestyle{code}{
    language=Python, % change if needed
    basicstyle=\small\ttfamily,
    numbers=left,
    numberstyle=\small\color{LightGray},
    stepnumber=1,
    numbersep=5pt,
    backgroundcolor=\color{white},
    showspaces=false,
    showstringspaces=false,
    showtabs=false,
    tabsize=4,
    captionpos=b,
    breaklines=true,
    breakatwhitespace=false,
    frame=single,
    rulecolor=\color{LightGray},
    linewidth=\linewidth,
    keywordstyle=\color{blue}\bfseries,
    commentstyle=\color{green!40!black},
    stringstyle=\color{violet},
    escapeinside={\%*}{*)},
    xleftmargin=10pt,
    xrightmargin=10pt,
    framexleftmargin=0pt,
    framexrightmargin=0pt
}
\lstset{style=code}

\hypersetup{
    colorlinks=true,
    linkcolor=blue,
    filecolor=magenta,
    urlcolor=cyan,
    pdftitle={ncs3},
    pdfauthor={Rumyantsev Alexey},
    pdfsubject={control},
    pdfkeywords={LaTeX, PDF},
    pdfpagemode=FullScreen,
}

\graphicspath{{src/images/}}

\begin{document}

\begin{titlepage}
    \begin{center}
        МИНИСТЕРСТВО НАУКИ И ВЫСШЕГО ОБРАЗОВАНИЯ РОССИЙСКОЙ ФЕДЕРАЦИИ\\
        \vspace*{2.5mm}
        Федеральное государственное автономное образовательное учреждение высшего образования
        «НАЦИОНАЛЬНЫЙ ИССЛЕДОВАТЕЛЬСКИЙ УНИВЕРСИТЕТ ИТМО»\\
        \vspace*{2.5mm}
        \textbf{ФАКУЛЬТЕТ СИСТЕМ УПРАВЛЕНИЯ И РОБОТОТЕХНИКИ}
        \vfill

        {\large ОТЧЕТ ПО ЛАБОРАТОРНОЙ РАБОТЕ №5}\\
        {\large по дисциплине}\\
        {\large\bfseries <<НЕЛИНЕЙНЫЕ СИСТЕМЫ УПРАВЛЕНИЯ>>}\\
        {\large на тему}\\
        {\large\bfseries <<СИНТЕЗ РАЗРЫВНОГО И НЕПРЕРЫВНОГО СТАБИЛИЗИРУЮЩИХ РЕГУЛЯТОРОВ НА ОСНОВЕ СКОЛЬЗЯЩИХ РЕЖИМОВ>>}\\
        \vfill

        \begin{flushright}
            Выполнили: студенты\\
            Румянцев А. А., R3441\\
            Дьячихин Д. Н., R3480\medskip\\

            Проверил: преподаватель\\
            Зименко К. А.
        \end{flushright}

        \vfill

        Санкт-Петербург\\
        2025
    \end{center}
\end{titlepage}

\setcounter{page}{2}
\tableofcontents
\newpage
\section{Задание 1}
\subsection{Условие}
Рассмотрим систему:
$$
\begin{cases}
    \dot{x}_1=x_2+\sin{x_1},\\
    \dot{x}_2=\theta_1x_1^2+\left( 2+\theta_2 \right)u,
\end{cases}
$$
где $|\theta_1|\leq1,|\theta_2|\leq1$. Весь вектор
состояния измерим. Необходимо:
\begin{enumerate}
    \item синтезировать стабилизирующий разрывный регулятор на основе скользящих режимов;
     \item синтезировать стабилизирующий непрерывный регулятор на основе скользящих режимов;
	\item провести соответствующий анализ устойчивости;
	\item провести математическое моделирование.
\end{enumerate}


\subsection{Выполнение}
Выберем скользящую поверхность,
на которой упростим динамику системы
и зададим ей желаемые свойства:
$$
s=ax_1+x_2=0,\ a>0
$$


На поверхности $s=0$:
$$
x_2=-ax_1,\ x_2=\dot{x}_1-\sin{x_1}\Rightarrow \dot{x}_1=-ax_1+\sin{x_1}
$$


При малых $x_1$ получаем $\sin{x_1}\sim x_1$:
$$
\dot{x}_1\approx -\left( a-1 \right)x_1
$$


Для асимптотической устойчивости $a>1\Rightarrow \lim\limits_{t\to\infty}x_1(t)=0$.


Динамика скользящей переменной:
$$
s=ax_1+x_2\Rightarrow \dot{s}=a\dot{x}_1+\dot{x}_2
$$


Подставим $\dot{x}_1,\dot{x}_2$:
$$
\dot{s}=a\left( x_2+\sin{x_1} \right)+\theta_1x_1^2+\left( 2+\theta_2 \right)u,
$$
$$
\dot{s}=\Delta(x)+\left( 2+\theta_2 \right)u,\ \Delta(x)\equiv ax_2+a\sin{x_1}+\theta_1x_1^2
$$


Необходимо выбрать $u$ так, чтобы
выполнялось условие достижения скользящего режима:
$$
s\dot{s}\leq-\eta|s|,\ \eta>0
$$


Это эквивалентно:
$$
\dot{V}\leq-\eta|s|,\ V=0.5s^2
$$


Синтезируем дискретный регулятор на основе скользящих режимов.


Выберем управление как сумму эквивалентной и переключающей составляющих:
$$
u=u_\text{eq}+u_\text{sw}
$$


Эквивалентная часть при $s=0$:
$$
u_\text{eq}=-\frac{\Delta(x)}{2+\theta_2}
$$


Параметры $\theta_{1},\theta_{2}$ неизвестны,
поэтому положим их нулевыми.


Такое допущение возможно вследствие
наличия переключающей части с
коэффициентом переключения $\beta_0$,
который при верном подборе компенсирует влияние отклонений $\theta_i$.


Кроме того, скользящий режим является робастным
к параметрическим неопределенностям и возмущениям.


Тогда, номинальная эквивалентная часть:
$$
u_\text{eq,nom}=-\frac{ax_2+a\sin{x_1}}{2}
$$


Добавим переключающую часть:
$$
u_\text{sw}=-\beta_0\operatorname{sign}(s),\ \beta_0>0
$$


Итоговый разрывный регулятор:
$$
u=-\frac{ax_2+a\sin{x_1}}{2}-\beta_0\operatorname{sign}(s),\ \beta_0>0
$$


Подставим $u$ в $\dot{s}$:
$$
\dot{s}=a\left( x_2+\sin{x_1} \right)+\theta_1x_1^2+\left( 2+\theta_2 \right)\left( -\frac{ax_2+a\sin{x_1}}{2}-\beta_0\operatorname{sign}(s) \right),
$$
$$
\dot{s}=\theta_1x_1^2-\frac{2+\theta_2}{2}\left( ax_2+a\sin{x_1} \right)+ax_2+a\sin{x_1}-\left( 2+\theta_2 \right)\beta_0\operatorname{sign}(s),
$$
$$
\dot{s}=\theta_1x_1^2-\frac{\theta_2}{2}\left( ax_2+a\sin{x_1} \right)-\left( 2+\theta_2 \right)\beta_0\operatorname{sign}(s)
$$


Оценим модуль <<невязки>> (первые два слагаемых).


Получим верхнюю оценку при $|\theta_1|\leq1,|\theta_2|\leq1,|\sin{x_1}|\leq|x_1|$:
$$
\left|\theta_1x_1^2-\frac{\theta_2}{2}a\left( x_2+\sin{x_1} \right)\right|\leq |x_1|^2+\frac{a}{2}\left( |x_2|+|x_1| \right)\equiv W(x)
$$


Тогда:
$$
\dot{s}\leq W(x)-\left( 2+\theta_2 \right)\beta_0\operatorname{sign}(s)
$$


Умножим выражение на $s$ и учтем $\operatorname{sign}(s)\cdot s=|s|$:
$$
s\dot{s}\leq s W(x)-\left( 2+\theta_2 \right)\beta_0|s|
$$


В правой части на $s$ не хватает модуля в первом слагаемом
-- заменим на верхнюю границу:
$$
W(x)\geq0\Rightarrow sW(x)\leq |s|W(x)
$$


Тогда:
$$
s\dot{s}\leq |s| W(x)-\left( 2+\theta_2 \right)\beta_0|s|
$$


Так как $2+\theta_2\geq g_0=1$,
достаточное условие достижения скольжения:
$$
\beta_0>\sup\limits_{x\in\mathcal{D}}{W(x)}+\delta,
$$
где $\mathcal{D}$ -- рассматриваемая область состояний,
$\delta>0$ -- небольшой запас.


Тогда:
$$
s\dot{s}\leq-\mu|s|,\ \mu=\left( \beta_0g_0-\sup{W(x)} \right)>0
$$


Т.е. гарантируем, что при минимально возможном $g_0$
и максимально возможном $W(x)$ их разница при $\beta_0g_0$ будет положительна.


Таким образом, для $V=0.5s^2$ имеем $\dot{V}=s\dot{s}\leq-\mu|s|,\mu>0$,
что гарантирует достижение $s=0$ за конечное время.


Синтезируем непрерывный регулятор на основе скользящих режимов.


\section{Задание 2}
\subsection{Условие}
Рассмотрим систему:
$$
\begin{cases}
    \dot{x}_1=x_2+a_1x_1\sin{x_1},\\
    \dot{x}_2=a_2x_1x_2+3u,
\end{cases}
$$
где $a_1,a_2$ -- неизвестные параметры,
$|a_1-1|\leq1,|a_2-1|\leq1$.
Весь вектор состояния измерим.
Необходимо синтезировать стабилизирующий
регулятор на основе скользящих режимов,
провести соответствующий анализ
устойчивости и провести математическое моделирование.


\subsection{Выполнение}
...


\section{Задание 3}
\subsection{Условие}
Рассмотрим уравнение движения для маятника в виде:
$$
ml\ddot{\theta}+mg\sin{\theta}+kl\dot{\theta}=
\frac{T}{l}+mh(t)\cos{\theta},
$$
где $h$ -- горизонтальное ускорение,
$T$ -- управляющий момент.


Предположим, что:
$$
0.8\leq l\leq 1,\ 0.5\leq m \leq 1,\ 0.1\leq k\leq 0.2,\ |h(t)|\leq0.5
$$
и $g = 9.81$.
Требуется стабилизировать маятник при
$\theta=0$ для произвольных начальных условий.
Необходимо разработать непрерывный регулятор
на основе скользящего режима с обратной связью по состоянию.


\subsection{Выполнение}
...


\section{Вывод}
...

\end{document}