\documentclass[a4paper,14pt]{extarticle}

\usepackage[T2A]{fontenc}
\usepackage[utf8]{inputenc}
\usepackage[english, russian]{babel}

\usepackage[left=30mm, right=10mm, top=20mm, bottom=20mm]{geometry}

\usepackage{tempora}
\usepackage{setspace}
\onehalfspacing

\usepackage{titlesec}
\titleformat{\section}[block]{\bfseries\centering\MakeUppercase}{\thesection.}{1em}{}
\titleformat{\subsection}[block]{\bfseries}{\thesubsection.}{1em}{}
\titleformat{\subsubsection}[block]{\bfseries}{\thesubsubsection.}{1em}{}

\renewcommand{\contentsname}{\hfill \textbf{СОДЕРЖАНИЕ} \hfill\null}

\usepackage{indentfirst}
\setlength{\parindent}{1.25cm}

\usepackage{amsmath, amsfonts, amssymb}
\usepackage{graphicx}
\usepackage{caption}
\usepackage{subcaption}
\usepackage{float}
\usepackage{tikz}
\usetikzlibrary{patterns}
\usepackage{cmap}
\usepackage{hyperref}
\usepackage{xcolor}
\usepackage{listings}

\definecolor{LightGray}{gray}{0.7}

\lstdefinestyle{code}{
    language=Python, % change if needed
    basicstyle=\small\ttfamily,
    numbers=left,
    numberstyle=\small\color{LightGray},
    stepnumber=1,
    numbersep=5pt,
    backgroundcolor=\color{white},
    showspaces=false,
    showstringspaces=false,
    showtabs=false,
    tabsize=4,
    captionpos=b,
    breaklines=true,
    breakatwhitespace=false,
    frame=single,
    rulecolor=\color{LightGray},
    linewidth=\linewidth,
    keywordstyle=\color{blue}\bfseries,
    commentstyle=\color{green!40!black},
    stringstyle=\color{violet},
    escapeinside={\%*}{*)},
    xleftmargin=10pt,
    xrightmargin=10pt,
    framexleftmargin=0pt,
    framexrightmargin=0pt
}
\lstset{style=code}

\hypersetup{
    colorlinks=true,
    linkcolor=blue,
    filecolor=magenta,
    urlcolor=cyan,
    pdftitle={ncs3},
    pdfauthor={Rumyantsev Alexey},
    pdfsubject={control},
    pdfkeywords={LaTeX, PDF},
    pdfpagemode=FullScreen,
}

\graphicspath{{src/images/}}

\begin{document}

\begin{titlepage}
    \begin{center}
        МИНИСТЕРСТВО НАУКИ И ВЫСШЕГО ОБРАЗОВАНИЯ РОССИЙСКОЙ ФЕДЕРАЦИИ\\
        \vspace*{2.5mm}
        Федеральное государственное автономное образовательное учреждение высшего образования
        «НАЦИОНАЛЬНЫЙ ИССЛЕДОВАТЕЛЬСКИЙ УНИВЕРСИТЕТ ИТМО»\\
        \vspace*{2.5mm}
        \textbf{ФАКУЛЬТЕТ СИСТЕМ УПРАВЛЕНИЯ И РОБОТОТЕХНИКИ}
        \vfill

        {\large ОТЧЕТ ПО ЛАБОРАТОРНОЙ РАБОТЕ №4}\\
        {\large по дисциплине}\\
        {\large\bfseries «НЕЛИНЕЙНЫЕ СИСТЕМЫ УПРАВЛЕНИЯ»}\\
        {\large на тему}\\
        {\large\bfseries «СИНТЕЗ РЕГУЛЯТОРА НА ОСНОВЕ МЕТОДА БЭКСТЕППИНГА»}\\
        \vfill

        \begin{flushright}
            Выполнил: студент гр. R3441\\
            Румянцев А. А.\medskip\\

            Проверил: преподаватель\\
            Зименко К. А.
        \end{flushright}

        \vfill

        Санкт-Петербург\\
        2025
    \end{center}
\end{titlepage}

\setcounter{page}{2}
\tableofcontents
\newpage

\section{Задание 1}
\subsection{Условие}
Рассмотрим систему:
\begin{align}
    \begin{cases}
        \dot{x}_1=x_2+\sin{\left( x_1 \right)}+x_1^2,\\
        \dot{x}_2=x_1^2+\left( 2+\sin{\left( x_1 \right)} \right)u
    \end{cases}
\end{align}


Весь вектор состояния измерим. Необходимо синтезировать
стабилизирующий регулятор на основе метода бэкстеппинга и провести
математическое моделирование.


\subsection{Выполнение}
...


\section{Задание 2}
\subsection{Условие}
Рассмотрим систему:
\begin{align}
    \begin{cases}
        \dot{x}_1=x_2-x_1^3,\\
        \dot{x}_2=x_1+u
    \end{cases}
\end{align}


Весь вектор состояния измерим. Необходимо синтезировать
стабилизирующий регулятор на основе метода бэкстеппинга и провести
математическое моделирование.


\subsection{Выполнение}
...


\section{Задание 3}
\subsection{Условие}
Рассмотрим систему:
\begin{align}
    \begin{cases}
        \dot{x}_1=\cos{\left( x_1 \right)}-x_2,\\
        \dot{x}_2=x_1+x_3,\\
        \dot{x}_3=x_1x_3+\left( 2-\sin{\left( x_3 \right)} \right)x_4,\\
        \dot{x}_4=x_2x_3+2u
    \end{cases}
\end{align}


Весь вектор состояния измерим. Необходимо синтезировать
стабилизирующий регулятор на основе метода бэкстеппинга и провести
математическое моделирование.


\subsection{Выполнение}
...


\section{Вывод}
...
\end{document}