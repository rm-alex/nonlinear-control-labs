\documentclass[a4paper,14pt]{extarticle}

\usepackage[T2A]{fontenc}
\usepackage[utf8]{inputenc}
\usepackage[english, russian]{babel}

\usepackage[left=30mm, right=10mm, top=20mm, bottom=20mm]{geometry}

\usepackage{tempora}
\usepackage{setspace}
\onehalfspacing

\usepackage{titlesec}
\titleformat{\section}[block]{\bfseries\centering\MakeUppercase}{\thesection.}{1em}{}
\titleformat{\subsection}[block]{\bfseries}{\thesubsection.}{1em}{}
\titleformat{\subsubsection}[block]{\bfseries}{\thesubsubsection.}{1em}{}

\renewcommand{\contentsname}{\hfill \textbf{СОДЕРЖАНИЕ} \hfill\null}

\usepackage{indentfirst}
\setlength{\parindent}{1.25cm}

\usepackage{amsmath, amsfonts, amssymb}
\usepackage{graphicx}
\usepackage{caption}
\usepackage{subcaption}
\usepackage{float}
\usepackage{tikz}
\usetikzlibrary{patterns}
\usepackage{cmap}
\usepackage{hyperref}
\usepackage{xcolor}
\usepackage{listings}

\definecolor{LightGray}{gray}{0.7}

\lstdefinestyle{code}{
    language=Python, % change if needed
    basicstyle=\small\ttfamily,
    numbers=left,
    numberstyle=\small\color{LightGray},
    stepnumber=1,
    numbersep=5pt,
    backgroundcolor=\color{white},
    showspaces=false,
    showstringspaces=false,
    showtabs=false,
    tabsize=4,
    captionpos=b,
    breaklines=true,
    breakatwhitespace=false,
    frame=single,
    rulecolor=\color{LightGray},
    linewidth=\linewidth,
    keywordstyle=\color{blue}\bfseries,
    commentstyle=\color{green!40!black},
    stringstyle=\color{violet},
    escapeinside={\%*}{*)},
    xleftmargin=10pt,
    xrightmargin=10pt,
    framexleftmargin=0pt,
    framexrightmargin=0pt
}
\lstset{style=code}

\hypersetup{
    colorlinks=true,
    linkcolor=blue,
    filecolor=magenta,
    urlcolor=cyan,
    pdftitle={ncs3},
    pdfauthor={Rumyantsev Alexey},
    pdfsubject={control},
    pdfkeywords={LaTeX, PDF},
    pdfpagemode=FullScreen,
}

\graphicspath{{src/images/}}

\begin{document}

\begin{titlepage}
    \begin{center}
        МИНИСТЕРСТВО НАУКИ И ВЫСШЕГО ОБРАЗОВАНИЯ РОССИЙСКОЙ ФЕДЕРАЦИИ\\
        \vspace*{2.5mm}
        Федеральное государственное автономное образовательное учреждение высшего образования
        «НАЦИОНАЛЬНЫЙ ИССЛЕДОВАТЕЛЬСКИЙ УНИВЕРСИТЕТ ИТМО»\\
        \vspace*{2.5mm}
        \textbf{ФАКУЛЬТЕТ СИСТЕМ УПРАВЛЕНИЯ И РОБОТОТЕХНИКИ}
        \vfill

        {\large ОТЧЕТ ПО ЛАБОРАТОРНОЙ РАБОТЕ №4}\\
        {\large по дисциплине}\\
        {\large\bfseries «НЕЛИНЕЙНЫЕ СИСТЕМЫ УПРАВЛЕНИЯ»}\\
        {\large на тему}\\
        {\large\bfseries «СИНТЕЗ РЕГУЛЯТОРА НА ОСНОВЕ МЕТОДА БЭКСТЕППИНГА»}\\
        \vfill

        \begin{flushright}
            Выполнил: студент гр. R3441\\
            Румянцев А. А.\medskip\\

            Проверил: преподаватель\\
            Зименко К. А.
        \end{flushright}

        \vfill

        Санкт-Петербург\\
        2025
    \end{center}
\end{titlepage}

\setcounter{page}{2}
\tableofcontents
\newpage

\section{Задание 1}
\subsection{Условие}
Рассмотрим систему:
\begin{align}
    \begin{cases}
        \dot{x}_1=x_2+\sin{\left( x_1 \right)}+x_1^2,\\
        \dot{x}_2=x_1^2+\left( 2+\sin{\left( x_1 \right)} \right)u
    \end{cases}
\end{align}


Весь вектор состояния измерим. Необходимо синтезировать
стабилизирующий регулятор на основе метода бэкстеппинга и провести
математическое моделирование.


\subsection{Выполнение}
Компонента $x_2$
является виртуальным
управлением подсистемы $\dot{x}_1$:
$$
\dot{x}_1=f(x_1)+\varphi(x_1),\ f(x_1)=\sin{\left( x_1 \right)}+x_1^2 ,\ \varphi(x_1)=x_2
$$


Функция Ляпунова:
$$
V_1(x_1)=0.5x_1^2,\ V_1(x_1)>0\,\forall x_1\neq0, \ V_1(0)=0
$$


Ее производная:
$$
\dot{V}_1(x_1)=x_1\dot{x}_1=x_1\left( x_2+\sin{\left( x_1 \right)+x_1^2} \right)=x_1x_2+x_1\sin{\left( x_1 \right)}+x_1^3
$$


Необходимо, чтобы $\dot{V}_1(x_1)<0\,\forall x_1\neq0$.
В таком случае нужно компенсировать
неопределенные по знаку слагаемые $x_1\sin{\left( x_1 \right)},x_1^3$
и сделать слагаемое $x_1x_2$ отрицательным четной степени
через $x_2=\varphi(x_1)$:
$$
\varphi(x_1)=-\sin{\left( x_1 \right)}-x_1^2-k_1x_1,\ k_1>0\Rightarrow \dot{x}_1=-k_1x_1
$$


Тогда, производная функции Ляпунова:
$$
\dot{V}_1(x_1)=-k_1x_1^2<0\,\forall x_1\neq0
$$


Сделаем замену, чтобы определить ошибку между реальным
$x_2$ и виртуальным управлением:
$$
z=x_2-\varphi(x_1)=x_2+\sin{\left( x_1 \right)}+x_1^2+k_1x_1
$$


Тогда:
$$
x_2=z-\sin{\left( x_1 \right)}-x_1^2-k_1x_1
$$


Подставим $x_2$ в подсистему $\dot{x}_1$:
$$
\dot{x}_1=x_2+\sin{\left( x_1 \right)}+x_1^2=z-k_1x_1
$$


Найдем $\dot{z}$:
$$
\dot{z}=\dot{x}_2-\dot{\varphi}_t(x_1)
$$


Найдем $\dot{\varphi}_t(x_1)$:
$$
\dot{\varphi}_t(x_1)=\left(-\cos{\left( x_1 \right)}-2x_1-k_1\right)\dot{x}_1=-\left(\cos{\left( x_1 \right)}+2x_1+k_1\right)\left( z-k_1x_1 \right)
$$


Тогда:
$$
\dot{z}=x_1^2+\left( 2+\sin{\left( x_1 \right)} \right)u+\left(\cos{\left( x_1 \right)}+2x_1+k_1\right)\left( z-k_1x_1 \right)
$$


Полная функция Ляпунова:
$$
V=V_1+0.5z^2,\ V_1=0.5x_1^2
$$


Ее производная:
$$
\dot{V}=x_1\dot{x}_1+z\dot{z},
$$
$$
\dot{V}=x_1\left( z-k_1x_1 \right)+z\left( x_1^2+\left( 2+\sin{\left( x_1 \right)} \right)u+\left(\cos{\left( x_1 \right)}+2x_1+k_1\right)\left( z-k_1x_1 \right) \right),
$$
$$
\dot{V}=-k_1x_1^2+z\left( 2+\sin{\left( x_1 \right)} \right)u+zx_1+zx_1^2+z\left( \cos{\left( x_1 \right)}+2x_1+k_1 \right)\left( z-k_1x_1 \right),
$$


Выберем управление $u(t)$ так, чтобы
компенсировать
неопределенные по знаку слагаемые и
добавить отрицательно определенное слагаемое с $z$:
$$
u=\frac{1}{2+\sin{\left( x_1 \right)}}\left( -x_1-x_1^2-\left( \cos{\left( x_1 \right)}+2x_1+k_1 \right)\left( z-k_1x_1 \right) -k_2z\right),\ k_2>0,
$$
$$
\dot{V}=-k_1x_1^2-k_2z^2<0\,\forall(x_1,z)\neq(0,0),k_1,k_2>0
$$


Начало координат глобально асимптотически устойчиво с таким законом
управления $u$.


Подставляя $z=x_2+\sin{\left( x_1 \right)}+x_1^2+k_1x_1$ в $u$, итоговый закон управления:
\begin{align*}
    u=&\frac{1}{2+\sin{\left( x_1 \right)}}\left( -x_1-x_1^2-\left( \cos{\left( x_1 \right)}+2x_1+k_1 \right)\left( x_2+\sin{\left( x_1 \right)}+x_1^2 \right)+\right.\\
    &\left.-k_2\left( x_2+\sin{\left( x_1 \right)+x_1^2+k_1x_1} \right) \right)
\end{align*}


Выполним моделирование системы при $k_1=2,k_2=3,x_0=(1,-0.5)$:
\begin{figure}[H]
    \centering
    \includegraphics[scale=0.55]{1xu.png}
    \caption{Графики $x_i(t),u(t)$}
    \label{fig:1xu}
\end{figure}


Вектор состояния и управление стремятся к нулю с течением времени.


\section{Задание 2}
\subsection{Условие}
Рассмотрим систему:
\begin{align}
    \begin{cases}
        \dot{x}_1=x_2-x_1^3,\\
        \dot{x}_2=x_1+u
    \end{cases}
\end{align}


Весь вектор состояния измерим. Необходимо синтезировать
стабилизирующий регулятор на основе метода бэкстеппинга и провести
математическое моделирование.


\subsection{Выполнение}
...


\section{Задание 3}
\subsection{Условие}
Рассмотрим систему:
\begin{align}
    \begin{cases}
        \dot{x}_1=\cos{\left( x_1 \right)}-x_2,\\
        \dot{x}_2=x_1+x_3,\\
        \dot{x}_3=x_1x_3+\left( 2-\sin{\left( x_3 \right)} \right)x_4,\\
        \dot{x}_4=x_2x_3+2u
    \end{cases}
\end{align}


Весь вектор состояния измерим. Необходимо синтезировать
стабилизирующий регулятор на основе метода бэкстеппинга и провести
математическое моделирование.


\subsection{Выполнение}
...


\section{Вывод}
...
\end{document}