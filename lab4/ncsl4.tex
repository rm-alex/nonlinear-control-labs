\documentclass[a4paper,14pt]{extarticle}

\usepackage[T2A]{fontenc}
\usepackage[utf8]{inputenc}
\usepackage[english, russian]{babel}

\usepackage[left=30mm, right=10mm, top=20mm, bottom=20mm]{geometry}

\usepackage{tempora}
\usepackage{setspace}
\onehalfspacing

\usepackage{titlesec}
\titleformat{\section}[block]{\bfseries\centering\MakeUppercase}{\thesection.}{1em}{}
\titleformat{\subsection}[block]{\bfseries}{\thesubsection.}{1em}{}
\titleformat{\subsubsection}[block]{\bfseries}{\thesubsubsection.}{1em}{}

\renewcommand{\contentsname}{\hfill \textbf{СОДЕРЖАНИЕ} \hfill\null}

\usepackage{indentfirst}
\setlength{\parindent}{1.25cm}

\usepackage{amsmath, amsfonts, amssymb}
\usepackage{graphicx}
\usepackage{caption}
\usepackage{subcaption}
\usepackage{float}
\usepackage{tikz}
\usetikzlibrary{patterns}
\usepackage{cmap}
\usepackage{hyperref}
\usepackage{xcolor}
\usepackage{listings}

\definecolor{LightGray}{gray}{0.7}

\lstdefinestyle{code}{
    language=Python, % change if needed
    basicstyle=\small\ttfamily,
    numbers=left,
    numberstyle=\small\color{LightGray},
    stepnumber=1,
    numbersep=5pt,
    backgroundcolor=\color{white},
    showspaces=false,
    showstringspaces=false,
    showtabs=false,
    tabsize=4,
    captionpos=b,
    breaklines=true,
    breakatwhitespace=false,
    frame=single,
    rulecolor=\color{LightGray},
    linewidth=\linewidth,
    keywordstyle=\color{blue}\bfseries,
    commentstyle=\color{green!40!black},
    stringstyle=\color{violet},
    escapeinside={\%*}{*)},
    xleftmargin=10pt,
    xrightmargin=10pt,
    framexleftmargin=0pt,
    framexrightmargin=0pt
}
\lstset{style=code}

\hypersetup{
    colorlinks=true,
    linkcolor=blue,
    filecolor=magenta,
    urlcolor=cyan,
    pdftitle={ncs3},
    pdfauthor={Rumyantsev Alexey},
    pdfsubject={control},
    pdfkeywords={LaTeX, PDF},
    pdfpagemode=FullScreen,
}

\graphicspath{{src/images/}}

\begin{document}

\begin{titlepage}
    \begin{center}
        МИНИСТЕРСТВО НАУКИ И ВЫСШЕГО ОБРАЗОВАНИЯ РОССИЙСКОЙ ФЕДЕРАЦИИ\\
        \vspace*{2.5mm}
        Федеральное государственное автономное образовательное учреждение высшего образования
        «НАЦИОНАЛЬНЫЙ ИССЛЕДОВАТЕЛЬСКИЙ УНИВЕРСИТЕТ ИТМО»\\
        \vspace*{2.5mm}
        \textbf{ФАКУЛЬТЕТ СИСТЕМ УПРАВЛЕНИЯ И РОБОТОТЕХНИКИ}
        \vfill

        {\large ОТЧЕТ ПО ЛАБОРАТОРНОЙ РАБОТЕ №4}\\
        {\large по дисциплине}\\
        {\large\bfseries «НЕЛИНЕЙНЫЕ СИСТЕМЫ УПРАВЛЕНИЯ»}\\
        {\large на тему}\\
        {\large\bfseries «СИНТЕЗ РЕГУЛЯТОРА НА ОСНОВЕ МЕТОДА БЭКСТЕППИНГА»}\\
        \vfill

        \begin{flushright}
            Выполнил: студент гр. R3441\\
            Румянцев А. А.\medskip\\

            Проверил: преподаватель\\
            Зименко К. А.
        \end{flushright}

        \vfill

        Санкт-Петербург\\
        2025
    \end{center}
\end{titlepage}

\setcounter{page}{2}
\tableofcontents
\newpage

\section{Задание 1}
\subsection{Условие}
Рассмотрим систему:
\begin{align}
    \begin{cases}
        \dot{x}_1=x_2+\sin{\left( x_1 \right)}+x_1^2,\\
        \dot{x}_2=x_1^2+\left( 2+\sin{\left( x_1 \right)} \right)u
    \end{cases}
\end{align}


Весь вектор состояния измерим. Необходимо синтезировать
стабилизирующий регулятор на основе метода бэкстеппинга и провести
математическое моделирование.


\subsection{Выполнение}
Компонента $x_2$
является виртуальным
управлением подсистемы $\dot{x}_1$:
$$
\dot{x}_1=f(x_1)+\varphi(x_1),\ f(x_1)=\sin{\left( x_1 \right)}+x_1^2 ,\ \varphi(x_1)=x_2
$$


Функция Ляпунова:
$$
V_1(x_1)=0.5x_1^2,\ V_1(x_1)>0\,\forall x_1\neq0, \ V_1(0)=0
$$


Ее производная:
$$
\dot{V}_1(x_1)=x_1\dot{x}_1=x_1\left( x_2+\sin{\left( x_1 \right)+x_1^2} \right)=x_1x_2+x_1\sin{\left( x_1 \right)}+x_1^3
$$


Необходимо, чтобы $\dot{V}_1(x_1)<0\,\forall x_1\neq0$.
В таком случае нужно компенсировать
неопределенные по знаку слагаемые $x_1\sin{\left( x_1 \right)},x_1^3$
и сделать слагаемое $x_1x_2$ отрицательным четной степени
через $x_2=\varphi(x_1)$:
$$
\varphi(x_1)=-\sin{\left( x_1 \right)}-x_1^2-k_1x_1,\ k_1>0\Rightarrow \dot{x}_1=-k_1x_1
$$


Тогда, производная функции Ляпунова:
$$
\dot{V}_1(x_1)=-k_1x_1^2<0\,\forall x_1\neq0, k_1>0
$$


Сделаем замену, чтобы определить ошибку между реальным
$x_2$ и виртуальным управлением:
$$
z=x_2-\varphi(x_1)=x_2+\sin{\left( x_1 \right)}+x_1^2+k_1x_1
$$


Тогда:
$$
x_2=z-\sin{\left( x_1 \right)}-x_1^2-k_1x_1
$$


Подставим $x_2$ в подсистему $\dot{x}_1$:
$$
\dot{x}_1=x_2+\sin{\left( x_1 \right)}+x_1^2=z-k_1x_1
$$


Найдем $\dot{z}$:
$$
\dot{z}=\dot{x}_2-\dot{\varphi}_t(x_1)
$$


Найдем $\dot{\varphi}_t(x_1)$:
$$
\dot{\varphi}_t(x_1)=\left(-\cos{\left( x_1 \right)}-2x_1-k_1\right)\dot{x}_1=-\left(\cos{\left( x_1 \right)}+2x_1+k_1\right)\left( z-k_1x_1 \right)
$$


Тогда:
$$
\dot{z}=x_1^2+\left( 2+\sin{\left( x_1 \right)} \right)u+\left(\cos{\left( x_1 \right)}+2x_1+k_1\right)\left( z-k_1x_1 \right)
$$


Полная функция Ляпунова:
$$
V=V_1+0.5z^2,\ V_1=0.5x_1^2
$$


Ее производная:
$$
\dot{V}=x_1\dot{x}_1+z\dot{z},
$$
$$
\dot{V}=x_1\left( z-k_1x_1 \right)+z\left( x_1^2+\left( 2+\sin{\left( x_1 \right)} \right)u+\left(\cos{\left( x_1 \right)}+2x_1+k_1\right)\left( z-k_1x_1 \right) \right),
$$
$$
\dot{V}=-k_1x_1^2+z\left( 2+\sin{\left( x_1 \right)} \right)u+zx_1+zx_1^2+z\left( \cos{\left( x_1 \right)}+2x_1+k_1 \right)\left( z-k_1x_1 \right),
$$


Выберем управление $u(t)$ так, чтобы
компенсировать
неопределенные по знаку слагаемые и
добавить отрицательно определенное слагаемое с $z$:
$$
u=\frac{1}{2+\sin{\left( x_1 \right)}}\left( -x_1-x_1^2-\left( \cos{\left( x_1 \right)}+2x_1+k_1 \right)\left( z-k_1x_1 \right) -k_2z\right),\ k_2>0,
$$
$$
\dot{V}=-k_1x_1^2-k_2z^2<0\,\forall(x_1,z)\neq(0,0),k_1,k_2>0
$$


Начало координат глобально асимптотически устойчиво с таким законом
управления $u$.


Подставляя $z=x_2+\sin{\left( x_1 \right)}+x_1^2+k_1x_1$ в $u$, итоговый закон управления:
\begin{align*}
    u=&\frac{1}{2+\sin{\left( x_1 \right)}}\left( -x_1-x_1^2-\left( \cos{\left( x_1 \right)}+2x_1+k_1 \right)\left( x_2+\sin{\left( x_1 \right)}+x_1^2 \right)+\right.\\
    &\left.-k_2\left( x_2+\sin{\left( x_1 \right)+x_1^2+k_1x_1} \right) \right)
\end{align*}


Выполним моделирование системы при $k_1=2,k_2=3,x_0=(1,-0.5)$:
\begin{figure}[H]
    \centering
    \includegraphics[scale=0.55]{1xu.png}
    \caption{Графики $x_i(t),u(t)$}
    \label{fig:1xu}
\end{figure}


Вектор состояния и управление стремятся к нулю с течением времени.


\section{Задание 2}
\subsection{Условие}
Рассмотрим систему:
\begin{align}
    \begin{cases}
        \dot{x}_1=x_2-x_1^3,\\
        \dot{x}_2=x_1+u
    \end{cases}
\end{align}


Весь вектор состояния измерим. Необходимо синтезировать
стабилизирующий регулятор на основе метода бэкстеппинга и провести
математическое моделирование.


\subsection{Выполнение}
Компонента $x_2$ -- виртуальное управление
подсистемы $\dot{x}_1$:
$$
\dot{x}_1=f(x_1)+\varphi(x_1),\ f(x_1)=-x_1^3,\ \varphi(x_1)=x_2
$$


Функция Ляпунова и ее производная:
$$
V_1(x_1)=0.5x_1^2,\ \dot{V}_1(x_1)=x_1\left( x_2-x_1^3 \right)=x_1x_2-x_1^4
$$


Чтобы $\dot{V}_1\leq0$, нужно сделать
слагаемое $x_1x_2$ отрицательно определенным
через $\varphi(x_1)=x_2$:
$$
\varphi(x_1)=-k_1x_1,\ k_1>0
$$


Тогда:
$$
\dot{V}_1(x_1)=x_1\left( -k_1x_1-x_1^3 \right)=-k_1x_1^2-x_1^4<0\,\forall x_1\neq0,k_1>0
$$


Замена:
$$
z=x_2-\varphi(x_1)=x_2+k_1x_1
$$


Тогда:
$$
x_2=z-k_1x_1
$$


Подставим $x_2$ в подсистему $\dot{x}_1$:
$$
\dot{x}_1=x_2-x_1^3=z-k_1x_1-x_1^3
$$


Найдем $\dot{z}$:
$$
\dot{z}=\dot{x}_2-\dot{\varphi}_t(x_1)
$$


Найдем $\dot{\varphi}_t(x_1)$:
$$
\dot{\varphi}_t(x_1)=-k_1\dot{x}_1=-k_1\left( z-k_1x_1-x_1^3 \right)
$$


Тогда:
$$
\dot{z}=x_1+u+k_1\left( z-k_1x_1-x_1^3 \right)
$$


Полная функция Ляпунова и ее производная:
$$
V=V_1+0.5z^2,\ V_1=0.5x_1^2,
$$
$$
\dot{V}=x_1\left( z-k_1x_1-x_1^3 \right)+z\left( x_1+u +k_1\left( z-k_1x_1-x_1^3 \right) \right),
$$
$$
\dot{V}=2zx_1-k_1x_1^2-x_1^4+zu+z^2k_1-zk_1^2x_1-zk_1x_1^3
$$


Закон управления:
$$
u=-2x_1-k_1z+k_1^2x_1+k_1x_1^3-k_2z
$$


Тогда:
$$
\dot{V}=-k_1x_1^2-x_1^4-k_2z^2<0\,\forall(x_1,z)\neq(0,0),k_1,k_2>0
$$


Начало координат глобально асимптотически устойчиво.


Итоговый закон управления:
$$
u=-2x_1-k_1\left( x_2+k_1x_1 \right)+k_1^2x_1+k_1x_1^3-k_2\left( x_2+k_1x_1 \right),
$$
$$
u=-2x_1-k_1x_2+k_1x_1^3-k_2x_2-k_2k_1x_1
$$


Моделирование системы при $k_1=2,k_2=3,x_0=(1,-0.5)$:
\begin{figure}[H]
    \centering
    \includegraphics[scale=0.55]{2xu.png}
    \caption{Графики $x_i(t),u(t)$}
    \label{fig:2xu}
\end{figure}


Вектор состояния и управление стремятся к нулю с течением времени.


\section{Задание 3}
\subsection{Условие}
Рассмотрим систему:
\begin{align}
    \begin{cases}
        \dot{x}_1=\cos{\left( x_1 \right)}-x_2,\\
        \dot{x}_2=x_1+x_3,\\
        \dot{x}_3=x_1x_3+\left( 2-\sin{\left( x_3 \right)} \right)x_4,\\
        \dot{x}_4=x_2x_3+2u
    \end{cases}
\end{align}


Весь вектор состояния измерим. Необходимо синтезировать
стабилизирующий регулятор на основе метода бэкстеппинга и провести
математическое моделирование.


\subsection{Выполнение}
Система четвёртого порядка в строгой обратносвязной форме:
\begin{align*}
    &\dot{x}_1=f_1(x_1)+g_1x_2,& \ f_1(x_1)=\cos{\left( x_1 \right)},g_1=-1,\\
    &\dot{x}_2=f_2(x_1)+g_2x_3,& \ f_2(x_1)=x_1,g_2=1,\\
    &\dot{x}_3=f_3(x_1,x_3)+g_3(x_3)x_4,&\ f_3(x_1,x_3)=x_1x_3,g_3(x_3)=2-\sin{\left( x_3 \right)},\\
    &\dot{x}_4=f_4(x_2,x_3)+g_4u&\ f_4(x_2,x_3)=x_2x_3,g_4=2
\end{align*}


Так как $g_3(x_3)=2-\sin{\left( x_3 \right)}\geq1>0,g_4=2>0$,
то управление эффективно.


Сделаем рекурсивный бэкстеппинг.


% Для подсистемы:
% $$
% \dot{x}_1=\cos{\left( x_1 \right)}-x_2
% $$
% $x_2$ -- виртуальное управление.


Функция Ляпунова и ее производная:
$$
V_1(x_1)=0.5x_1^2,\ \dot{V}_1(x_1)=x_1\left( \cos{\left( x_1 \right)}-x_2 \right)=x_1\cos{\left( x_1 \right)}-x_1x_2
$$


Виртуальное управление:
$$
\varphi_1(x_1)=\cos{\left( x_1 \right)}+k_1x_1,\ k_1>0
$$


Тогда, при $x_2=\varphi_1(x_1)$:
$$
\dot{x}_1=-k_1x_1\Rightarrow \dot{V}_1(x_1)=-k_1x_1^2<0\,\forall x_1\neq0,k_1>0
$$


Ошибка отслеживания:
$$
z_1=x_2-\varphi_1(x_1)=x_2-\cos{\left( x_1 \right)}-k_1x_1
$$


Тогда:
$$
x_2=z_1+\cos{\left( x_1 \right)}+k_1x_1,
$$
$$
\dot{x}_1=\cos{\left( x_1 \right)}-x_2=-z_1-k_1x_1
$$


Найдем $\dot{z}_1$:
$$
\dot{z}_1=\dot{x}_2-\dot{\varphi}_{1,t}(x_1),
$$
$$
\dot{\varphi}_{1,t}(x_1)=\left(-\sin{\left( x_1 \right)}+k_1\right)\dot{x}_1=\left(\sin{\left( x_1 \right)}-k_1\right)\left( z_1+k_1x_1 \right),
$$
$$
\dot{z}_1=x_1+x_3-\left( \sin{\left( x_1 \right)}-k_1 \right)\left( z_1+k_1x_1 \right)
$$


Расширенная функция Ляпунова:
$$
V_2=V_1+0.5z_1^2,\ V_1=0.5x_1^2
$$


Ее производная:
$$
\dot{V}_2=x_1\dot{x}_1+z_1\dot{z}_1=-x_1\left( z_1+k_1x_1 \right)
+z_1\left( x_1+x_3-\left( \sin{\left( x_1 \right)}-k_1 \right)\left( z_1+k_1x_1 \right) \right),
$$
$$
\dot{V}_2=-k_1x_1^2+z_1x_3-z_1^2\sin{\left( x_1 \right)}-z_1k_1x_1\sin{\left( x_1 \right)}+z_1^2k_1+z_1k_1^2x_1
$$


Виртуальное управление $x_3=\varphi_2(x_1,x_2)$:
$$
x_3=z_1\sin{\left( x_1 \right)}+k_1x_1\sin{\left( x_1 \right)}-z_1k_1-k_1^2x_1-z_1k_2,\ k_2>0
$$


Тогда, после подстановки $z_1x_3$:
$$
\dot{V}_2=-k_1x_1^2-k_2z_1^2<0\,\forall(x_1,z_1)\neq(0,0),k_2>0
$$


Подставим $z_1$ в $\varphi_2(x_1,x_2)$:
\begin{align*}
    \varphi_2(x_1,x_2)=&x_2\sin{\left( x_1 \right)}-\sin{\left( x_1 \right)}\cos{\left( x_1 \right)}
    -k_1x_2+\\ &+k_1\cos{\left( x_1 \right)}-k_2x_2+k_2\cos{\left( x_1 \right)}+k_2k_1x_1
\end{align*}


Введем следующую ошибку отслеживания и проведем аналогичные действия:
$$
z_2=x_3-\varphi_2(x_1,x_2)\Rightarrow \dot{z}_2=\dot{x}_3-\dot{\varphi}_{2,t}(x_1,x_2),
$$
\begin{align*}
    \dot{\varphi}_{2,t}(x_1,x_2)=&\dot{x}_2\sin{\left( x_1 \right)}-k_1\dot{x}_2
    -k_2\dot{x}_2+\dot{x}_1x_2\cos{\left( x_1 \right)}+\\ &-\dot{x}_1\cos{\left( 2x_1 \right)}
    -k_1\dot{x}_1\sin{\left( x_1 \right)}-k_2\dot{x}_1\sin{\left( x_1 \right)}+k_2k_1\dot{x}_1
\end{align*}


Подставим $\dot{x}_1,\dot{x}_2$ в $\dot{\varphi}_{2,t}(x_1,x_2)$:
\begin{align*}
    \dot{\varphi}_{2,t}(x_1,x_2)=&x_1\sin{\left( x_1 \right)}+x_3\sin{\left( x_1 \right)}
    +x_2\cos^2{\left( x_1 \right)}-x_2^2\cos{\left( x_1 \right)}+\\
    &-\cos{\left( x_1 \right)}\cos{\left( 2x_1 \right)}+x_2\cos{\left( 2x_1 \right)}-k_1x_1-k_1x_3+\\
    &-\frac{k_1}{2}\sin{\left( 2x_1 \right)}+k_1x_2\sin{\left( x_1 \right)}-k_2x_1-k_2x_3+\\
    &-\frac{k_2}{2}\sin{\left( 2x_1 \right)}+k_2x_2\sin{\left( x_1 \right)}+k_2k_1\cos{\left( x_1 \right)}-k_2k_1x_2
\end{align*}


Тогда:
\begin{align*}
    \dot{z}_2=&x_1x_3+\left( 2-\sin{\left( x_3 \right)} \right)x_4
    -x_1\sin{\left( x_1 \right)}-x_3\sin{\left( x_1 \right)}+\\
    &-x_2\cos^2{\left( x_1 \right)}+x_2^2\cos{\left( x_1 \right)}+
    \cos{\left( x_1 \right)}\cos{\left( 2x_1 \right)}-x_2\cos{\left( 2x_1 \right)}+\\
    &+k_1x_1+k_1x_3+\frac{k_1}{2}\sin{\left( 2x_1 \right)}-k_1x_2\sin{\left( x_1 \right)}+k_2x_1+k_2x_3+\\
    &+\frac{k_2}{2}\sin{\left( 2x_1 \right)}-k_2x_2\sin{\left( x_1 \right)}-k_2k_1\cos{\left( x_1 \right)}+k_2k_1x_2
\end{align*}


Расширенная функция Ляпунова:
$$
V_3=V_2+0.5z_2^2,\ V_2=0.5x_1^2+0.5z_1^2
$$


Ее производная:
$$
\dot{V}_3=-k_1x_1^2-k_2z_1^2+z_2\dot{z}_2
$$


При умножении $z_2$ на $\dot{z}_2$
все слагаемые, кроме $zx_4\left( 2-\sin{\left( x_3 \right)} \right)$,
получатся неопределенными по знаку.
Нужно компенсировать их с помощью
виртуального управления $x_4=\varphi_3(x_1,x_2,x_3)$,
а также добавить отрицательно определенное слагаемое с $z_2$:
\begin{align*}
    \varphi_3(x_1,x_2,x_3)=&\frac{1}{2-\sin{\left( x_3 \right)}}\left(-x_1x_3\right.
    +x_1\sin{\left( x_1 \right)}+x_3\sin{\left( x_1 \right)}+\\
    &+x_2\cos^2{\left( x_1 \right)}-x_2^2\cos{\left( x_1 \right)}
    -\cos{\left( x_1 \right)}\cos{\left( 2x_1 \right)}+x_2\cos{\left( 2x_1 \right)}+\\
    &-k_1x_1-k_1x_3-\frac{k_1}{2}\sin{\left( 2x_1 \right)}+k_1x_2\sin{\left( x_1 \right)}
    -k_2x_1-k_2x_3+\\
    &-\frac{k_2}{2}\sin{\left( 2x_1 \right)}+k_2x_2\sin{\left( x_1 \right)}
    \left.+k_2k_1\cos{\left( x_1 \right)}-k_2k_1x_2-k_3z_2\right),\\
    &k_3>0
\end{align*}


% Получилось выражение в общем виде:
% $$
% \varphi_3(x_1,x_2,x_3)=\frac{1}{2-\sin{\left( x_3 \right)}}\left( -x_1x_3+\dot{\varphi}_{2,t}(x_1,x_2)-k_3\left( x_3-\varphi_2\left( x_1,x_2 \right) \right) \right)
% $$


% При этом:
% $$
% \dot{z}_2=x_1x_3+\left( 2-\sin{\left( x_3 \right)} \right)x_4-\dot{\varphi}_{2,t}(x_1,x_2)
% $$


% Подставим в $\dot{z}_2$ виртуальное управление $x_4=\varphi(x_1,x_2,x_3)$ в общем виде:
% $$
% \dot{z}_2=-k_3\left( x_3-\varphi_2(x_1,x_2) \right)=-k_3z_2
% $$


Тогда:
$$
\dot{V}_3=-k_1x_1^2-k_2z_1^2-k_3z_2^2<0\,\forall(x_1,z_1,z_2)\neq(0,0,0),k_1,k_2,k_3>0
$$


% Таким образом, в данном случае при синтезе регулятора методом бэкстеппинга
% достаточно использовать производные $\dot{\varphi}_{i,t}$ в общем виде,
% так как взятие производных не порождает отрицательно определенные
% по знаку слагаемые, которые не нужно компенсировать виртуальным
% или реальным управлением, а значит $\dot{\varphi}_{i,t}$
% нужно компенсировать полностью. Это упростит решение в дальнейшем.
% Вычисления производных оставим программам уже при расчете системы.


Введем следующую ошибку и выполним аналогичные действия:
$$
z_3=x_4-\varphi_3(x_1,x_2,x_3)\Rightarrow \dot{z}_3=\dot{x}_4-\dot{\varphi}_{3,t}(x_1,x_2,x_3)
$$


Расширенная функция Ляпунова:
$$
V(x_1,z_1,z_2,z_3)=V_3+0.5z_3^2,\ V_3=0.5x_1^2+0.5z_1^2+0.5z_2^2
$$


Ее производная:
$$
\dot{V}(x_1,z_1,z_2,z_3)=-k_1x_1^2-k_2z_1^2-k_3z_2^2+z_3\dot{z}_3
$$


Рассмотрим $z_3\dot{z}_3$:
$$
z_3\dot{z}_3=z_3\left( x_2x_3+2u-\dot{\varphi}_{3,t}(x_1,x_2,x_3) \right)
$$


Достаточно компенсировать реальным управлением влияние $x_2x_3$ и $\dot{\varphi}_{3,t}$
и добавить слагаемое с $z_3$, которое в производной функции Ляпунова будет отрицательно
определенным:
$$
u=\frac{1}{2}\left( -x_2x_3+\dot{\varphi}_{3,t}(x_1,x_2,x_3) -k_4\left( x_4-\varphi_3(x_1,x_2,x_3) \right)\right),\ k_4>0
$$


Таким образом:
$$
\dot{V}=-k_1x_1^2-k_2z_1^2-k_3z_2^2-k_4z_3\left( x_4-\varphi_3(x_1,x_2,x_3) \right),
$$
$$
\dot{V}=-k_1x_1^2-k_2z_1^2-k_3z_2^2-k_4z_3^2<0\,\forall(x_1,z_1,z_2,z_3)\neq(0,0,0,0),k_i>0
$$


Начало координат глобально асимптотически устойчиво с законом
управления:
$$
u=\frac{1}{2}\left( -x_2x_3+\dot{\varphi}_{3,t}(x_1,x_2,x_3) -k_4\left( x_4-\varphi_3(x_1,x_2,x_3) \right)\right),\ k_i>0
$$


Моделирование системы при $k_i=1,x_0=(0.5, 0, 0, 0)$:
\begin{figure}[H]
    \centering
    \includegraphics[scale=0.65]{3xu.png}
    \caption{Графики $x_i(t),u(t)$}
    \label{fig:3xu}
\end{figure}


Система с данным законом управления стремится к положению
равновесия $x=(0,1,0,0)$, управление стремится к нулю.


Рассмотрим $x_2(t)$:
$$
x_2(t)=\varphi_1(x_1(t))+z_1(t)\xrightarrow[t\to\infty]{}\varphi_1(0), z_1(t)\xrightarrow[t\to\infty]{}0,
$$
$$
\varphi_1(x_1(t)=0)=\cos{\left( 0 \right)}+k_1\cdot0=1\Rightarrow x_2(t)\xrightarrow[t\to\infty]{}1
$$


Чтобы $x_2(t)\xrightarrow[t\to\infty]{}0$ необходимо,
чтобы ошибка $z_1(t)$ была ненулевой для $\varphi_1(x_1)$,
однако синтез бэкстеппинг регулятора подразумевает
стремление всех ошибок к нулю в устойчивом состоянии.
\vfill


\section{Вывод}
В ходе выполнения
данной лабораторной работы
были синтезированы
регуляторы на основе метода
бэкстеппинга для различных нелинейных
систем. Было выполнено моделирование
систем, показывающее, что
регуляторы синтезированы корректно.
\end{document}