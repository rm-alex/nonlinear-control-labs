\documentclass[a4paper,14pt]{extarticle}

\usepackage[T2A]{fontenc}
\usepackage[utf8]{inputenc}
\usepackage[english, russian]{babel}

\usepackage[left=30mm, right=10mm, top=20mm, bottom=20mm]{geometry}

\usepackage{tempora}
\usepackage{setspace}
\onehalfspacing

\usepackage{titlesec}
\titleformat{\section}[block]{\bfseries\centering\MakeUppercase}{\thesection.}{1em}{}
\titleformat{\subsection}[block]{\bfseries}{\thesubsection.}{1em}{}
\titleformat{\subsubsection}[block]{\bfseries}{\thesubsubsection.}{1em}{}

\renewcommand{\contentsname}{\hfill \textbf{СОДЕРЖАНИЕ} \hfill\null}

\usepackage{indentfirst}
\setlength{\parindent}{1.25cm}

\usepackage{amsmath, amsfonts, amssymb}
\usepackage{graphicx}
\usepackage{caption}
\usepackage{subcaption}
\usepackage{float}
\usepackage{tikz}
\usetikzlibrary{patterns}
\usepackage{cmap}
\usepackage{hyperref}
\usepackage{xcolor}
\usepackage{listings}

\definecolor{LightGray}{gray}{0.7}

\lstdefinestyle{code}{
    language=Python, % change if needed
    basicstyle=\small\ttfamily,
    numbers=left,
    numberstyle=\small\color{LightGray},
    stepnumber=1,
    numbersep=5pt,
    backgroundcolor=\color{white},
    showspaces=false,
    showstringspaces=false,
    showtabs=false,
    tabsize=4,
    captionpos=b,
    breaklines=true,
    breakatwhitespace=false,
    frame=single,
    rulecolor=\color{LightGray},
    linewidth=\linewidth,
    keywordstyle=\color{blue}\bfseries,
    commentstyle=\color{green!40!black},
    stringstyle=\color{violet},
    escapeinside={\%*}{*)},
    xleftmargin=10pt,
    xrightmargin=10pt,
    framexleftmargin=0pt,
    framexrightmargin=0pt
}
\lstset{style=code}

\hypersetup{
    colorlinks=true,
    linkcolor=blue,
    filecolor=magenta,
    urlcolor=cyan,
    pdftitle={ncs1},
    pdfauthor={Rumyantsev Alexey},
    pdfsubject={control},
    pdfkeywords={LaTeX, PDF},
    pdfpagemode=FullScreen,
}

\graphicspath{{src/images/}}

\begin{document}

\begin{titlepage}
    \begin{center}
        МИНИСТЕРСТВО НАУКИ И ВЫСШЕГО ОБРАЗОВАНИЯ РОССИЙСКОЙ ФЕДЕРАЦИИ\\
        \vspace*{2.5mm}
        Федеральное государственное автономное образовательное учреждение высшего образования
        «НАЦИОНАЛЬНЫЙ ИССЛЕДОВАТЕЛЬСКИЙ УНИВЕРСИТЕТ ИТМО»\\
        \vspace*{2.5mm}
        \textbf{ФАКУЛЬТЕТ СИСТЕМ УПРАВЛЕНИЯ И РОБОТОТЕХНИКИ}
        \vfill

        {\large ОТЧЕТ ПО ЛАБОРАТОРНОЙ РАБОТЕ №3}\\
        {\large по дисциплине}\\
        {\large\bfseries «НЕЛИНЕЙНЫЕ СИСТЕМЫ УПРАВЛЕНИЯ»}\\
        {\large на тему}\\
        {\large\bfseries «ЛИНЕАРИЗАЦИЯ ПО ВХОДУ–ВЫХОДУ И ПРОЕКТИРОВАНИЕ ОБРАТНОЙ СВЯЗИ ПО СОСТОЯНИЮ»}\\
        \vfill

        \begin{flushright}
            Выполнил: студент гр. R3441\\
            Румянцев А. А.\medskip\\

            Проверил: преподаватель\\
            Зименко К. А.
        \end{flushright}

        \vfill

        Санкт-Петербург\\
        2025
    \end{center}
\end{titlepage}

\setcounter{page}{2}
\tableofcontents
\newpage

\section{Задание 1}
\subsection{Условие}
Для данной системы определить:
\begin{itemize}
    \item Является ли эта система линеаризуемой по входу–выходу?
    \item Если да, преобразуйте её в нормальную форму и укажите область определения соответствующего преобразования
    \item Является ли эта система минимально-фазовой?
\end{itemize}


\subsection{Выполнение}
Рассмотрим систему:
\begin{align}
    \begin{cases}
        \dot{x}_1=-x_1+x_2-x_3,\\
        \dot{x}_2=-x_1x_3-x_2+u,\\
        \dot{x}_3=-x_1+u,\\
        y=x_3
    \end{cases}\label{syseq:1}
\end{align}


Точки равновесия при $\dot{x}=0,u=0$:
$$
\begin{cases}
        -x_1+x_2-x_3=0,\\
        -x_1x_3-x_2=0,\\
        -x_1=0
    \end{cases}\Rightarrow\begin{cases}
        x_1=0,\\x_2=0,\\x_3=x_2
    \end{cases}\Rightarrow\left( 0,0,0 \right)
$$


Нелинейная система:
$$
\dot{x}=f(x)+g(x)u,\ f(x)=\begin{bmatrix}
    -x_1+x_2-x_3\\-x_1x_3-x_2\\-x_1
\end{bmatrix},\ g(x)=\begin{bmatrix}
    0\\1\\1
\end{bmatrix}
$$


Производная от выхода $y=h(x)$:
$$
\dot{y}=L_fh(x)+L_gh(x)u
$$


Производная Ли функции h вдоль поля f:
$$
L_fh(x)=\frac{\partial h}{\partial x}f(x)=\begin{bmatrix}
    0&0&1
\end{bmatrix}\begin{bmatrix}
    -x_1+x_2-x_3\\-x_1x_3-x_2\\-x_1
\end{bmatrix}=-x_1
$$


Вдоль поля g:
$$
L_gh(x)=\frac{\partial h}{\partial x}g(x)=\begin{bmatrix}
    0&0&1
\end{bmatrix}\begin{bmatrix}
    0\\1\\1
\end{bmatrix}=1
$$


Производная выхода:
$$
\dot{y}=\dot{x}_3=-x_1+u
$$


Первая производная выхода зависит от $u$,
следовательно относительная степень
нелинейной системы $\rho=1$.


Так как $L_gL_f^{\rho-1}h(x)=L_gh(x)=1\neq0$, $\rho<2$ и $\rho=1<n=3$, то система линеаризуема по входу–выходу
(вход появляется именно в $\rho$-й производной; так как производная потребовалась всего одна, то
условие на непоявление входа раньше $\rho$-й производной $L_gL_f^kh(x)=0,k\in\left[ 0,\rho-2 \right]$ проверять не нужно;
$\rho\leq n$ -- относительная степень не больше порядка системы).


% Обратная связь:
% $$
% u=\frac{-L_f^{\rho}h(x)+v}{L_gL_f^{\rho-1}h(x)}=\frac{-L_fh(x)+v}{L_gh(x)}=x_1+v
% $$


% Подставим полученное $u$ в производную выхода:
% $$
% \dot{y}=-x_1+\left( x_1+v \right)=v
% $$


% Динамика между $v$ и $y$ сведена к одному интегратору.


Замена переменных:
$$
z=T(x)=\begin{bmatrix}
    \phi_1(x)\\ \phi_2(x) \\ h(x)
\end{bmatrix}=\begin{bmatrix}
    \eta_1\\\eta_2\\\xi
\end{bmatrix}
$$


Условие:
$$
\frac{\partial \phi_i}{\partial x}g(x)=0,\ i=1,...,n-\rho
$$


Исходя из $g(x)$:
$$
\frac{\partial \phi_i}{\partial x_2}+\frac{\partial \phi_i}{\partial x_3}=0
$$


Пусть:
$$
\begin{cases}
    \phi_1=x_1,\\
    \phi_2=x_2-x_3
\end{cases}
$$


Тогда:
$$
\frac{\partial \phi_1}{\partial x}g(x)=\begin{bmatrix}
    1&0&0
\end{bmatrix}\begin{bmatrix}
    0\\1\\1
\end{bmatrix}=0,
$$
$$
\frac{\partial \phi_2}{\partial x}g(x)=\begin{bmatrix}
    0&1&-1
\end{bmatrix}\begin{bmatrix}
    0\\1\\1
\end{bmatrix}=1-1=0
$$



Проверим обратимость преобразования $z=T(x)$:
$$
z=\begin{bmatrix}
    x_1\\x_2-x_3\\x_3
\end{bmatrix}\Rightarrow J_T(x)=\frac{\partial\left( \eta_1,\eta_2,\xi \right)}{\partial\left( x_1,x_2,x_3 \right)}
=\begin{bmatrix}
    1&0&0\\0&1&-1\\0&0&1
\end{bmatrix},
$$
$$
\det{J_T(x)}=1\neq0\,\forall x\in\mathbb{R}^3
$$


Преобразование $T(x)$ является гладко обратимым в $\mathbb{R}^3$,
т.е. глобальным диффеоморфизмом.


Область определения соответствующего преобразования:
$$
D_0=\left\{ x\in\mathbb{R}^3|\det{J_T(x)}\neq0 \right\}=\mathbb{R}^3
$$


Преобразование координат $x=T^{-1}(z)$:
$$
\begin{cases}
    x_1=\eta_1,\\x_2=\eta_2+\xi,\\x_3=\xi
\end{cases}
$$


Вычислим производные:
$$
\dot{\eta}_1=\dot{x}_1=-x_1+x_2-x_3=-\eta_1+\eta_2+\xi-\xi=-\eta_1+\eta_2,
$$
$$
\dot{\eta}_2=\dot{x}_2-\dot{x}_3=-x_1x_3-x_2+u+x_1-u=-x_1x_3-x_2+x_1=-\eta_1\xi- \eta_2-\xi+\eta_1,
$$
$$
\dot{\xi}=\dot{x}_3=-x_1+u=-\eta_1+u
$$


Система в нормальной (канонической) форме:
$$
\begin{cases}
    \dot{\eta}_1=-\eta_1+\eta_2,\\
    \dot{\eta}_2=-\eta_1\xi-\eta_2-\xi+\eta_1,\\
    \dot{\xi}=-\eta_1+u,\\y=\xi
\end{cases}
$$


Проверка равновесия в нормальной форме:
$$
\left( \eta_1,\eta_2,\xi \right)=\left( 0,0,0 \right),u=0\Rightarrow\dot{\eta}=0,\dot{\xi}=0
$$


% Каноническое представление цепи $\rho$ интеграторов (один интегратор при $\rho=1$):
% $$
% \left( A_c,B_c,C_c \right)=\left( 0,1,1 \right)
% $$


% Внешняя составляющая $\xi\in\mathbb{R}^{\rho=1}$, внутренняя
% составляющая $\eta\in\mathbb{R}^{n-\rho=2}$.


% Линеаризация внешней составляющей:
% $$
% u=\alpha(x)+\gamma(x)^{-1}v
% $$
% $$
% \alpha(x)=-\frac{L_f^\rho h(x)}{L_gL_f^{\rho-1}h(x)}=x_1=\eta_1,
% $$
% $$
% \gamma(x)=L_gL_f^{\rho-1}h(x)=1,
% $$
% $$
% u=\eta_1+v
% $$


% Подстановка:
% $$
% \dot{\xi}=u-\eta_1=\eta_1-\eta_1+v=v
% $$


% Внешняя подсистема (по входу–выходу) становится линейной интегрирующей связью $\dot{y}=v$.


Нулевая динамика:
$$
\dot{\eta}=f_0(\eta,\xi=0)%\Rightarrow \xi=y=0, v=0
$$


Подставим $\xi=0$ в $\dot{\eta}$:
$$
\begin{cases}
    \dot{\eta}_1=-\eta_1+\eta_2,\\
    \dot{\eta}_2=-\eta_2+\eta_1
\end{cases}\Rightarrow\dot{\eta}=\begin{bmatrix}
    -1&1\\1&-1
\end{bmatrix}\eta
$$


Собственные числа:
$$
\det{\begin{vmatrix}
    -1-\lambda&1\\1&-1-\lambda
\end{vmatrix}}=\lambda\left( \lambda+2 \right)=0\Rightarrow\sigma=\left\{ 0,-2 \right\}
$$


Система неасимптотически устойчива.


Система называется минимально-фазовой,
если начало координат для уравнения нуль-динамики асимптотически устойчиво
при $T(x)$ таком, что начало координат $\left( \eta=0,\xi=0 \right)$ является
точкой равновесия.


Следовательно, система не минимально-фазовая.


\section{Задание 2}
\subsection{Условие}
На основе метода линеаризации обратной связью найдите закон
управления с обратной связью по состоянию, обеспечивающий
глобальную стабилизацию начала координат для данной системы.


\subsection{Выполнение}
Рассмотрим систему:
\begin{align}
    \begin{cases}
        \dot{x}_1=-x_1+x_2,\\
        \dot{x}_2=x_1-x_2-x_1x_3+u,\\
        \dot{x}_3=x_1+x_1x_2-2x_3
    \end{cases}\label{syseq:2}
\end{align}


Нелинейная система:
$$
\dot{x}=f(x)+g(x)u,\ f(x)=\begin{bmatrix}
    -x_1+x_2\\x_1-x_2-x_1x_3\\x_1+x_1x_2-2x_3
\end{bmatrix},\ g(x)=\begin{bmatrix}
    0\\1\\0
\end{bmatrix}
$$


Управление влияет только на $\dot{x}_2$.


Линеаризация обратной связью:
$$
u=\alpha(x)+\beta(x)v
$$
% Динамика системы с таким законом управления должна стать линейной.


Компенсируем нелинейность $-x_1x_3$ в $\dot{x}_2$ с помощью $u$:
$$
\alpha(x)=x_1x_3\Rightarrow u=x_1x_3+\beta(x)v
$$


Тогда $\dot{x}_2$:
$$
\dot{x}_2=x_1-x_2-x_1x_3+x_1x_3+\beta(x)v=x_1-x_2+\beta(x)v
$$


Выберем линейный закон такой, чтобы $\left( x_1,x_2 \right)\to0$:
$$
v=-k_1x_1-k_2x_2,\ \beta(x)=1\Rightarrow u=x_1x_3-k_1x_1-k_2x_2
$$


Нелинейность $x_1x_2$ в $\dot{x}_3$
косвенно компенсируется,
т.к. $\left( x_1,x_2 \right)\to0\Rightarrow x_1x_2\to0$.


Подставим $\beta(x)v$ в $\dot{x}_2$:
$$
\dot{x}_2=x_1-x_2-k_1x_1-k_2x_2=-\left( k_1-1 \right)x_1-\left( k_2+1 \right)x_2
$$


Система примет вид:
$$
\begin{cases}
    \dot{x}_1=-x_1+x_2,\\
        \dot{x}_2=-\left( k_1-1 \right)x_1-\left( k_2+1 \right)x_2,\\
        \dot{x}_3=x_1+x_1x_2-2x_3
\end{cases}
$$


Рассмотрим линейную подсистему по состояниям $\left( x_1,x_2 \right)$:
$$
\begin{cases}
    \dot{x}_1=-x_1+x_2,\\
        \dot{x}_2=-\left( k_1-1 \right)x_1-\left( k_2+1 \right)x_2
\end{cases}
$$


Найдем матрицу $A$:
$$
A=\begin{bmatrix}
    -1&1\\-\left( k_1-1 \right)&-\left( k_2+1 \right)
\end{bmatrix}
$$


Чтобы матрица $A$ была Гурвицевой, должны выполняться условия:
$$
\begin{cases}
    \operatorname{trace}{A}<0,\\\det{A}>0,\\\Re{\lambda_i}<0
\end{cases}\Rightarrow\begin{cases}-k_2-2<0,\\ k_1+k_2>0\end{cases}
\Rightarrow \begin{cases}
    k_2>-2,\\ k_1>-k_2
\end{cases}
$$


Тогда линейная подсистема будет экспоненциально устойчива:
$$
||\left( x_1(t),x_2(t) \right)||\leq Ce^{-\alpha t}||\left( x_1(0),x_2(0) \right)||,\ C>0,\ \alpha>0
$$


Рассмотрим $\dot{x}_3$:
$$
\dot{x}_3=-2x_3+x_1\left( x_2+1 \right)=-2x_3+f(t)
$$


Рассмотрим возмущение $f(t)$:
$$
f(t)=x_1(t)\left( x_2(t)+1 \right)
$$


Ее убывание:
$$
|x_1(t)|\leq C_1e^{-\alpha t},\ |x_2(t)|\leq C_2e^{-\alpha t},
$$
$$
|f(t)|=|x_1(t)\left( x_2(t)+1 \right)|\leq C_1e^{-\alpha t}\left( C_2e^{-\alpha t}+1 \right)\leq \left( C_1 + C_1C_2 \right)e^{-\alpha t}= Ce^{-\alpha t}
$$


То есть:
$$
|f(t)|\leq Ce^{-\alpha t}
$$


Решение линейного уравнения с известным источником $f(t)$ в общем виде:
$$
\dot{x}(t)=\alpha x(t)+f(t),\ x_h(t)=e^{\alpha t}x(0),
$$
$$
x(t)=e^{\alpha t}C(t),\ \dot{C}(t)=e^{-\alpha t}f(t)\Rightarrow C(t)=\int\limits_{0}^{t}e^{-\alpha s}f(s)\,ds +C_0,
$$
$$
x(t)=e^{\alpha t}C_0+e^{\alpha t}\int\limits_{0}^{t}e^{-\alpha s}f(s)\,ds=e^{\alpha t}x(0)+\int\limits_{0}^{t}e^{\alpha \left( t-s \right)}f(s)\,ds,
$$


Решим $\dot{x}_3=-2x_3+f(t)$:
$$
x_3(t)=e^{-2 t}x(0)+\int\limits_{0}^{t}e^{-2\left( t-s \right)}f(s)\,ds
$$


Первый член стремится к нулю экспоненциально.


Оценим интеграл при $|f(s)|\leq Ce^{-\alpha s}$:
$$
\Bigg|\int\limits_{0}^{t}e^{-2 \left( t-s \right)}f(s)\,ds\Bigg|\leq
\int\limits_{0}^{t}e^{-2 \left( t-s \right)}Ce^{-\alpha s}\,ds=Ce^{-2t}\int\limits_{0}^{t}e^{\left(2-\alpha\right) s}\,ds,
$$
$$
\int\limits_{0}^{t}e^{\left(2-\alpha\right) s}\,ds=\begin{cases}
    \frac{1}{2-\alpha}\left( e^{\left( 2-\alpha \right)t} -1\right),&\alpha\neq2,\\
    t,&\alpha=2
\end{cases}
$$


В случае $\alpha\neq2$ для больших $t$ главное слагаемое: $e^{\left( 2-\alpha \right)t}$.


Тогда интеграл можно оценить:
$$
e^{-2t}\int\limits_{0}^{t}e^{\left(2-\alpha\right) s}\,ds\sim e^{-2t}\cdot e^{\left(2-\alpha\right) t}=e^{-\alpha t}
$$


Следовательно, интеграл убывает экспоненциально с тем же показателем $\alpha$, что и $f(t)$.


Таким образом, $x_3(t)$ и $\dot{x}_3(t)$ убывают экспоненциально.


Подобранный закон управления обеспечивает глобальную стабилизацию начала координат данной системы.


Промоделируем систему при $$k_2=-1,k_1=4,$$ $$x_0=\begin{bmatrix}
    100\\ -70\\ 45
\end{bmatrix}$$
\begin{figure}[H]
    \centering
    \includegraphics[scale=0.5]{sch.png}
    \caption{Схема моделирования системы}
    \label{fig:sch}
\end{figure}


Графики вектора состояния системы и управления: % 1001,205,953,498
\begin{figure}[H]
    \centering
    \includegraphics[scale=0.6]{x.png}
    \caption{Вектор состояния объекта управления}
    \label{fig:x}
\end{figure}
\begin{figure}[H]
    \centering
    \includegraphics[scale=0.6]{u.png}
    \caption{Управление $u=x_1x_3-k_1x_1-k_2x_2$}
    \label{fig:u}
\end{figure}


\section{Вывод}
В ходе выполнения работы
было выяснено, что первая
нелинейная система
является линеаризуемой по входу-выходу.
Она была преобразована к канонической форме,
область определения определена как $\mathbb{R}^3$.
Система не является минимально фазовой.


Для второй нелинейной системы удалось
найти закон управления с обратной связью по состоянию,
обеспечивающий глобальную стабилизацию начала координат.
Использовался метод линеаризации обратной связью.


% Докажем глобальную стабилизацию начала координат.


% Функция Ляпунова:
% $$
% V=\frac{1}{2}\left( x_1^2+x_2^2+x_3^2 \right)
% $$


% Ее производная:
% $$
% \dot{V}=x_1\dot{x}_1+x_2\dot{x}_2+x_3\dot{x}_3=
% -x_1^2-\left( k_2+1 \right)x_2^2-2x_3^2+x_1x_2\left( 2-k_1+x_3 \right)+x_1x_3
% $$


% При $k_2>-2$ член $-\left( k_2+1 \right)$ всегда неположителен.


% Рассмотрим члены, которые могут быть неотрицательны.


% По неравенству типа Коши-Буняковского:
% $$
% |x_1x_3|\leq\frac{1}{2}\left( x_1^2+x_3^2 \right),
% $$
% $$
% |x_1x_2\left( 2-k_1+x_3 \right)|\leq\frac{1}{2}|2-k_1+x_3|\left( x_1^2+x_2^2 \right)
% $$


% Неравенство сверху:
% $$
% \dot{V}\leq -x_1^2-\left( k_2+1 \right)x_2^2-2x_3^2+\frac{1}{2}|2-k_1+x_3|\left( x_1^2+x_2^2 \right)+\frac{1}{2}\left( x_1^2+x_3^2 \right),
% $$
% $$
% \dot{V}\leq -0.5x_1^2\left(1-|2-k_1+x_3| \right)-x_2^2\left( 1+k_2-0.5\cdot|2-k_1+x_3| \right)-1.5x_3^2
% $$


% Чтобы $\dot{V}<0\,\forall x\neq0$ необходимо, чтобы все скобки были положительными:
% $$
% 1-|2-k_1+x_3|>0\Rightarrow|2-k_1+x_3|<1
% $$
\end{document}