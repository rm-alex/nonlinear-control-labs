\documentclass[a4paper,14pt]{extarticle}

\usepackage[T2A]{fontenc}
\usepackage[utf8]{inputenc}
\usepackage[english, russian]{babel}

\usepackage[left=30mm, right=10mm, top=20mm, bottom=20mm]{geometry}

\usepackage{tempora}
\usepackage{setspace}
\onehalfspacing

\usepackage{titlesec}
\titleformat{\section}[block]{\bfseries\centering\MakeUppercase}{\thesection.}{1em}{}
\titleformat{\subsection}[block]{\bfseries}{\thesubsection.}{1em}{}
\titleformat{\subsubsection}[block]{\bfseries}{\thesubsubsection.}{1em}{}

\renewcommand{\contentsname}{\hfill \textbf{СОДЕРЖАНИЕ} \hfill\null}

\usepackage{indentfirst}
\setlength{\parindent}{1.25cm}

\usepackage{amsmath, amsfonts, amssymb}
\usepackage{graphicx}
\usepackage{caption}
\usepackage{subcaption}
\usepackage{float}
\usepackage{tikz}
\usetikzlibrary{patterns}
\usepackage{cmap}
\usepackage{hyperref}
\usepackage{xcolor}
\usepackage{listings}

\definecolor{LightGray}{gray}{0.7}

\lstdefinestyle{code}{
    language=Python, % change if needed
    basicstyle=\small\ttfamily,
    numbers=left,
    numberstyle=\small\color{LightGray},
    stepnumber=1,
    numbersep=5pt,
    backgroundcolor=\color{white},
    showspaces=false,
    showstringspaces=false,
    showtabs=false,
    tabsize=4,
    captionpos=b,
    breaklines=true,
    breakatwhitespace=false,
    frame=single,
    rulecolor=\color{LightGray},
    linewidth=\linewidth,
    keywordstyle=\color{blue}\bfseries,
    commentstyle=\color{green!40!black},
    stringstyle=\color{violet},
    escapeinside={\%*}{*)},
    xleftmargin=10pt,
    xrightmargin=10pt,
    framexleftmargin=0pt,
    framexrightmargin=0pt
}
\lstset{style=code}

\hypersetup{
    colorlinks=true,
    linkcolor=blue,
    filecolor=magenta,
    urlcolor=cyan,
    pdftitle={ncs1},
    pdfauthor={Rumyantsev Alexey},
    pdfsubject={control},
    pdfkeywords={LaTeX, PDF},
    pdfpagemode=FullScreen,
}

\graphicspath{{src/images/}}

\begin{document}

\begin{titlepage}
    \begin{center}
        МИНИСТЕРСТВО НАУКИ И ВЫСШЕГО ОБРАЗОВАНИЯ РОССИЙСКОЙ ФЕДЕРАЦИИ\\
        \vspace*{2.5mm}
        Федеральное государственное автономное образовательное учреждение высшего образования
        «НАЦИОНАЛЬНЫЙ ИССЛЕДОВАТЕЛЬСКИЙ УНИВЕРСИТЕТ ИТМО»\\
        \vspace*{2.5mm}
        \textbf{ФАКУЛЬТЕТ СИСТЕМ УПРАВЛЕНИЯ И РОБОТОТЕХНИКИ}
        \vfill

        {\large ОТЧЕТ ПО ЛАБОРАТОРНОЙ РАБОТЕ №2}\\
        {\large по дисциплине}\\
        {\large\bfseries «НЕЛИНЕЙНЫЕ СИСТЕМЫ УПРАВЛЕНИЯ»}\\
        {\large на тему}\\
        {\large\bfseries «ФУНКЦИЯ ЛЯПУНОВА, УСТОЙЧИВОСТЬ И РЕГУЛЯТОРЫ»}\\
        \vfill

        \begin{flushright}
            Выполнил: студент гр. R3441\\
            Румянцев А. А.\medskip\\

            Проверил: преподаватель\\
            Зименко К. А.
        \end{flushright}

        \vfill

        Санкт-Петербург\\
        2025
    \end{center}
\end{titlepage}

\setcounter{page}{2}
\tableofcontents
\newpage

\section{Задание 1}
\subsection{Условие}
Для каждой из данных систем используйте кандидат
квадратичной функции Ляпунова, чтобы показать, что начало
координат асимптотически устойчиво.


\subsection{Выполнение}
\subsubsection{Первая система}
Рассмотрим систему:
\begin{align}
    \begin{cases}
        \dot{x}_1=-x_1+x_1x_2,\\
        \dot{x}_2=-2x_2
    \end{cases}
\end{align}\label{syseq:1}


Функция Ляпунова в квадратичной форме:
$$
V(x)=x^TPx,\text{ пусть } P=\begin{bmatrix}
    0.5&0\\0&0.5
\end{bmatrix}\Rightarrow V(x)=\begin{bmatrix}
    x_1&x_2
\end{bmatrix}\begin{bmatrix}
    0.5&0\\0&0.5
\end{bmatrix}\begin{bmatrix}
    x_1\\x_2
\end{bmatrix},
$$
$$
V=\frac{1}{2}\left( x_1^2+x_2^2 \right),\ V\left( x_1,x_2 \right)>0\, \forall \left( x_1,x_2 \right)\neq\left( 0,0 \right),\ V\left( 0,0 \right)=0
$$


Ее производная:
$$
\dot{V}=x_1\dot{x}_1+x_2\dot{x}_2=x_1\left( -x_1+x_1x_2 \right)+x_2\left( -2x_2 \right)=-x_1^2+x_1^2x_2-2x_2^2,
$$
$$
\dot{V}=-x_1^2\left( 1-x_2 \right)-2x_2^2
$$


Скобка должна быть неотрицательной:
$$
1-x_2\geq0\Rightarrow x_2\leq1
$$


Тогда:
$$
\forall\left( x_1,x_2 \right)\neq\left( 0,0 \right),x_2\leq1:\dot{V}<0
$$


В окрестности нуля $\dot{V}<0$, следовательно начало координат локально асимптотически устойчиво.


Глобальная асимптотическая устойчивость достигалась бы в случае,
когда $\forall \left( x_1,x_2 \right)\in\mathbb{R}^2\,\backslash\left\{ 0 \right\}:\dot{V}<0$.


\subsubsection{Вторая система}
Рассмотрим систему:
\begin{align}
    \begin{cases}
        \dot{x}_1=-x_2-x_1\left( 1-x_1^2-x_2^2 \right),\\
        \dot{x}_2=x_1-x_2\left( 1-x_1^2-x_2^2 \right)
    \end{cases}\label{syseq:2}
\end{align}


Функция Ляпунова:
$$
V=\frac{1}{2}\left( x_1^2+x_2^2 \right),\ V\left( x_1,x_2 \right)>0\, \forall \left( x_1,x_2 \right)\neq\left( 0,0 \right),\ V\left( 0,0 \right)=0
$$


Ее производная:
$$
\dot{V}=x_1\dot{x}_1+x_2\dot{x}_2=x_1\left( -x_2-x_1\left( 1-x_1^2-x_2^2 \right) \right)+x_2\left( x_1-x_2\left( 1-x_1^2-x_2^2 \right) \right),
$$
$$
\dot{V}=x_1^4+x_2^4+2x_1^2x_2^2-x_1^2-x_2^2=\left( x_1^2+x_2^2 \right)\left( x_1^2+x_2^2 -1\right)
$$


Сделаем замену $r^2=x_1^2+x_2^2$:
$$
\dot{V}=r^2\left( r^2-1 \right)
$$


Тогда:
\begin{align*}
    &r^2-1<0\Rightarrow r\in\left( -1,1 \right):\dot{V}<0,\\
    &r^2-1=0\Rightarrow r=\pm 1:\dot{V}=0,\\
    &r^2-1>0\Rightarrow r\in\left( -\infty,-1 \right)\vee \left( 1,\infty \right):\dot{V}>0
\end{align*}


Внутри единичной окружности $r^2=x_1^2+x_2^2<1$ начало координат
локально асимптотически устойчиво.


Локальная асимптотическая устойчивость исключает глобальную.


\subsubsection{Третья система}
Рассмотрим систему:
\begin{align}
    \begin{cases}
        \dot{x}_1=x_2\left( 1-x_1^2 \right)-2x_1,\\
        \dot{x}_2=-\left( x_1+x_2 \right)\left( 1-x_1^2 \right)
    \end{cases}\label{syseq:3}
\end{align}


Функция Ляпунова:
$$
V=\frac{1}{2}\left( x_1^2+x_2^2 \right),\ V\left( x_1,x_2 \right)>0\, \forall \left( x_1,x_2 \right)\neq\left( 0,0 \right),\ V\left( 0,0 \right)=0
$$


Ее производная:
$$
\dot{V}=x_1\dot{x}_1+x_2\dot{x}_2=x_1\left( x_2\left( 1-x_1^2 \right)-2x_1 \right)+x_2\left( -\left( x_1+x_2 \right)\left( 1-x_1^2 \right) \right),
$$
$$
\dot{V}=x_1x_2\left( 1-x_1^2 \right)-2x_1^2-x_1x_2\left( 1-x_1^2 \right)-x_2^2\left( 1-x_1^2 \right)=-2x_1^2-x_2^2\left( 1-x_1^2 \right)
$$


Скобка должна быть неотрицательной:
$$
1-x_1^2\geq0\Rightarrow|x_1|\leq1
$$


Тогда:
$$
\forall\left( x_1,x_2 \right)\neq\left( 0,0 \right),|x_1|\leq1:\dot{V}<0
$$


В окрестности нуля $\dot{V}<0$, следовательно начало координат
локально асимптотически устойчиво.


\subsubsection{Четвертая система}
Рассмотрим систему:
\begin{align}
    \begin{cases}
        \dot{x}_1=-3x_1-x_2,\\
        \dot{x}_2=2x_1-x_2^3
    \end{cases}\label{syseq:4}
\end{align}


Функция Ляпунова:
$$
V=\frac{1}{2}\left( x_1^2+x_2^2 \right),\ V\left( x_1,x_2 \right)>0\, \forall \left( x_1,x_2 \right)\neq\left( 0,0 \right),\ V\left( 0,0 \right)=0
$$


Ее производная:
$$
\dot{V}=x_1\dot{x}_1+x_2\dot{x}_2=x_1\left( -3x_1-x_2 \right)+x_2\left( 2x_1-x_2^3 \right)=-3x_1^2+x_1x_2-x_2^4
$$


Найдем корни $\dot{V}=0$:
$$
-3x_1^2+x_1x_2-x_2^4=0\Rightarrow\left( 0,0 \right),\left( -\frac{1}{16},-\frac{1}{4} \right),\left( \frac{1}{16},\frac{1}{4} \right)
$$


Найдем точки равновесия системы (\ref{syseq:4}):
$$
\begin{cases}
    -3x_1-x_2=0,\\
    2x_1-x_2^3=0
\end{cases}\Rightarrow\left( 0,0 \right)
$$


Точки $\left( -\frac{1}{16},-\frac{1}{4} \right),\left( \frac{1}{16},\frac{1}{4} \right)$ не являются
равновесиями системы, из них траектории уходят. Следовательно, $\dot{V}=0$ в этих
точках не принадлежит инвариантному множеству.


Таким образом, наибольшее инвариантное множество, содержащееся в $\dot{V}=0$ -- это начало координат.


По теореме ЛаСалля начало координат асимптотически устойчиво.


По неравенству Коши-Буняковского:
$$
x_1x_2\leq\frac{1}{2}\left( x_1^2+x_2^2 \right)
$$


Тогда:
$$
\dot{V}\leq-3x_1^2+\frac{1}{2}x_1^2+\frac{1}{2}x_2^2-x_2^4=-\frac{5}{2}x_1^2+x_2^2\left( \frac{1}{2}-x_2^2 \right)
$$


Скобка должна быть отрицательной:
$$
\frac{1}{2}-x_2^2\leq0\Rightarrow x \in \left(-\infty,\,-\frac{\sqrt{2}}{2}\right) \,\vee\, \left(\frac{\sqrt{2}}{2},\,\infty\right)
$$


Таким образом, начало координат не может быть глобально
асимптотически устойчиво -- оно локально асимптотически устойчиво.


\subsubsection{Пятая система}
Рассмотрим систему:
\begin{align}
    \dot{x}=-\arctan{\left( x \right)}\label{syseq:5}
\end{align}


Функция Ляпунова:
$$
V(x)=\frac{1}{2}x^2,\ V(x)>0\,\forall x\neq0,\ V(0)=0
$$


Ее производная:
$$
\dot{V}=x\dot{x}=x\left( -\arctan{\left( x \right)} \right)=-x\arctan{\left( x \right)}
$$


Для всех $x\in\mathbb{R}$:
$$
\arctan{\left( x \right)}\in\left( -\frac{\pi}{2},\frac{\pi}{2} \right)
$$


и $\arctan{\left( x \right)}$ -- неубывающая нечетная функция. Она имеет тот же знак, что $x$.


Тогда:
\begin{align*}
    &x>0\Rightarrow \arctan{\left( x \right)}>0\Rightarrow \dot{V}=-x\arctan{\left( x \right)}<0\,\forall x\neq0,\\
    &x=0\Rightarrow \arctan{\left( x \right)}=0\Rightarrow\dot{V}=0,\\
    &x<0\Rightarrow \arctan{\left( x \right)}<0\Rightarrow \dot{V}=-x\arctan{\left( x \right)}<0\,\forall x\neq0
\end{align*}


Таким образом, для всех $x\neq0:\dot{V}<0$, следовательно
начало координат глобально асимптотически устойчиво.


\section{Задание 2}
\subsection{Условие}
Рассмотрим скалярную систему $\dot{x}=\alpha x^p+h(x)$,
где $p$ -- натуральное число, а $h(x)$ удовлетворяет
условию $|h(x)|\leq k|x|^{p+1}$ в некоторой
окрестности точки начала координат. При каких
условиях система асимптотически устойчива?


\subsection{Выполнение}
Положим влияние $h(x)$ незначительным при малых $x$
около начала координат, так как оно имеет порядок $x^{p+1}$
и стремится к нулю быстрее, чем $x^p$.


Рассмотрим упрощенную систему:
$$
\dot{x}=\alpha x^p
$$


При нечетном $p$ функция $x^p$ сохраняет
знак $x$:
\begin{align*}
    &x>0\Rightarrow x^p>0,\\
    &x<0\Rightarrow x^p<0
\end{align*}


Направление поля на прямой зависит от знака $\alpha$:
\begin{align*}
    &\alpha>0\Rightarrow \operatorname{sign}\left( \dot{x} \right)=\operatorname{sign}\left( x \right)\Rightarrow x>0:\dot{x}>0,\, x<0:\dot{x}<0\\
    &\alpha<0\Rightarrow \operatorname{sign}\left( \dot{x} \right)=-\operatorname{sign}\left( x \right)\Rightarrow x>0:\dot{x}<0,\, x<0:\dot{x}>0
\end{align*}


При $\alpha>0$ функция $x(t)$
или растет или убывает
в противоположном от нуля направлении
-- начало координат неустойчиво.


При $\alpha<0$ функция $x(t)$
либо растет к нулю слева, либо
убывает к нулю справа -- начало
координат асимптотически устойчиво.


При четном $p$ функция $x^p$ всегда
положительна.


Направления полей:
\begin{align*}
    &\alpha>0\Rightarrow \dot{x}=\alpha x^p>0\, \forall x\neq0\Rightarrow x>0:\dot{x}>0,\, x<0:\dot{x}>0\\
    &\alpha<0\Rightarrow \dot{x}=\alpha x^p<0\, \forall x\neq0\Rightarrow x>0:\dot{x}<0,\, x<0:\dot{x}<0
\end{align*}


При $\alpha>0$ функция
$x(t)$ приближается к нулю слева,
но справа расходится -- неустойчиво.


При $\alpha<0$ функция
$x(t)$ приближается к нулю справа,
но слева расходится -- неустойчиво.


Таким образом, $p$ должно быть нечетным, а $\alpha<0$.


Теперь учтем возмущение $h(x)$:
$$
\dot{x}=\alpha x^p+h(x)
$$


Оно мало в окрестности нуля.


Для $x\neq0$ перепишем:
$$
\dot{x}=x^p\left( \alpha+\frac{h(x)}{x^p} \right)
$$


Из условия $|h(x)|\leq k|x|^{p+1}$ следует:
$$
\bigg|\frac{h(x)}{x^p}\bigg|\leq k|x|
$$


Возьмем $\varepsilon>0$ такой, что:
$$
k\varepsilon<\frac{|\alpha|}{2},
$$
тогда для всех $|x|<\varepsilon$:
$$
\alpha+\frac{h(x)}{x^p}\leq \alpha+k|x|\leq\frac{\alpha}{2}<0
$$


Следовательно, для $|x|<\varepsilon$:
$$
\dot{x}\cdot\operatorname{sign}\left( x \right)=|x|^p\left( \alpha+\frac{h(x)}{x^p} \right)\leq\frac{\alpha}{2}|x|^p<0,
$$
то есть $|x|$ монотонно убывает вдоль траекторий,
пока решение остаётся в окрестности нуля.


Таким образом, $x(t)\xrightarrow[t\to\infty]{}0$ при малых $x$, нечетном $p$ и $\alpha<0$ -- начало координат локально асимптотически устойчиво.


\section{Задание 3}
\subsection{Условие}
На основе применения LMI построить линейный регулятор,
стабилизирующий систему экспоненциально со степенью 2:
\begin{align}
    \begin{cases}
        \dot{x}_1=x_2,\\
        \dot{x}_2=2x_1+u
    \end{cases}\label{syseq:7}
\end{align}


\subsection{Выполнение}
Приведем систему к виду:
$$
\dot{x}=Ax+Bu,\ u=Kx
$$


Матрицы $A,B$:
$$
A=\begin{bmatrix}
    0&1\\2&0
\end{bmatrix},\ B=\begin{bmatrix}
    0\\1
\end{bmatrix}
$$


Собственные числа матрицы $A$:
$$
\sigma\left( A \right)=\left\{ 1.41,-1.41 \right\}
$$


Проверим управляемость:
$$
A=PJP^{-1},B_J=P^{-1}B\Rightarrow J=\begin{bmatrix}
    1.41       &0\\
    0   &-1.41
\end{bmatrix},\ B_J=\begin{bmatrix}
    0.5\\0.5
\end{bmatrix}
$$


Система полностью управляема -- можно задать регулятору любую степень устойчивости.


Синтезируем LMI со степенью устойчивости $\alpha=2$ без ограничения на управление
при помощи матричного неравенства типа Ляпунова:
$$
PA^T+AP+2\alpha P+Y^TB^T+BY\preceq 0,\ K=YP^{-1}
$$


Получим:
$$
K=\begin{bmatrix}
    -39.02  &-10.81
\end{bmatrix}
$$


Проверим собственные числа замкнутой системы $A+BK$:
$$
\sigma\left( A+BK \right)=\left\{ -5.4 \pm 2.8i \right\}
$$


Замкнутая система асимптотически устойчива, и, убывает быстрее, чем
требуемая скорость $e^{-2t}$ (см. действительную часть собственных чисел).


Схема моделирования при измеримом $x$:
\begin{figure}[H]
    \centering
    \includegraphics[scale=0.6]{sch1.png}
    \caption{Схема моделирования системы}
    \label{fig:sch1}
\end{figure}


Графики вектора состояния системы и управления: % 1001,205,953,498
\begin{figure}[H]
    \centering
    \includegraphics[scale=0.5]{1.png}
    \caption{Вектор состояния объекта управления}
    \label{fig:1}
\end{figure}
\begin{figure}[H]
    \centering
    \includegraphics[scale=0.5]{2.png}
    \caption{Управление}
    \label{fig:2}
\end{figure}


Регулятор работает корректно.


\section{Задание 4}
\subsection{Условие}
Найти ограничивающее условие на параметр $\gamma$, при котором
система является экспоненциально устойчивой со степенью 1. Закон
управления взять из предыдущего задания.
\begin{align}
    \begin{cases}
        \dot{x}_1=x_2+\gamma\sin{\left( x_2 \right)},\\
        \dot{x}_2=2x_1+u,
    \end{cases}\label{syseq:8}
\end{align}
$$
u=Kx,\ K=\begin{bmatrix}
    -39.02  &-10.81
\end{bmatrix}
$$


\subsection{Выполнение}
Подставим закон управления:
$$
\begin{cases}
    \dot{x}_1=x_2+\gamma\sin{\left( x_2 \right)},\\
    \dot{x}_2=2x_1+Kx=2x_1-39.02x_1-10.81x_2
\end{cases}
$$


Получим систему:
$$
\begin{cases}
    \dot{x}_1=x_2+\gamma\sin{\left( x_2 \right)},\\
    \dot{x}_2=-37.02x_1-10.81x_2
\end{cases}
$$


Замкнутая система будет иметь вид:
$$
\dot{x}=\left( A+BK \right)x+\gamma f(x),
$$
$$
A+BK=\begin{bmatrix}
    0&1\\-37.02&-10.81
\end{bmatrix},\ f(x)=\begin{bmatrix}
    \sin{\left( x_2 \right)}\\0
\end{bmatrix}
$$


Линеаризуем систему. При малых $x_2$:
$$
\sin{\left( x_2 \right)}\approx x_2
$$


Тогда система примет вид:
$$
\dot{x}=\left[ \left( A+BK \right)+\gamma\begin{bmatrix}
    0&1\\0&0
\end{bmatrix} \right]x=A_\gamma x,
$$
$$
A_\gamma=\begin{bmatrix}
    0&1+\gamma\\-37.02&-10.81
\end{bmatrix}
$$


Экспоненциальная устойчивость степени 1 требует:
$$
\operatorname{Re}\left( \lambda_{A_\gamma\, i} \right)\leq-1
$$


Характеристический полином:
$$
\det{\left(\lambda I-A_\gamma\right)}=\lambda^2+10.81\lambda+37.02\left( \gamma+1 \right)=0
$$


Собственные числа:
$$
\lambda_{1,2}=\frac{-10.81\pm\sqrt{10.81^2-4\cdot37.02\left( \gamma+1 \right)}}{2}
$$


Рассмотрим дискриминант:
$$
D=10.81^2-4\cdot37.02\left( \gamma+1 \right)
$$


Если $D<0$, то корни комплексно-сопряженные с отрицательной действительной частью $-10.81/2<-1$:
$$
10.81^2-4\cdot37.02\left( \gamma+1 \right)<0\Rightarrow \gamma>\frac{10.81^2}{4\cdot37.02}-1\Rightarrow\gamma>-0.21
$$


Если $D\geq0$, то корни действительные и должны удовлетворять условию:
$$
\lambda_i=\frac{-10.81\pm\sqrt{10.81^2-4\cdot37.02\left( \gamma+1 \right)}}{2}\leq-1,
$$


Так как $\lambda_{1+}\geq\lambda_{2-}$, то достаточно проверить $\lambda_{1+}$:
$$
\sqrt{10.81^2-4\cdot37.02\left( \gamma+1 \right)}\leq8.81\Rightarrow\gamma\geq\dfrac{8.81^2-10.81^2}{-4\cdot37.02}-1\Rightarrow\gamma\geq-0.74
$$


Таким образом, экспоненциальная устойчивость степени 1 достигается при:
$$
\gamma\geq-0.74
$$


При $\gamma>-0.21$ собственные числа будут комплексными.


\section{Задание 5}
\subsection{Условие}
Рассмотрим систему:
\begin{align}
    \begin{cases}
        \dot{x}_1=x_2-0.5x_1^3,\\
        \dot{x}_2=u,\\
        u=Kx
    \end{cases}\label{syseq:9}
\end{align}


Весь вектор состояния $x=\begin{bmatrix}
    x_1&x_2
\end{bmatrix}^T$ измерим.


\begin{enumerate}
    \item Синтезируйте линейный регулятор с обратной связью по
    состоянию, чтобы глобально стабилизировать начало координат.
    \item Исследуйте устойчивость по входу к состоянию при наличии шумов измерений.
    \item Исследуйте устойчивость по входу к состоянию при наличии аддитивных возмущений.
\end{enumerate}


\subsection{Выполнение}
...


\section{Задание 6}
\subsection{Условие}
Исследуйте устойчивость по входу к состоянию системы по
отношению к возмущению $d$:
\begin{align}
    \begin{cases}
        \dot{x}_1=-2x_1+x_2,\\
        \dot{x}_2=-x_1-\sigma\left( x_1 \right)-x_2+d,
    \end{cases}\label{syseq:10}
\end{align}
где $\sigma$ -- локально липшицева, $\sigma\left( 0 \right)=0,\, y\sigma\left( y \right)\geq0$.


\subsection{Выполнение}
...


\section{Вывод}
...
\end{document}