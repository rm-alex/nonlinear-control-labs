\documentclass[a4paper,14pt]{extarticle}

\usepackage[T2A]{fontenc}
\usepackage[utf8]{inputenc}
\usepackage[english, russian]{babel}

\usepackage[left=30mm, right=10mm, top=20mm, bottom=20mm]{geometry}

\usepackage{tempora}
\usepackage{setspace}
\onehalfspacing

\usepackage{titlesec}
\titleformat{\section}[block]{\bfseries\centering\MakeUppercase}{\thesection.}{1em}{}
\titleformat{\subsection}[block]{\bfseries}{\thesubsection.}{1em}{}
\titleformat{\subsubsection}[block]{\bfseries}{\thesubsubsection.}{1em}{}

\renewcommand{\contentsname}{\hfill \textbf{СОДЕРЖАНИЕ} \hfill\null}

\usepackage{indentfirst}
\setlength{\parindent}{1.25cm}

\usepackage{amsmath, amsfonts, amssymb}
\usepackage{graphicx}
\usepackage{caption}
\usepackage{subcaption}
\usepackage{float}
\usepackage{tikz}
\usetikzlibrary{patterns}
\usepackage{cmap}
\usepackage{hyperref}
\usepackage{xcolor}
\usepackage{listings}

\definecolor{LightGray}{gray}{0.7}

\lstdefinestyle{code}{
    language=Python, % change if needed
    basicstyle=\small\ttfamily,
    numbers=left,
    numberstyle=\small\color{LightGray},
    stepnumber=1,
    numbersep=5pt,
    backgroundcolor=\color{white},
    showspaces=false,
    showstringspaces=false,
    showtabs=false,
    tabsize=4,
    captionpos=b,
    breaklines=true,
    breakatwhitespace=false,
    frame=single,
    rulecolor=\color{LightGray},
    linewidth=\linewidth,
    keywordstyle=\color{blue}\bfseries,
    commentstyle=\color{green!40!black},
    stringstyle=\color{violet},
    escapeinside={\%*}{*)},
    xleftmargin=10pt,
    xrightmargin=10pt,
    framexleftmargin=0pt,
    framexrightmargin=0pt
}
\lstset{style=code}

\hypersetup{
    colorlinks=true,
    linkcolor=blue,
    filecolor=magenta,
    urlcolor=cyan,
    pdftitle={ncs1},
    pdfauthor={Rumyantsev Alexey},
    pdfsubject={control},
    pdfkeywords={LaTeX, PDF},
    pdfpagemode=FullScreen,
}

\graphicspath{{src/images/}}

\begin{document}

\begin{titlepage}
    \begin{center}
        МИНИСТЕРСТВО НАУКИ И ВЫСШЕГО ОБРАЗОВАНИЯ РОССИЙСКОЙ ФЕДЕРАЦИИ\\
        \vspace*{2.5mm}
        Федеральное государственное автономное образовательное учреждение высшего образования
        «НАЦИОНАЛЬНЫЙ ИССЛЕДОВАТЕЛЬСКИЙ УНИВЕРСИТЕТ ИТМО»\\
        \vspace*{2.5mm}
        \textbf{ФАКУЛЬТЕТ СИСТЕМ УПРАВЛЕНИЯ И РОБОТОТЕХНИКИ}
        \vfill

        {\large ОТЧЕТ ПО ЛАБОРАТОРНОЙ РАБОТЕ №3}\\
        {\large по дисциплине}\\
        {\large\bfseries «НЕЛИНЕЙНЫЕ СИСТЕМЫ УПРАВЛЕНИЯ»}\\
        {\large на тему}\\
        {\large\bfseries «ЛИНЕАРИЗАЦИЯ ПО ВХОДУ–ВЫХОДУ И ПРОЕКТИРОВАНИЕ ОБРАТНОЙ СВЯЗИ ПО СОСТОЯНИЮ»}\\
        \vfill

        \begin{flushright}
            Выполнил: студент гр. R3441\\
            Румянцев А. А.\medskip\\

            Проверил: преподаватель\\
            Зименко К. А.
        \end{flushright}

        \vfill

        Санкт-Петербург\\
        2025
    \end{center}
\end{titlepage}

\setcounter{page}{2}
\tableofcontents
\newpage

\section{Задание 1}
\subsection{Условие}
Для данной системы определить:
\begin{itemize}
    \item Является ли эта система линеаризуемой по входу–выходу?
    \item Если да, преобразуйте её в нормальную форму и укажите область определения соответствующего преобразования
    \item Является ли эта система минимально-фазовой?
\end{itemize}


\subsection{Выполнение}
Рассмотрим систему:
\begin{align}
    \begin{cases}
        \dot{x}_1=-x_1+x_2-x_3,\\
        \dot{x}_2=-x_1x_3-x_2+u,\\
        \dot{x}_3=-x_1+u,\\
        y=x_3
    \end{cases}\label{syseq:1}
\end{align}


Нелинейная система:
$$
\dot{x}=f(x)+g(x)u
$$


Производная от выхода $y=h(x)$:
$$
\dot{y}=L_fh(x)+L_gh(x)u
$$


Производная Ли функции h вдоль поля f:
$$
L_fh(x)=\frac{\partial h}{\partial x}f(x)=\begin{bmatrix}
    0&0&1
\end{bmatrix}\begin{bmatrix}
    -x_1+x_2-x_3\\-x_1x_3-x_2\\-x_1
\end{bmatrix}=-x_1
$$


Вдоль поля g:
$$
L_gh(x)=\frac{\partial h}{\partial x}g(x)=\begin{bmatrix}
    0&0&1
\end{bmatrix}\begin{bmatrix}
    0\\1\\1
\end{bmatrix}=1
$$


Производная выхода:
$$
\dot{y}=\dot{x}_3=-x_1+u
$$


Первая производная выхода зависит от $u$,
следовательно относительная степень
нелинейной системы $\rho=1$.


Так как $L_gL_f^{\rho-1}h(x)=L_gh(x)=1\neq0$, $\rho<2$ и $\rho=1<n=3$, то система линеаризуема по входу–выходу
(вход появляется именно в $\rho$-й производной; так как производная потребовалась всего одна, то
условие на непоявление входа раньше $\rho$-й производной $L_gL_f^kh(x)=0,k\in\left[ 0,\rho-2 \right]$ проверять не нужно;
$\rho\leq n$ -- относительная степень не больше порядка системы).


Обратная связь:
$$
u=\frac{-L_f^{\rho}h(x)+v}{L_gL_f^{\rho-1}h(x)}=\frac{-L_fh(x)+v}{L_gh(x)}=x_1+v
$$


Подставим полученное $u$ в производную выхода:
$$
\dot{y}=-x_1+\left( x_1+v \right)=v
$$


Динамика между $v$ и $y$ сведена к одному интегратору.


\section{Задание 2}
\subsection{Условие}
На основе метода линеаризации обратной связью найдите закон
управления с обратной связью по состоянию, обеспечивающий
глобальную стабилизацию начала координат для данной системы.


\subsection{Выполнение}
Рассмотрим систему:
\begin{align}
    \begin{cases}
        \dot{x}_1=-x_1+x_2,\\
        \dot{x}_2=x_1-x_2-x_1x_3+u,\\
        \dot{x}_3=x_1+x_1x_2-2x_3
    \end{cases}\label{syseq:2}
\end{align}
\end{document}