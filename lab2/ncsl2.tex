\documentclass[a4paper,14pt]{extarticle}

\usepackage[T2A]{fontenc}
\usepackage[utf8]{inputenc}
\usepackage[english, russian]{babel}

\usepackage[left=30mm, right=10mm, top=20mm, bottom=20mm]{geometry}

\usepackage{tempora}
\usepackage{setspace}
\onehalfspacing

\usepackage{titlesec}
\titleformat{\section}[block]{\bfseries\centering\MakeUppercase}{\thesection.}{1em}{}
\titleformat{\subsection}[block]{\bfseries}{\thesubsection.}{1em}{}
\titleformat{\subsubsection}[block]{\bfseries}{\thesubsubsection.}{1em}{}

\renewcommand{\contentsname}{\hfill \textbf{СОДЕРЖАНИЕ} \hfill\null}

\usepackage{indentfirst}
\setlength{\parindent}{1.25cm}

\usepackage{amsmath, amsfonts, amssymb}
\usepackage{graphicx}
\usepackage{caption}
\usepackage{subcaption}
\usepackage{float}
\usepackage{tikz}
\usetikzlibrary{patterns}
\usepackage{cmap}
\usepackage{hyperref}
\usepackage{xcolor}
\usepackage{listings}

\definecolor{LightGray}{gray}{0.7}

\lstdefinestyle{code}{
    language=Python, % change if needed
    basicstyle=\small\ttfamily,
    numbers=left,
    numberstyle=\small\color{LightGray},
    stepnumber=1,
    numbersep=5pt,
    backgroundcolor=\color{white},
    showspaces=false,
    showstringspaces=false,
    showtabs=false,
    tabsize=4,
    captionpos=b,
    breaklines=true,
    breakatwhitespace=false,
    frame=single,
    rulecolor=\color{LightGray},
    linewidth=\linewidth,
    keywordstyle=\color{blue}\bfseries,
    commentstyle=\color{green!40!black},
    stringstyle=\color{violet},
    escapeinside={\%*}{*)},
    xleftmargin=10pt,
    xrightmargin=10pt,
    framexleftmargin=0pt,
    framexrightmargin=0pt
}
\lstset{style=code}

\hypersetup{
    colorlinks=true,
    linkcolor=blue,
    filecolor=magenta,
    urlcolor=cyan,
    pdftitle={ncs1},
    pdfauthor={Rumyantsev Alexey},
    pdfsubject={control},
    pdfkeywords={LaTeX, PDF},
    pdfpagemode=FullScreen,
}

\graphicspath{{src/images/}}

\begin{document}

\begin{titlepage}
    \begin{center}
        МИНИСТЕРСТВО НАУКИ И ВЫСШЕГО ОБРАЗОВАНИЯ РОССИЙСКОЙ ФЕДЕРАЦИИ\\
        \vspace*{2.5mm}
        Федеральное государственное автономное образовательное учреждение высшего образования
        «НАЦИОНАЛЬНЫЙ ИССЛЕДОВАТЕЛЬСКИЙ УНИВЕРСИТЕТ ИТМО»\\
        \vspace*{2.5mm}
        \textbf{ФАКУЛЬТЕТ СИСТЕМ УПРАВЛЕНИЯ И РОБОТОТЕХНИКИ}
        \vfill

        {\large ОТЧЕТ ПО ЛАБОРАТОРНОЙ РАБОТЕ №2}\\
        {\large по дисциплине}\\
        {\large\bfseries «НЕЛИНЕЙНЫЕ СИСТЕМЫ УПРАВЛЕНИЯ»}\\
        {\large на тему}\\
        {\large\bfseries «ФУНКЦИЯ ЛЯПУНОВА, УСТОЙЧИВОСТЬ И РЕГУЛЯТОРЫ»}\\
        \vfill

        \begin{flushright}
            Выполнил: студент гр. R3441\\
            Румянцев А. А.\medskip\\

            Проверил: преподаватель\\
            Зименко К. А.
        \end{flushright}

        \vfill

        Санкт-Петербург\\
        2025
    \end{center}
\end{titlepage}

\setcounter{page}{2}
\tableofcontents
\newpage

\section{Задание 1}
\subsection{Условие}
Для каждой из данных систем используйте кандидат
квадратичной функции Ляпунова, чтобы показать, что начало
координат асимптотически устойчиво.


\subsection{Выполнение}
\subsubsection{Первая система}
Рассмотрим систему:
\begin{align}
    \begin{cases}
        \dot{x}_1=-x_1+x_1x_2,\\
        \dot{x}_2=-2x_2
    \end{cases}
\end{align}\label{syseq:1}


Функция Ляпунова в квадратичной форме:
$$
V(x)=x^TPx,\text{ пусть } P=\begin{bmatrix}
    0.5&0\\0&0.5
\end{bmatrix}\Rightarrow V(x)=\begin{bmatrix}
    x_1&x_2
\end{bmatrix}\begin{bmatrix}
    0.5&0\\0&0.5
\end{bmatrix}\begin{bmatrix}
    x_1\\x_2
\end{bmatrix},
$$
$$
V=\frac{1}{2}\left( x_1^2+x_2^2 \right),\ V\left( x_1,x_2 \right)>0\, \forall \left( x_1,x_2 \right)\neq\left( 0,0 \right),\ V\left( 0,0 \right)=0
$$


Ее производная:
$$
\dot{V}=x_1\dot{x}_1+x_2\dot{x}_2=x_1\left( -x_1+x_1x_2 \right)+x_2\left( -2x_2 \right)=-x_1^2+x_1^2x_2-2x_2^2,
$$
$$
\dot{V}=-x_1^2\left( 1-x_2 \right)-2x_2^2
$$


Скобка должна быть неотрицательной:
$$
1-x_2\geq0\Rightarrow x_2\leq1
$$


Тогда:
$$
\forall\left( x_1,x_2 \right)\neq\left( 0,0 \right),x_2\leq1:\dot{V}<0
$$


В окрестности нуля $\dot{V}<0$, следовательно начало координат локально асимптотически устойчиво.


Глобальная асимптотическая устойчивость достигалась бы в случае,
когда $\forall \left( x_1,x_2 \right)\in\mathbb{R}^2\,\backslash\left\{ 0 \right\}:\dot{V}<0$.


\subsubsection{Вторая система}
Рассмотрим систему:
\begin{align}
    \begin{cases}
        \dot{x}_1=-x_2-x_1\left( 1-x_1^2-x_2^2 \right),\\
        \dot{x}_2=x_1-x_2\left( 1-x_1^2-x_2^2 \right)
    \end{cases}\label{syseq:2}
\end{align}


Функция Ляпунова:
$$
V=\frac{1}{2}\left( x_1^2+x_2^2 \right),\ V\left( x_1,x_2 \right)>0\, \forall \left( x_1,x_2 \right)\neq\left( 0,0 \right),\ V\left( 0,0 \right)=0
$$


Ее производная:
$$
\dot{V}=x_1\dot{x}_1+x_2\dot{x}_2=x_1\left( -x_2-x_1\left( 1-x_1^2-x_2^2 \right) \right)+x_2\left( x_1-x_2\left( 1-x_1^2-x_2^2 \right) \right),
$$
$$
\dot{V}=x_1^4+x_2^4+2x_1^2x_2^2-x_1^2-x_2^2=\left( x_1^2+x_2^2 \right)\left( x_1^2+x_2^2 -1\right)
$$


Сделаем замену $r^2=x_1^2+x_2^2$:
$$
\dot{V}=r^2\left( r^2-1 \right)
$$


Тогда:
\begin{align*}
    &r^2-1<0\Rightarrow r\in\left( -1,1 \right):\dot{V}<0,\\
    &r^2-1=0\Rightarrow r=\pm 1:\dot{V}=0,\\
    &r^2-1>0\Rightarrow r\in\left( -\infty,-1 \right)\vee \left( 1,\infty \right):\dot{V}>0
\end{align*}


Внутри единичной окружности $r^2=x_1^2+x_2^2<1$ начало координат
локально асимптотически устойчиво.


Локальная асимптотическая устойчивость исключает глобальную.


\subsubsection{Третья система}
Рассмотрим систему:
\begin{align}
    \begin{cases}
        \dot{x}_1=x_2\left( 1-x_1^2 \right)-2x_1,\\
        \dot{x}_2=-\left( x_1+x_2 \right)\left( 1-x_1^2 \right)
    \end{cases}\label{syseq:3}
\end{align}


Функция Ляпунова:
$$
V=\frac{1}{2}\left( x_1^2+x_2^2 \right),\ V\left( x_1,x_2 \right)>0\, \forall \left( x_1,x_2 \right)\neq\left( 0,0 \right),\ V\left( 0,0 \right)=0
$$


Ее производная:
$$
\dot{V}=x_1\dot{x}_1+x_2\dot{x}_2=x_1\left( x_2\left( 1-x_1^2 \right)-2x_1 \right)+x_2\left( -\left( x_1+x_2 \right)\left( 1-x_1^2 \right) \right),
$$
$$
\dot{V}=x_1x_2\left( 1-x_1^2 \right)-2x_1^2-x_1x_2\left( 1-x_1^2 \right)-x_2^2\left( 1-x_1^2 \right)=-2x_1^2-x_2^2\left( 1-x_1^2 \right)
$$


Скобка должна быть неотрицательной:
$$
1-x_1^2\geq0\Rightarrow|x_1|\leq1
$$


Тогда:
$$
\forall\left( x_1,x_2 \right)\neq\left( 0,0 \right),|x_1|\leq1:\dot{V}<0
$$


В окрестности нуля $\dot{V}<0$, следовательно начало координат
локально асимптотически устойчиво.


\subsubsection{Четвертая система}
Рассмотрим систему:
\begin{align}
    \begin{cases}
        \dot{x}_1=-3x_1-x_2,\\
        \dot{x}_2=2x_1-x_2^3
    \end{cases}\label{syseq:4}
\end{align}
\end{document}