\documentclass[a4paper,14pt]{extarticle}

\usepackage[T2A]{fontenc}
\usepackage[utf8]{inputenc}
\usepackage[english, russian]{babel}

\usepackage[left=30mm, right=10mm, top=20mm, bottom=20mm]{geometry}

\usepackage{tempora}
\usepackage{setspace}
\onehalfspacing

\usepackage{titlesec}
\titleformat{\section}[block]{\bfseries\centering\MakeUppercase}{\thesection.}{1em}{}
\titleformat{\subsection}[block]{\bfseries}{\thesubsection.}{1em}{}
\titleformat{\subsubsection}[block]{\bfseries}{\thesubsubsection.}{1em}{}

\renewcommand{\contentsname}{\hfill \textbf{СОДЕРЖАНИЕ} \hfill\null}

\usepackage{indentfirst}
\setlength{\parindent}{1.25cm}

\usepackage{amsmath, amsfonts, amssymb}
\usepackage{graphicx}
\usepackage{caption}
\usepackage{subcaption}
\usepackage{float}
\usepackage{tikz}
\usetikzlibrary{patterns}
\usepackage{cmap}
\usepackage{hyperref}
\usepackage{xcolor}
\usepackage{listings}

\definecolor{LightGray}{gray}{0.7}

\lstdefinestyle{code}{
    language=Python, % change if needed
    basicstyle=\small\ttfamily,
    numbers=left,
    numberstyle=\small\color{LightGray},
    stepnumber=1,
    numbersep=5pt,
    backgroundcolor=\color{white},
    showspaces=false,
    showstringspaces=false,
    showtabs=false,
    tabsize=4,
    captionpos=b,
    breaklines=true,
    breakatwhitespace=false,
    frame=single,
    rulecolor=\color{LightGray},
    linewidth=\linewidth,
    keywordstyle=\color{blue}\bfseries,
    commentstyle=\color{green!40!black},
    stringstyle=\color{violet},
    escapeinside={\%*}{*)},
    xleftmargin=10pt,
    xrightmargin=10pt,
    framexleftmargin=0pt,
    framexrightmargin=0pt
}
\lstset{style=code}

\hypersetup{
    colorlinks=true,
    linkcolor=blue,
    filecolor=magenta,
    urlcolor=cyan,
    pdftitle={ncs3},
    pdfauthor={Rumyantsev Alexey},
    pdfsubject={control},
    pdfkeywords={LaTeX, PDF},
    pdfpagemode=FullScreen,
}

\graphicspath{{src/images/}}

\begin{document}

\begin{titlepage}
    \begin{center}
        МИНИСТЕРСТВО НАУКИ И ВЫСШЕГО ОБРАЗОВАНИЯ РОССИЙСКОЙ ФЕДЕРАЦИИ\\
        \vspace*{2.5mm}
        Федеральное государственное автономное образовательное учреждение высшего образования
        «НАЦИОНАЛЬНЫЙ ИССЛЕДОВАТЕЛЬСКИЙ УНИВЕРСИТЕТ ИТМО»\\
        \vspace*{2.5mm}
        \textbf{ФАКУЛЬТЕТ СИСТЕМ УПРАВЛЕНИЯ И РОБОТОТЕХНИКИ}
        \vfill

        {\large ОТЧЕТ ПО ЛАБОРАТОРНОЙ РАБОТЕ №6}\\
        {\large по дисциплине}\\
        {\large\bfseries <<НЕЛИНЕЙНЫЕ СИСТЕМЫ УПРАВЛЕНИЯ>>}\\
        {\large на тему}\\
        {\large\bfseries <<СИНТЕЗ ФИНИТНОГО ОДНОРОДНОГО РЕГУЛЯТОРА>>}\\
        \vfill

        \begin{flushright}
            Выполнили: студенты\\
            Румянцев А. А., R3441\\
            Дьячихин Д. Н., R3480\medskip\\

            Проверил: преподаватель\\
            Зименко К. А.
        \end{flushright}

        \vfill

        Санкт-Петербург\\
        2025
    \end{center}
\end{titlepage}

\setcounter{page}{2}
\tableofcontents
\newpage
\section{Задание}
\subsection{Условие}
\begin{enumerate}
    \item Выберите нелинейную управляемую систему с одним входом, одним выходом;
    \item Предполагая, что математическая модель содержит существенные неопределенности, представьте систему в виде цепи интеграторов с согласованной обобщенной неизвестной динамикой, используя безмодельный подход;
    \item Реализуйте финитный однородный регулятор с дополнительной компенсацией обобщенной неизвестной динамики.
\end{enumerate}


\subsection{Выполнение}
Рассмотрим нелинейную систему
с одним входом и одним выходом
-- несимметричный маятник с сухим и вязким трением:
$$
\ddot{y}=-\frac{g}{l}\sin{y}-\frac{b}{ml^2}\dot{y}-\frac{c}{ml^2}\operatorname{sign}(\dot{y})+\frac{1}{ml^2}u,
$$
где $y$ угол отклонения от нижнего положения,
$|u|\leq u_{\max}$ момент на оси, $m,l$
масса и длина стержня, $g$ ускорение
свободного падения, $b$ коэффициент вязкого
трения, $c$ коэффициент сухого трения.


Обозначим момент инерции $I=ml^2>0$. Перепишем систему в виде:
$$
\ddot{y}=d(t,y,\dot{y})+\frac{1}{I}u,
$$
где:
$$
d(t,y,\dot{y})=-\frac{g}{l}\sin{y}-\frac{b}{I}\dot{y}-\frac{c}{I}\operatorname{sign}(\dot{y})
$$
обобщенная неизвестная динамика, содержащая существенные неопределенности.
В безмодельном подходе считаем $d(\cdot)$ неизвестной,
но ограниченной или растущей не быстрее некоторой однородной функции.


Введем состояние:
$$
x_1=y,\ x_2=\dot{y},
$$
тогда, система примет канонический вид цепи интеграторов с возмущением:
$$
\dot{x}_1=x_2,\ \dot{x}_2=d(t,x)+\frac{1}{I}u
$$


Выберем веса матрицы расширения $D(\lambda>0)$:
$$
r_1=1,\ r_2=1-\rho,\ \rho\in(0,1),
$$
что соответствует отрицательной степени однородности $-\rho$
для цепи интеграторов $\dot{x}_1=x_2,\dot{x}_2=v$.
При таком выборе переменные связаны масштабированием:
$$
x_2\sim x_1^{r_2/r_1}=x_1^{1-\rho}
$$


Определим нелинейную скользящую поверхность:
$$
s(x)=x_2+\gamma\,[x_1]^{1-\rho},
$$
где $[s]^\alpha=|s|^\alpha\operatorname{sign}(s)$ -- однородная нелинейность,
$\gamma>0$ коэффициент усиления.


Поверхность $s(x)$ взвешенно-однородна степени
$r_2=1-\rho$ относительно матрицы расширения $D(\lambda)=\operatorname{diag}(\lambda^{r_1},\lambda^{r_2})$,
что обеспечивает однородность замкнутой системы при соответствующем выборе управления.


Для системы с возмущением $|d(t,x)|\leq\bar{d}$
стабилизирующий регулятор имеет вид:
$$
u=-I\left[ \hat{d}(t)+k \left[ s(x) \right]^\rho \right],\ k>0,\rho\in(0,1),
$$
где $\hat{d}(t)$ оценка (компенсация) неизвестной динамики
$d(t,x)$, $[s]^\rho=|s|^\rho\operatorname{sign}(s)$ обеспечивает
конечное время сходимости:
$$
\sigma=\rho<1\Rightarrow T(x_0)\leq\frac{V_0^{1-\rho}}{\beta(1-\rho)}<\infty,\ V_0:Q(V_0,x_0)=0
$$


Для уменьшения дрожания в управлении заменим $\operatorname{sign}(s)$
на $\tanh(s/\varepsilon)$, $\varepsilon>0$. Тогда, закон управления:
$$
u(t)=-I\left[ \hat{d}(t)+k\left|s(t)\right|^\rho\tanh(s(t)/\varepsilon) \right],
$$
где:
$$
s(t)=\dot{y}(t)+\gamma\operatorname{sign}(y(t))|y(t)|^{1-\rho},\ \hat{d}(t)\approx d(t,y,\dot{y})
$$


Так как $\ddot{y}=d+u/I$, то:
$$
d=\ddot{y}-u/I
$$
Построим оценку $\hat{d}$ на основе двухступенчатой фильтрации
для вычисления $\hat{\dot{y}},\hat{\ddot{y}}$ из измерения $y$:
$$
\hat{\dot{y}}(t)=\frac{2}{2\tau_1+\Delta t}\left[ y(t)-y(t-\Delta t) \right]-\frac{2\tau_1-\Delta t}{2\tau_1+\Delta t}\hat{\dot{y}}\left( t-\Delta t \right),
$$
$$
\hat{\ddot{y}}(t)=\frac{2}{2\tau_2+\Delta t}\left[ \hat{\dot{y}}(t)-\hat{\dot{y}}(t-\Delta t) \right]-\frac{2\tau_2-\Delta t}{2\tau_2+\Delta t}\hat{\ddot{y}}\left( t-\Delta t \right)
$$
-- билинейная дискретизация фильтров первого порядка.


ФНЧ-сглаживания оценки:
$$
\hat{d}(t)=(1-\beta_F)\hat{d}( t-\Delta t )+\beta_F\left[ \hat{\ddot{y}}-u(t-\Delta t)/I \right],\ \beta_F=\frac{\Delta t}{\tau_F+\Delta t},
$$
что обеспечивает ограниченную погрешность $||d-\hat{d}||_\infty\leq\delta$.
Для решения задачи достаточно, чтобы ошибка была согласованной.


Выполним моделирование системы:
\end{document}